\documentclass{beamer}

\def\Tiny{\fontsize{6pt}{6pt}\selectfont}
\def\supertiny{\fontsize{4pt}{4pt}\selectfont}

\mode<presentation>
{
  \usetheme{Warsaw}
  % \setbeamercovered{transparent}
  \usecolortheme{crane}
}

\usepackage{graphicx, ifthen, listings, fancyvrb}

\usepackage[czech]{babel}
% \usefonttheme{professionalfonts}
\usepackage{times}
\usepackage{amsmath}
\usepackage[utf8]{inputenc}
\usepackage{wrapfig}

\usepackage[T1]{fontenc}

\lstset{ basicstyle=\tiny, stringstyle=\ttfamily, showstringspaces=false }

\everymath{\displaystyle}

\setbeamerfont{frametitle}{size=\large}
\setbeamerfont{subsection in toc}{size=\scriptsize}

\makeatletter\newenvironment{blackbox}{%
   \begin{lrbox}{\@tempboxa}\begin{minipage}{0.95\columnwidth}}{\end{minipage}\end{lrbox}%
   \colorbox{black}{\usebox{\@tempboxa}}
}\makeatother

\title[IMF (8)]{Informatika pro moderní fyziky (8)\\Regulární výrazy. Tvorba textových dokumentů.}

\author[Franti\v{s}ek HAVL\r{U}J, ORF ÚJV Řež]{Franti\v{s}ek HAVL\r{U}J\\{\scriptsize \emph{e-mail: haf@ujv.cz}}}

\date{akademický rok 2020/2021\\23. listopadu 2020}

\institute[ORF ÚJV Řež]
{ÚJV Řež\\oddělení Reaktorové fyziky a podpory palivového cyklu}

\AtBeginSection[]
{
\begin{frame}<beamer>
\frametitle{Obsah}
\tableofcontents[currentsection,hideothersubsections]
\end{frame}
}

\begin{document}

\begin{frame}
  \titlepage
\end{frame}

\begin{frame}
  \tableofcontents
\end{frame}

\section{Co jsme se naučili minule}

\begin{frame}{}
  \begin{itemize}
    \item generování textového dokumentu
    \item tabulky v latexu
  \end{itemize}
\end{frame}

\section{Najdi to, nevím co: regex}

\begin{frame}{Úkoly}
  \begin{itemize}
    \item najít všechna celá čísla: jaké je nejmenší a největší číslo a kolik jich je
    \item najít všechny emailové adresy: vypište je do souboru seskupené podle domény
    \item vlnka: nedělitelné mezery v textu před k, s, v, z (výsledek vypište do souboru)
    \item nahradit desetinné tečky čárkami (výsledek vypište do souboru)
  \end{itemize}
\end{frame}

\begin{frame}{Jak na to: regulární výrazy (regexy)}
  \begin{itemize}
    \item můžu definovat chytrou `masku'
    \item nejjednodušší: /text/ - text
    \item skupiny písmen: /1[abc]/ - 1a, 1b ...
    \item speciální skupiny: /\\da/ - 1a, 2a, 3a ...
    \item opakování: +, *
    \item vytažení části výrazu do tzv. capture group -- stačí uzávorkovat
    \item hračka a vysvětlení: rubular.com
  \end{itemize}
\end{frame}

\begin{frame}{Regexy v Ruby}
  \begin{itemize}
    \item \texttt{String\#match} vrací MatchData objekt, m[0] je ten samotný řetězec, m[i] jsou skupiny
    \item \texttt{String\#gsub} nahrazuje; můžu použít s blokem, nicméně tam nemám capture groups (ale můžu použít \texttt{Regexp.last\_match})
    \item pokud použiju formu bez bloku, mám capture groups v \textbackslash \textbackslash 1 a \textbackslash \textbackslash 2
    \item \texttt{String\#scan} vrací pole všech výskytů
  \end{itemize}
\end{frame}


\section{Na šablony chytře}

\begin{frame}{Co se nám nelíbí na generování dokumentu}
  \begin{itemize}
    \item je to hrozně roztahané uvnitř zdrojáku
    \item je to dost nepřehledné
    \item nevidíme strukturu texového dokumentu
    \item představte si složitější dokument...
  \end{itemize}
\end{frame}

\begin{frame}{ERb (Embedded Ruby)}
  \begin{itemize}
    \item lepší šablona - ``aktivní text''
    \item používá se například ve webových aplikacích
    \item hodí se ale i na generování latexových dokumentů, resp. všude, kde nám nesejde na whitespace
    \item poměrně jednoduchá syntaxe, zvládne skoro všechno
  \end{itemize}
\end{frame}

\begin{frame}[fragile]{Základní syntaxe ERb (1)}
  \begin{block}{ }
    Jakýkoli Ruby příkaz, přiřazení, výpočet ...
    \scriptsize
    \begin{verbatim}
      <% a = b + 5 %>
      <% list = ary * ", " %>
    \end{verbatim}
  \end{block}
\end{frame}

\begin{frame}[fragile]{Základní syntaxe ERb (2)}
  \begin{block}{ }
    Pokud chci něco vložit, stačí přidat rovnítko
    \scriptsize
    \begin{verbatim}
      <%= a %>
      <%= ary[1] %>
      <%= b + 5 %>
    \end{verbatim}
  \end{block}
\end{frame}

\begin{frame}[fragile]{Základní syntaxe ERb (3)}
  \begin{block}{ }
    Radost je možnost použít bloky a tedy i iterátory apod. v propojení s vkládaným textem:
    \scriptsize
    \begin{verbatim}
      <% (1..5).each do |i| %>
      Number <%= i %>
      <% end %>
      <% ary.each do |x| %>
      Array contains <%= x %>
      <% end %>
    \end{verbatim}
  \end{block}
\end{frame}

\begin{frame}{ERb -- shrnutí}
  \begin{itemize}
    \item dobrý sluha, ale špatný pán
    \item můžu s tím vyrobit hromadu užitečných věcí na malém prostoru
    \item daň je velké riziko zamotaného kódu a nízké přehlednosti (struktura naprosto není patrná na první pohled, proto je namístě ji držet maximálně jednoduchou)
  \end{itemize}
\end{frame}

\begin{frame}{Důležité upozornění}
  \begin{itemize}
    \item oddělení modelu a view
    \item přestože lze provádět zpracování dat a výpočty přímo v ERb, je to nejvíc nejhorší nápad
    \item je chytré si všechno připravit v modelu (tj. v Ruby skriptu, kterým data chystáme)
    \item a kód ve view (tj. v ERb šabloně) omezit na naprosté minimum
  \end{itemize}
\end{frame}

\begin{frame}[fragile]{Jak ze šablony udělat výsledek}
  \scriptsize
  \begin{block}{Příklad překladu ERb}
    \scriptsize
    \begin{verbatim}
require "erb_compiler"

erb(template, filename, {:x => 1, :y => 2})
    \end{verbatim}
  \end{block}
  třetí parametr je hash, který nám vlastně definuje proměnné dostupné uvnitř šablony při překladu
\end{frame}

\begin{frame}[fragile]{Příklad -- kreslení grafů z minula}
  \begin{block}{template.gp}
    \scriptsize
    \begin{verbatim}
set terminal png
set output "plot_<%=n%>.png"
plot "data_<%=n%>.csv"
    \end{verbatim}
  \end{block}
  \begin{block}{}
    \scriptsize
    \begin{verbatim}
(1..10).each do |i|
  erb("template.gp", "plot_#{i}.gp", {:n => i})
end
    \end{verbatim}
  \end{block}
\end{frame}

\begin{frame}[fragile]{Takže v latexu třeba}
\scriptsize
\begin{verbatim}
  \subsection{Koncentrace kyseliny borité}

  <% cycles.each do |i| %>

  \subsubsection{Kampaň <%= i %>}

  \begin{center}
    \includegraphics[width=0.8\textwidth]{bc_<%= "%02d" % i %>_bc.eps}
  \end{center}

  <% end %>
\end{verbatim}
\end{frame}

\begin{frame}{Dva úkoly na začátek}
  \begin{enumerate}
    \item pomocí ERb vyrobte v LaTeXu malou násobilku do tabulky
    \item s jedinou šablonou vyrobte 3 dokumenty s násobilkou 7x7, 9x9, 11x11 (musíte parametrizovat)
  \end{enumerate}
\end{frame}

\section{Výroba dokumentu v~praxi - ERb}

\begin{frame}{Úkol na dnešek}
  \begin{itemize}
    \item přepracovat náš krásný report z JE Třeskoprsky s použitím ERb
    \item tabulky nicméně zajisté předělávat nebudeme -- použijeme hotové
  \end{itemize}
\end{frame}

\begin{frame}{Co je podstatné}
  \begin{itemize}
    \item nezapomenout, kdy jsem v Ruby, kdy v ERb, kdy v latexu a vždy dodržet zadanou syntaxi
    \item mít na paměti, jak dostat data ze skriptu do šablony
    \item potřebuju mít načtený seznam všech kampaní do nějakého pole a to poslat do šablony
  \end{itemize}
\end{frame}

\begin{frame}{A to je vše, přátelé!}
  \begin{center}
    \includegraphics[width=0.8\textwidth]{looney_tunes}
  \end{center}
\end{frame}

\end{document}
