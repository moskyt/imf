\documentclass{beamer}

\def\Tiny{\fontsize{6pt}{6pt}\selectfont}
\def\supertiny{\fontsize{4pt}{4pt}\selectfont}

\mode<presentation>
{
  \usetheme{Warsaw}
  % \setbeamercovered{transparent}
  \usecolortheme{crane}
}


\usepackage{graphicx, ifthen, listings}

\usepackage[czech]{babel}
% \usefonttheme{professionalfonts}
\usepackage{times}
\usepackage{amsmath}
\usepackage[utf8]{inputenc}
\usepackage{wrapfig}

\usepackage[T1]{fontenc}

\lstset{ basicstyle=\tiny, stringstyle=\ttfamily, showstringspaces=false }

\everymath{\displaystyle}

\setbeamerfont{frametitle}{size=\large}
\setbeamerfont{subsection in toc}{size=\scriptsize}

\makeatletter\newenvironment{blackbox}{%
   \begin{lrbox}{\@tempboxa}\begin{minipage}{0.95\columnwidth}}{\end{minipage}\end{lrbox}%
   \colorbox{black}{\usebox{\@tempboxa}}
}\makeatother

\title[IMF (4)]{Informatika pro moderní fyziky (4)\\rozšíření RubyGems, tvorba excelovské tabulky a zpracování textu}

\author[Franti\v{s}ek HAVL\r{U}J, ORF ÚJV Řež]{Franti\v{s}ek HAVL\r{U}J\\{\scriptsize \emph{e-mail: haf@ujv.cz}}}

\date{akademický rok 2020/2021, 26. října 2020}

\institute[ORF ÚJV Řež]
{ÚJV Řež\\oddělení Reaktorové fyziky a podpory palivového cyklu}

\AtBeginSection[]
{
\begin{frame}<beamer>
\frametitle{Obsah}
\tableofcontents[currentsection,hideothersubsections]
\end{frame}
}

\begin{document}

\begin{frame}
  \titlepage
\end{frame}

\begin{frame}
  \tableofcontents
\end{frame}

\section{Co jsme se naučili minule}

\begin{frame}{}
  \begin{itemize}
    \item práci s datovými soubory
    \item zpracování většího množství dat
  \end{itemize}
\end{frame}

\section{Rozšíření Ruby: RubyGems a Bundler}

\begin{frame}{Knihovny (gemy) jsou základ}
  \begin{itemize}
    \item existují mnohá rozšíření, tzv. knihovny -- v ruby se jim říká \textbf{rubygems}
    \item aktuálně nás zajímá něco na práci s excelovskými soubory
    \item gemy jdou sice instalovat na systémové úrovni, ale z toho je pak zase jenom neštěstí
    \item použijeme radši \textbf{bundler}, správce gemů pro každého: vyřeší za nás závislosti a postará se o snadnou instalaci
  \end{itemize}
\end{frame}

\begin{frame}{Máme bundler?}
  \begin{itemize}
    \item otestujeme rubygems: \texttt{gem -v}
    \item pokud není, zapláčeme, protože jsme asi špatně nainstalovali Ruby
    \item otestujeme bundler: \texttt{bundle -v}
    \item pokud bundler není, doinstalujeme \texttt{gem install bundler}
  \end{itemize}
\end{frame}

\begin{frame}{Jak na to}
  \begin{itemize}
    \item najdu si, která knihovna mě zajímá (třeba na rubygems.org nebo kdekoli jinde): my bychom rádi \textbf{rubyXL} \texttt{https://github.com/weshatheleopard/rubyXL}
    \item vytvořím si prázdný \texttt{Gemfile} -- tam se specifikuje, které gemy chci používat: \texttt{bundle init}
    \item do gemfilu -- je to normální Ruby skript! -- dopíšu\\ \texttt{gem "rubyXL"}
    \item nainstaluju: \texttt{bundle install -{}-path vendor/bundle} (ten parametr stačí poprvé, bundler si to pak pamatuje v konfiguráku \texttt{bundle/.config})
  \end{itemize}
\end{frame}

\begin{frame}{Jak použít?}
  \begin{itemize}
    \item na začátku svého skriptu pak musím nahrát bundler:
    \item \texttt{require "bundler/setup"}
    \item a teď už můžu nahrát jakýkoli gem:
    \item \texttt{require "rubyXL"}
  \end{itemize}
\end{frame}

\subsection{Vytvoříme excelovskou tabulku}

\begin{frame}[fragile]{RTFM, RTFM, RTFM}
  \begin{itemize}
    \item na stránkách \texttt{rubyXL} se nachází spousta příkladů a návodů -- https://github.com/weshatheleopard/rubyXL
    \item kromě toho má i slušnou dokumentaci (GIYF / "rubyxl docs") -- http://www.rubydoc.info/gems/rubyXL/3.3.15
    \item napoprvé navedu do začátku:
  \end{itemize}
  {\scriptsize
  \begin{verbatim}
    workbook = RubyXL::Workbook.new
    worksheet = workbook[0]
    worksheet.add_cell(0, 0, "A1")
    workbook.write("data.xlsx")
  \end{verbatim}
  }
\end{frame}

\begin{frame}{Jednoduché cvičení}
  \begin{itemize}
    \item použijte soubor \texttt{data\_two\_1.csv}
    \item vytvořte excelovský soubor se dvěma listy, na obou bude sloupec 1, sloupec 2 (data) a třetí sloupec součet
    \item na jednom listu součet bude jako číslo (sečte to váš skript)
    \item na druhém listu bude součet jako excelovský vzorec
  \end{itemize}
\end{frame}

\section{Zpracování textu}

\subsection{Obecný rozbor}

\begin{frame}{Problém č. 2: mnoho výpočtů, inženýrova smrt}
  \begin{block}{Zadání}
    Při přípravě základního kritického experimentu je pomocí MCNP potřeba najít kritickou polohu regulační tyče R2.

    Jak se tato poloha změní při změně polohy tyče R1?
  \end{block}
\end{frame}

\begin{frame}{Co máme k~dispozici?}
  \begin{block}{MCNP}
    Pokud připravíme vstupní soubor (v~netriviální formě obsahující polohy regulačních tyčí R1 a R2), spočítá nám keff.
  \end{block}
  Potřebovali bychom ale něco na:
  \begin{enumerate}
    \item vytvoření velkého množství vstupních souborů
    \item extrakci keff z~výstupních souborů
    \item popřípadě na vyhodnocení získaných poloh tyčí a keff
  \end{enumerate}
\end{frame}

\begin{frame}{Pracovní postup}
  \begin{enumerate}
    \item načíst keff z~výstupního souboru MCNP
    \pause
    \item vygenerovat potřebné vstupní soubory
    \pause
    \item vyrobit BAT soubor na spuštění výpočtů
    \pause
    \item načíst výsledky ze všech výstupních souborů do jedné tabulky
  \end{enumerate}
\end{frame}

\subsection{Načítání výstupního souboru}

\begin{frame}[fragile]{}
  Nejprve najdeme, kde je ve výstupu z~MCNP žádané keff:
  {\tiny
  \begin{verbatim}
.....
         the k(trk length) cycle values appear normally distributed at the 95 percent confidence level

-----------------------------------------------------------------------------------------------------------------------------------
|                                                                                                                                 |
| the final estimated combined collision/absorption/track-length keff = 1.00353 with an estimated standard deviation of 0.00024   |
|                                                                                                                                 |
| the estimated 68, 95, & 99 percent keff confidence intervals are 1.00329 to 1.00377, 1.00305 to 1.00400, and 1.00289 to 1.00416 |
|                                                                                                                                 |
| the final combined (col/abs/tl) prompt removal lifetime = 1.0017E-04 seconds with an estimated standard deviation of 2.7364E-08 |
.....
  \end{verbatim}
  }
\end{frame}

\begin{frame}{Algoritmus}
  \begin{enumerate}
    \item najít řádek s~keff
    \pause
    \item vytáhnout z~něj keff, takže například:
    \pause
    \item rozdělit podle rovnítka
    \pause
    \item druhou část rozdělit podle mezer
    \pause
    \item vzít první prvek
  \end{enumerate}
\end{frame}


\begin{frame}[fragile]{Realizace (1/5)}
  \scriptsize
  \begin{verbatim}
    keff = nil

    File.foreach("c1_1o") do |line|

    end

    puts keff
  \end{verbatim}
\end{frame}

\begin{frame}[fragile]{Realizace (2/5)}
  \scriptsize
  \begin{verbatim}
    keff = nil

    File.foreach("c1_1o") do |line|
      if line.include?("final estimated combined")

      end
    end

    puts keff
  \end{verbatim}
\end{frame}

\begin{frame}[fragile]{Realizace (3/5)}
  \scriptsize
  \begin{verbatim}
    keff = nil

    File.foreach("c1_1o") do |line|
      if line.include?("final estimated combined")
        a = line.split("=")

      end
    end

    puts keff
  \end{verbatim}
\end{frame}

\begin{frame}[fragile]{Realizace (4/5)}
  \scriptsize
  \begin{verbatim}
    keff = nil

    File.foreach("c1_1o") do |line|
      if line.include?("final estimated combined")
        a = line.split("=")
        b = a[1].split

      end
    end

    puts keff
  \end{verbatim}
\end{frame}

\begin{frame}[fragile]{Realizace (5/5)}
  \scriptsize
  \begin{verbatim}
    keff = nil

    File.foreach("c1_1o") do |line|
      if line.include?("final estimated combined")
        a = line.split("=")
        b = a[1].split
        keff = b[0]
      end
    end

    puts keff
  \end{verbatim}
\end{frame}

\subsection{Zápis všech výsledků do tabulky}

\begin{frame}{Jak na to}
  Máme všechno, co potřebujeme:
  \begin{itemize}
    \item načtení keff z~jednoho výstupního souboru (\texttt{File.foreach}, \texttt{include} a \texttt{split})
    \item procházení adresáře (\texttt{Dir.each})
    \item zápis do souboru (\texttt{File.open} s~parametrem \texttt{w} anebo \texttt{File.write})
  \end{itemize}
  Takže už to stačí jen vhodným způsobem spojit dohromady!
\end{frame}

\begin{frame}[fragile]{Realizace}
  \scriptsize
  \begin{verbatim}
    Dir["*o"].each do |filename|
      keff = nil

      File.foreach(filename) do |line|
        if line.include?("final estimated combined")
          a = line.split("=")
          b = a[1].split
          keff = b[0]
        end
      end

      puts "#{filename} #{keff}"
    end
  \end{verbatim}
\end{frame}

\begin{frame}[fragile]{Výstup}
  Výsledkem je perfektní tabulka:
  {\scriptsize
  \begin{verbatim}
    outputs/c_0_0o 0.94800
    outputs/c_0_10o 0.99800
    outputs/c_0_1o 0.94850
    outputs/c_0_2o 0.95000
    outputs/c_0_3o 0.95250
    outputs/c_0_4o 0.95600
    ...
  \end{verbatim}}
  Hloupé je, že nikde nemáme tu polohu tyčí.
\end{frame}

\begin{frame}[fragile]{Víc by se nám hodilo}
  něco jak toto:
  {\scriptsize
  \begin{verbatim}
    0   0  0.94800
    0 640  0.99800
    0  64  0.94850
    0 128  0.95000
    0 192  0.95250
    0 256  0.95600
    ...
  \end{verbatim}
  }
  (rozsah je 0 -- 640, my máme kroky 0 -- 10)
\end{frame}

\begin{frame}[fragile]{A protože přehlednost je nade vše}
  něco jak toto:
  {\scriptsize
  \begin{verbatim}
    keff           0           64          128    ...
      0      0.94800      0.94900      0.95000    ...
     64      0.94850      0.94950      0.95050    ...
    128      0.95000      0.95100      0.95200    ...
    192      0.95250      0.95350      0.95450    ...
    256      0.95600      0.95700      0.95800    ...
    320      0.96050      0.96150      0.96250    ...
    384      0.96600      0.96700      0.96800    ...
    448      0.97250      0.97350      0.97450    ...
    512      0.98000      0.98100      0.98200    ...
    576      0.98850      0.98950      0.99050    ...
    640      0.99800      0.99900      1.00000    ...
  \end{verbatim}
  }
  -- alternativně můžete vyrobit tabulku v excelu!
\end{frame}


\begin{frame}[fragile]{Hash to the rescue}
  Možná se bude velmi hodit hash!
  {\scriptsize
  \begin{verbatim}
    data = {}
    data[3] = 0.99
  \end{verbatim}
  }
  no a dokonce, protože klíč může být cokoliv:
  {\scriptsize
  \begin{verbatim}
    data = {}
    data[[1,2]] = 0.99
    data[[7,8]] = 1.05
    puts data[[1,2]]
  \end{verbatim}
  }
\end{frame}

\begin{frame}{Navážeme na úspěchy z minulých týdnů}
  \begin{itemize}
    \item vykreslit graf! pro každou z 11 poloh R1 jedna čára (závislost keff na R2)
    \item (= csv soubor, gnuplot, znáte to)
    \item najít automaticky kritickou polohu R2 pro každou z 11 poloh R1
    \item a zase graf... (kritická poloha R2 v závislosti na R1)
  \end{itemize}
\end{frame}

\begin{frame}{A to je vše, přátelé!}
  \begin{center}
    \includegraphics[width=0.8\textwidth]{looney_tunes}
  \end{center}
\end{frame}

\end{document}
