\documentclass{beamer}

\def\Tiny{\fontsize{6pt}{6pt}\selectfont}
\def\supertiny{\fontsize{4pt}{4pt}\selectfont}

\mode<presentation>
{
  \usetheme{Warsaw}
  % \setbeamercovered{transparent}
  \usecolortheme{crane}
}

\usepackage{graphicx, ifthen, listings, fancyvrb}

\usepackage[czech]{babel}
% \usefonttheme{professionalfonts}
\usepackage{times}
\usepackage{amsmath}
\usepackage[utf8]{inputenc}
\usepackage{wrapfig}

\usepackage[T1]{fontenc}

\lstset{ basicstyle=\tiny, stringstyle=\ttfamily, showstringspaces=false }

\everymath{\displaystyle}

\setbeamerfont{frametitle}{size=\large}
\setbeamerfont{subsection in toc}{size=\scriptsize}

\makeatletter\newenvironment{blackbox}{%
   \begin{lrbox}{\@tempboxa}\begin{minipage}{0.95\columnwidth}}{\end{minipage}\end{lrbox}%
   \colorbox{black}{\usebox{\@tempboxa}}
}\makeatother

\title[IMF (12)]{Informatika pro moderní fyziky (12)\\ API - dokončení, úvod do OOP, zadání zápočtových úloh}

\author[Franti\v{s}ek HAVL\r{U}J, ORF ÚJV Řež]{Franti\v{s}ek HAVL\r{U}J\\{\scriptsize \emph{e-mail: haf@ujv.cz}}}

\date{akademický rok 2020/2021\\21. prosince 2020}

\institute[ORF ÚJV Řež]
{ÚJV Řež\\oddělení Reaktorové fyziky a podpory palivového cyklu}

\AtBeginSection[]
{
\begin{frame}<beamer>
\frametitle{Obsah}
\tableofcontents[currentsection,hideothersubsections]
\end{frame}
}

\begin{document}

\begin{frame}
  \titlepage
\end{frame}

\begin{frame}
  \tableofcontents
\end{frame}

\section{Co jsme se naučili minule}

\begin{frame}{}
  \begin{itemize}
    \item práci s cizím API
    \item vytvoření jednoduché aplikace s interaktivní mapou
  \end{itemize}
\end{frame}


\section{Použití API a zpracování dat - dokončení}

\begin{frame}{Propojení více tabulek - opakování}
  \begin{itemize}
    \item procvičíme: k výpisu všech závodů přidáme i získávání výsledků
    \item Seznam závodů: \url{https://oris.orientacnisporty.cz/API/?format=json\&method=getEventList\&sport=3\&datefrom=2019-01-01\&dateto=2019-12-31}
    \item Výsledky: \url{https://oris.orientacnisporty.cz/API/?format=json&method=getEventResults&eventid=2077}
    \item vyrobte skript, který v letech 2015 až 2019 vypíše všechny závody a pro každý i vítězku ženské elitní kategorie (W21E)
  \end{itemize}
\end{frame}

\begin{frame}[fragile]{Připomenutí / co potřebujeme}
\begin{verbatim}
  require "open-uri"
  raw_data = URI.open("https://.....").read
  hash = JSON[raw_data]
\end{verbatim}
\end{frame}


\section{Úvod do OOP}

\begin{frame}{Co je OOP (1)}
  \begin{itemize}
    \item zatím jsme používali tzv. procedurální programování – máme data a pak procedury/funkce
    \item Ruby je nicméně čistě objektový jazyk, i když jsme se tomu zatím spíš vyhýbali
    \item objekt je entita, která má \emph{vlastnosti} (properties) a \emph{metody} (methods) a poskytuje okolnímu světu nějaké \emph{rozhraní} (interface)
  \end{itemize}
\end{frame}

\begin{frame}{Co je OOP (2)}
  \begin{itemize}
    \item většinou se jako hlavní důvody pro OOP uvádí polymorfismus a dědičnost (inheritance), ale to hlavní je posun uvažování od dat a operací nad nimi k ``inteligentním'' objektům – a spolu s tím zapouzdření (encapsulation)
    \item je potřeba si uvědomit, že všechna `zázračná' paradigmata v programování jsou pouze odlišné formalismy, může nám to hodně pomoct a stojí za to se tomu věnovat, na druhou stranu je potřeba se z toho nezbláznit a nedělat z toho náboženství
  \end{itemize}
\end{frame}

\begin{frame}{Třída}
  \begin{itemize}
    \item `typ' objektu - definuju, jak se objekty chovají
    \item neboli definuju metody
    \item zjednodušeně řečeno jsou to jen metody a ne data
    \item potkali jsme třeba \texttt{File}, \texttt{CSV}
  \end{itemize}
\end{frame}

\begin{frame}[containsverbatim]{Instance}
  \begin{itemize}
    \item konkrétní objekt – má svoje data
    \item většinou vytváříme pomocí \texttt{Class.new}, resp \texttt{Class.new(arg1, arg2, ...)}
    \item k datům přistupuju pomocí \emph{instance variables} -- \verb|@data|
    \item leckdy se hodí speciální metoda \texttt{initialize}, tzv. \emph{konstruktor} -- volá se při \texttt{new} a nastavují se zde výchozí hodnoty vlastností
  \end{itemize}
\end{frame}

\begin{frame}{Class method}
  \begin{itemize}
    \item můžu mít i metody, které nepatří k žádné instanci – nepracují s žádnými daty
    \item dávat je do třídy má pak smysl pouze organizační, nemá to pak žádnou reálnou výhodu proti obyčejným funkcím
    \item např \texttt{File.foreach} nebo \texttt{CSV.read}
  \end{itemize}
\end{frame}

\begin{frame}[containsverbatim]{Jak to vypadá v Ruby}
  \begin{Verbatim}[fontsize=\tiny]
  class Table
    def initialize
      @data = {}
    end
    def get(key)
      @data[key]
    end
    def set(key, value)
      @data[key] = value
    end
    def print
      @data.each do |key, value|
        puts "#{key} #{value}"
      end
    end
  end

  t = Table.new
  t.set("a", 123)
  t.print
  \end{Verbatim}
\end{frame}

\section{Úloha – chemické vzorce}

\begin{frame}{Zadání}
  \begin{itemize}
    \item vytvořte skript, který pro zadané chemické vzorce sloučenin:
    \item spočítá molární hmotnost (g/mol)
    \item vypíše slovy, z jakých prvků a z kolika atomů se skládá
    \item (použijte OOP řešení)
  \end{itemize}
\end{frame}

\begin{frame}[fragile]{Příklad}
  \begin{Verbatim}[fontsize=\small]
NaCl
2
1 atom of sodium + 1 atom of chlorine
M = 58.443

H2O
3
2 atoms of hydrogen + 1 atom of oxygen
M = 18.015

CH3CH2OH
9
2 atoms of carbon + 6 atoms of hydrogen + 1 atom of oxygen
M = 46.069
  \end{Verbatim}
\end{frame}

\begin{frame}[fragile]{Jak by to mělo fungovat}
  \begin{Verbatim}[fontsize=\small]
    c = Compound.new("H3BO3")
    puts c
    puts c.size
    puts c.as_text
    puts "M = #{c.mass}"
  \end{Verbatim}
\end{frame}


\begin{frame}[fragile]{Různé rady}
  \begin{itemize}
    \item začnu třídou \texttt{Compound} a jejím konstruktorem
    \item (zde využiju metodu \texttt{scan} s regulárním výrazem)
    \item budu chtít vyrobit hash \verb|{"H"=>3,"B"=>1,"O"=>3}| a uložit si ho jako instance variable
    \item (to jsou moje vlastnosti na třídě \texttt{Compound})
    \item pak můžu udělat metody \texttt{size} a \texttt{as\_text}
  \end{itemize}
\end{frame}

\begin{frame}[fragile]{Různé rady (2)}
  \begin{itemize}
    \item při počítání molární hmotnosti je třeba odolat několika svodům:
    \item nenahrávejte tabulku prvků pro každou instanci Compound
    \item nezavádějte třídu pro jednotlivé prvky
    \item dobrá rada: vytvořte třídu pro tabulku prvků (nebo prostě udělejte hash), jednu její instanci a tu předávejte do konstruktoru instance Compound
  \end{itemize}
\end{frame}

\section{K zápočtovým úlohám}

\begin{frame}{Obecně:}
\begin{itemize}
  \item ke každému zadání jsou k dispozici vzorová data
  \item já to budu testovat i na datech jiných
  \item očekávám, že všechno proběhne na jedno spuštění skriptu
  \item každý má k dispozici jeden pokus řádný a jeden opravný
\end{itemize}
\end{frame}

\begin{frame}{Klasifikace}
  \begin{itemize}
    \item F - nejde to spustit, ani pro zadaná data to v podstatných bodech nesplňuje zadání
    \item E - pro zadaná data to funguje, ale pro jiná čísla to nechodí
    \item D - obecně to funguje, ale stejně chybí drobnosti ze zadání
    \item C - všechno funguje jak má
    \item B - funguje a navíc jsou výstupy hezké a přehledné, soubory nejsou generovány “na velkou hromadu”, ale roztříděné do složek apod.
    \item A - kromě výše uvedeného jsou splněny i požadavky formy (správné odsazování, rozumná jména funkcí a proměnných) a efektivity (je to rozumně naprogramované - vhodné použití funkcí, datových struktur atd.)
  \end{itemize}
\end{frame}

\begin{frame}{Známka se snižuje o stupeň, pokud:}
  \begin{itemize}
    \item jsou někde ve skriptech použity absolutní cesty, takže je budu muset upravovat (výjimkou jsou cesty k programům jako např. gnuplot, které ovšem musí být umístěny v proměnné někde na začátku skriptu (abych to nemusel lovit)
    \item bude v kódu něco, co limituje použití na OS Windows (backslash v cestě, kódování win1250 atd.)
  \end{itemize}
  (a podobně)
\end{frame}


\begin{frame}{A to je vše, přátelé!}
  \begin{center}
    \includegraphics[width=0.8\textwidth]{looney_tunes}
  \end{center}
\end{frame}

\end{document}
