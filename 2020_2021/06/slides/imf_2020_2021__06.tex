\documentclass{beamer}

\def\Tiny{\fontsize{6pt}{6pt}\selectfont}
\def\supertiny{\fontsize{4pt}{4pt}\selectfont}

\mode<presentation>
{
  \usetheme{Warsaw}
  % \setbeamercovered{transparent}
  \usecolortheme{crane}
}


\usepackage{graphicx, ifthen, listings}

\usepackage[czech]{babel}
% \usefonttheme{professionalfonts}
\usepackage{times}
\usepackage{amsmath}
\usepackage[utf8]{inputenc}
\usepackage{wrapfig}

\usepackage[T1]{fontenc}

\lstset{ basicstyle=\tiny, stringstyle=\ttfamily, showstringspaces=false }

\everymath{\displaystyle}

\setbeamerfont{frametitle}{size=\large}
\setbeamerfont{subsection in toc}{size=\scriptsize}

\makeatletter\newenvironment{blackbox}{%
   \begin{lrbox}{\@tempboxa}\begin{minipage}{0.95\columnwidth}}{\end{minipage}\end{lrbox}%
   \colorbox{black}{\usebox{\@tempboxa}}
}\makeatother

\title[IMF (6)]{Informatika pro moderní fyziky (6)\\výstupní a vstupní soubory pro výpočetní programy, datové struktury}

\author[Franti\v{s}ek HAVL\r{U}J, ORF ÚJV Řež]{Franti\v{s}ek HAVL\r{U}J\\{\scriptsize \emph{e-mail: haf@ujv.cz}}}

\date{akademický rok 2020/2021, 9. listopadu 2020}

\institute[ORF ÚJV Řež]
{ÚJV Řež\\oddělení Reaktorové fyziky a podpory palivového cyklu}

\AtBeginSection[]
{
\begin{frame}<beamer>
\frametitle{Obsah}
\tableofcontents[currentsection,hideothersubsections]
\end{frame}
}

\begin{document}

\begin{frame}
  \titlepage
\end{frame}

\begin{frame}
  \tableofcontents
\end{frame}

\section{Co jsme se naučili minule}

\begin{frame}{}
  \begin{itemize}
    \item komplexní zpracování dat z výpočetních programů
    \item použití hashe jako univerzální datové struktury (místo pole)
    \item opakování: tvorba grafů s použitím gnuplotu
  \end{itemize}
\end{frame}

\section{Automatizace tvorby vstupů}

\begin{frame}[fragile]{Velké množství podobných výpočtů}
  \begin{itemize}
    \item vrátíme se na začátek problému – pracovali jsme s výstupními soubory, ale potřebujeme vyřešit provedení těchto výpočtů
    \item neboli vytvoření 11 x 11 vstupních souborů
    \item liší se pouze polohou tyčí R1 a R2
    \item pochopitelně nebudeme dělat ručně
    \item řešením je parametrizovaná šablona
  \end{itemize}
\end{frame}

\begin{frame}[fragile]{Určení poloh tyčí}
  Ve vstupním souboru si najdeme relevantní část:
  \scriptsize
  \begin{verbatim}
    c ---------------------------------
    c polohy tyci (z-plochy)
    c ---------------------------------
    c
    67 pz 47.6000    $ dolni hranice absoberu r1
    68 pz 40.4980    $ dolni hranice hlavice r1
    69 pz 44.8000    $ dolni hranice absoberu r2
    70 pz 37.6980    $ dolni hranice hlavice r2
  \end{verbatim}
\end{frame}

\begin{frame}[fragile]{Výroba šablon}
  Jak dostat polohy tyčí do vstupního souboru? Vyrobíme šablonu, tzn nahradíme
  \begin{verbatim}
    67 pz 47.6000    $ dolni hranice absoberu r1
  \end{verbatim}
  \pause
  nějakou značkou (\emph{placeholder}):
  \begin{verbatim}
    67 pz %r1%    $ dolni hranice absoberu r1
  \end{verbatim}
  Vezměme to tak (i když to možná není pravda), že polohy jsou od 0 do 640 mm a že je to přímo kóta spodní hrany absorbéru.
\end{frame}

\begin{frame}{Chytáky a zádrhele}
  \begin{itemize}
    \item kromě samotné plochy konce absorbéru je nutno správně umístit i z-plochu konce hlavice o 7,102 cm níže
    \item obecně je na místě ohlídat si, že placeholder nebude kolidovat s ničím jiným
  \end{itemize}
  Doporučené nástroje jsou:
  \begin{itemize}
    \item již známá funkce \texttt{sub} / \texttt{gsub} pro nahrazení jednoho řetězce jiným
    \item pro pragmatické lenochy funkce \texttt{File.read} načítající celý soubor do řetězce (na což nelze v mnoha programovacích jazycích ani pomyslet)
    \item možno ovšem použít i \texttt{File.readlines} (v čem je to lepší?)
  \end{itemize}
\end{frame}

\begin{frame}[fragile]{Realizace}
  \scriptsize
  \begin{verbatim}
    DELTA = 44.8000 - 37.6980

    template = File.read("template")
    (0..10).each do |i1|
      (0..10).each do |i2|
        r1 = i1 * 6.4
        r2 = i2 * 6.4
        s = template.gsub("%r1%", r1.to_s)
        s = s.gsub("%r1_%", (r1 - DELTA).to_s)
        s = s.gsub("%r2%", r2.to_s)
        s = s.gsub("%r2_%", (r2 - DELTA).to_s)
        File.write("inputs/c_#{i1}_#{i2}", s)
      end
    end
  \end{verbatim}
\end{frame}

\section{Automatizace tvorby vstupů -- zobecnění}

\begin{frame}{A co takhle trocha zobecnění?}
  \begin{itemize}
    \item když budu chtít přidat další tyče nebo jiné parametry, bude to děsně bobtnat
    \item funkce \texttt{process("template", "inputs/c\_\#\{i1\}\_\#\{i2\}", \{'r1' => r1, 'r2' => r2, .....\})}
    \item všechno víme, známe, umíme...
    \item rozšiřte tak, že třeba tyč B1 bude mezi R1 a R2, B2 mezi R1 a B1, B3 mezi R2 a dolní hranicí palivového článku (Z = 1 cm)
  \end{itemize}
\end{frame}

\section{Načítání složitějšího výstupu}

\begin{frame}[fragile]{HELIOS}
  Tabulka výstupů:
  \scriptsize
  \begin{verbatim}
List name       : list
List Title(s)  1) This is a table
               2) of some data
               3) in many columns
               4) and has a long title!

               bup        kinf          ab          ab        u235        u238       pu239
 0001     0.00E+00     1.16949  9.7053E-03  7.6469E-02  1.8806E-04  7.5509E-03  1.0000E-20
 0002     0.00E+00     1.13213  9.7478E-03  7.9058E-02  1.8806E-04  7.5509E-03  1.0000E-20
 0003     1.00E+01     1.13149  9.7488E-03  7.9070E-02  1.8797E-04  7.5509E-03  2.3647E-09
 0004     5.00E+01     1.13004  9.7521E-03  7.9093E-02  1.8760E-04  7.5506E-03  5.3144E-08
 0005     1.00E+02     1.12826  9.7559E-03  7.9218E-02  1.8714E-04  7.5503E-03  1.8746E-07
 0006     1.50E+02     1.12664  9.7594E-03  7.9407E-02  1.8668E-04  7.5500E-03  3.7495E-07
 0007     2.50E+02     1.12399  9.7657E-03  7.9869E-02  1.8577E-04  7.5493E-03  8.4139E-07
 0008     5.00E+02     1.12007  9.7812E-03  8.1065E-02  1.8351E-04  7.5476E-03  2.1882E-06
 0009     1.00E+03     1.11561  9.8203E-03  8.3169E-02  1.7914E-04  7.5443E-03  4.8723E-06
 0010     2.00E+03     1.10542  9.9329E-03  8.6731E-02  1.7088E-04  7.5376E-03  9.6873E-06
 0011     3.00E+03     1.09354  1.0067E-02  8.9717E-02  1.6316E-04  7.5309E-03  1.3879E-05
 0012     4.00E+03     1.08126  1.0207E-02  9.2299E-02  1.5591E-04  7.5242E-03  1.7571E-05
 0013     6.00E+03     1.05755  1.0474E-02  9.6562E-02  1.4258E-04  7.5105E-03  2.3760E-05
  \end{verbatim}
\end{frame}

\begin{frame}{Co bychom chtěli}
  \begin{itemize}
    \item mít načtené jednotlivé tabulky (zatím jen jednu, ale bude jich víc)
    \item asi po jednotlivých sloupcích, sloupec = pole (hodnot po řádcích)
    \item sloupce se nějak jmenují, tedy použijeme \texttt{Hash}
    \item \texttt{table["kinf"]}
    \item pozor na \texttt{ab}, asi budeme muset vyrobit něco jako \texttt{ab1}, \texttt{ab2} (ale to až za chvíli)
  \end{itemize}
\end{frame}

\begin{frame}{Nástrahy, chytáky a podobně}
  \begin{itemize}
    \item tabulka skládající se z více bloků
    \item více tabulek
    \item tabulky mají jméno \texttt{list name} a popisek \texttt{list title(s)}
  \end{itemize}
\end{frame}

\begin{frame}[fragile]{Jak uspořádat data?}
  \begin{itemize}
    \item pole s tabulkami + pole s názvy + pole s titulky?
    \pause
    \item co hashe tabulky[název] a titulky[název]?
    \pause
    \item nejchytřeji: \verb!{"a"=>{:title => "Table title",! \\ \verb!:data => {"kinf"=>...}}}!
    \pause
    \item ``nová'' syntaxe: \verb!{"a"=>{title: "Table title",! \\ \verb!data: {"kinf"=>...}}}!
  \end{itemize}
\end{frame}

\begin{frame}{Z příkazové řádky}
  \begin{itemize}
    \item a co takhle z toho udělat skript, který lze pustit s~argumentem = univerzální
    \item \texttt{ruby read\_helios.rb helios1.out}
    \item vypíše seznam všech tabulek, seznam jejich sloupců, počet řádků
    \item pole \texttt{ARGV} -- seznam všech argumentů
    \item vylepšení – provede pro všechny zadané soubory: \texttt{ruby read\_helios.rb helios1.out helios2.out} (tip: využívejte vlastní metody, kde to jen jde)
  \end{itemize}
\end{frame}

\begin{frame}{A to je vše, přátelé!}
  \begin{center}
    \includegraphics[width=0.8\textwidth]{looney_tunes}
  \end{center}
\end{frame}

\end{document}
