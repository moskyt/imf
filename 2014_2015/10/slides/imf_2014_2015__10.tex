\documentclass{beamer}

\def\Tiny{\fontsize{6pt}{6pt}\selectfont}
\def\supertiny{\fontsize{4pt}{4pt}\selectfont}

\mode<presentation>
{
  \usetheme{Warsaw}
  % \setbeamercovered{transparent}
  \usecolortheme{crane}
}

\usepackage{graphicx, ifthen, listings, fancyvrb}

\usepackage[czech]{babel}
% \usefonttheme{professionalfonts}
\usepackage{times}
\usepackage{amsmath}
\usepackage[utf8]{inputenc}
\usepackage{wrapfig}

\usepackage[T1]{fontenc}

\lstset{ basicstyle=\tiny, stringstyle=\ttfamily, showstringspaces=false }

\everymath{\displaystyle}

\setbeamerfont{frametitle}{size=\large}
\setbeamerfont{subsection in toc}{size=\scriptsize}

\makeatletter\newenvironment{blackbox}{%
   \begin{lrbox}{\@tempboxa}\begin{minipage}{0.95\columnwidth}}{\end{minipage}\end{lrbox}%
   \colorbox{black}{\usebox{\@tempboxa}}
}\makeatother

\title[IMF (10)]{Informatika pro moderní fyziky (10)\\ složitější interaktivní dokument, získávání informací z webu}

\author[Franti\v{s}ek HAVL\r{U}J, ORF ÚJV Řež]{Franti\v{s}ek HAVL\r{U}J\\{\scriptsize \emph{e-mail: haf@ujv.cz}}}

\date{akademický rok 2014/2015\\24. listopadu 2014}

\institute[ORF ÚJV Řež]
{ÚJV Řež\\oddělení Reaktorové fyziky a podpory palivového cyklu}

\AtBeginSection[]
{
\begin{frame}<beamer>
\frametitle{Obsah}
\tableofcontents[currentsection,hideothersubsections]
\end{frame}
}

\begin{document}

\begin{frame}
  \titlepage
\end{frame}

\begin{frame}
  \tableofcontents
\end{frame}

\section{Co jsme se naučili minule}

\begin{frame}{}
  \begin{itemize}
    \item další procvičení zpracování dat - data s vazbami
    \item CSS selektory
    \item získávání informací z webu
    \item LaTex a ERb -- opakování
  \end{itemize}
\end{frame}

\section{Navážeme na předminulou hodinu}

\begin{frame}{Zadání -- připomenutí}
  \begin{itemize}
    \item každý den data z 1-9 detektorů (\texttt{data/*.csv})
    \item detektor má svoji polohu v AZ (VR-1 Vrabec, 8x8 čtvercových pozic) -- včetně data je uvedena na prvním řádku CSV souboru
    \item je potřeba hezky zobrazit na každý den mapu AZ a grafy signálů z detektorů
    \item viz \texttt{html/document.html}
  \end{itemize}
\end{frame}

\begin{frame}{Co už máme}
  \begin{itemize}
    \item grafy pro jednotlivé detektory ve formátu PNG
    \item mapy zóny, ale zatím neklikací (SVG)
    \item umíme načíst data o jednotlivých detektorech -- víme, které pozice jsou v jednotlivé dny obsazeny
  \end{itemize}
\end{frame}

\begin{frame}{Co nám chybí}
  \begin{itemize}
    \item mít schované i názvy souborů pro jednotlivé detektory (hash! hash!)
    \item vyrobit si HTML dokument, který by zobrazoval jednotlivé mapy
    \item doplnit interaktivitu do SVG obrázků
    \item zobrazovat po kliknutí do mapy správný graf
  \end{itemize}
\end{frame}

\begin{frame}{Krok číslo jedna}
  \begin{itemize}
    \item z datových souborů si vyrobit takovou datovou strukturu, která mi bude říkat, pro který den a na které poloze je který soubor
    \item tj. radím například takovýto hash: {datum => {poloha => soubor}}
  \end{itemize}
\end{frame}

\begin{frame}{Další kroky}
  \begin{itemize}
    \item dopsat do mapy detektorů text - polohu
    \item vyrobit zatím verzi, kde si budu moci prohlížet jednotlivé mapy (zatím bez grafů)
    \item zobrazit grafy detektorů - zatím všechny
    \item dokončit -- zobrazovat grafy
  \end{itemize}
\end{frame}

\begin{frame}{Další JS chytrosti}
  \begin{itemize}
    \item v jQuery už známe \texttt{\$(`\#id')}
    \item ale ve skutečnosti jde použít jakýkoli CSS selektor, takže třeba \texttt{\$(`p')}
    \item pokročilý CSS selektor -- vnoření: \texttt{\#my\_list img} vybere všechny obrázky (\texttt{img}) které jsou uvnitř elementu s id \texttt{my\_list}
    \item ... použiju v situacích, kdy chci schovat nějakou množinu elementů a pak jeden z nich zobrazit (tj. když mám hromadu obrázků a chci, aby byl vidět jen jeden)
  \end{itemize}
\end{frame}

\begin{frame}{Další CSS chytrosti}
  \begin{itemize}
    \item normálně se jednotlivé elementy řadí pod sebe
    \item můžu místo toho použít tzv. floating, kdy se začnou elementy řadit nalevo nebo napravo
    \item efekt znáte např. z webových galerií, kde mi fotky vyplní celou šířku okna a jdou po řádcích
    \item \texttt{float:left}
    \item barvu pozadí nastavím např. \texttt{background-color:red} nebo \texttt{background-color:\#ddddff}
  \end{itemize}
\end{frame}


\begin{frame}{Další SVG chytrosti}
  \begin{itemize}
    \item kromě \texttt{rect} se bude hodit také \texttt{text}
    \item jako text se zobrazí obsah příslušného elementu
    \item opět použiju atributy \texttt{x}, \texttt{y} (levý dolní roh) a můžu přihodit \texttt{text-anchor=``middle''}, aby to byl dolní prostředek
    \item pozor, text mi překryje čtvereček, takže budu muset zopakovat onclick! (musí být na textu i na čtverečku)
  \end{itemize}
\end{frame}

\section{Předvánoční oddych -- komiks (= HTML scraping)}

\begin{frame}{Zpracování cizích zdrojů na webu -- HTML scraping}
  \begin{itemize}
    \item dosud jsme zpracovávali pouze lokální, hezky formátovaná data
    \item i leckteré externí služby poskytují pěkná API ve formátech JSON nebo XML
    \item ale leckdy taky ne a zajímavé informace
    \item protože jsme chytré horákyně, naučíme se, jak data automaticky získat
  \end{itemize}
\end{frame}

\begin{frame}{Zadání úkolu}
  \begin{itemize}
    \item protože po práci si chceme oddychnout a netrápit se přitom náročnou intelektuální činností, přečteme si rádi dobrý komiks
    \item a protože jsme přiměřeně cyničtí a také si chceme pocvičit angličtinu, přečteme si redmeat
    \item www.redmeat.com
    \item .. ale nechceme klikat a nechceme číst na internetu, takže si zhotovíme PDFko se všemi díly najednou
    \item zvládneme to snadno v LaTeXu, ale potřebujeme postahovat ty obrázky
  \end{itemize}
\end{frame}

\begin{frame}{Knihovny pro Ruby -- příkaz require}
  \begin{itemize}
    \item už umíme používat \texttt{require} pro lokální soubory
    \item také funguje pro \emph{core library} -- soubor knihoven dodávaných v rámci distribuce Ruby
    \item kromě toho můžu využít i další zdroj knihoven -- ruby gems
  \end{itemize}
\end{frame}

\begin{frame}{Knihovny pro Ruby -- \emph{Ruby Gems}}
  \begin{itemize}
    \item viz www.rubygems.org
    \item nepřeberné hromady
    \item require `rubygems'
    \item require `něco'
  \end{itemize}
\end{frame}

\begin{frame}{Správa knihoven -- bundler}
  \begin{itemize}
    \item  nejjednodušší je instalovat knihovny někam ``k sobě''
    \item  \texttt{bundle -v}
    \item  \texttt{bundle init}
    \item  Gemfile -- seznam potřebných gemů(knihoven)
    \item  \texttt{bundle install --path vendor/bundle}
  \end{itemize}
\end{frame}

\begin{frame}{Gem nokogiri}
  \begin{itemize}
    \item nejlepší nástroj na parsování XML/HTML pod Ruby
    \item nevýhoda -- binární (C extension), takže mohou být potíže s instalací
    \item umí spoustu věcí, nám postačí přistupovat k elementům pomocí CSS selektorů (protože ty už umíme)
    \item každý element má metody \texttt{attributes} a \texttt{text}
  \end{itemize}
\end{frame}

\begin{frame}[fragile]{Vypiš všechny odkazy}
  Otevřu si dokument (buď řetězec, nebo jakýkoli \emph{stream}, což je například soubor) a pak můžu iterovat přes všechny elementy vyhovující danému selektoru:
  \scriptsize
  \begin{verbatim}
    doc = Nokogiri::HTML(File.open("redmeat.html"))
    doc.css("a").each do |x|
      puts "#{x.text} => #{x.attributes['href']}"
    end
  \end{verbatim}
\end{frame}

\begin{frame}{Jak na to}
  \begin{itemize}
    \item \texttt{http://www.redmeat.com/}, meat locker
    \item podíváme se na zdroják a koukáme -- vidíme, že máme velké štěstí, protože tam jsou dva \texttt{ul} seznamy
    \item po rozkliknutí a inspekci vidíme, že se obrázek jmenuje stejně, což je výhra
    \item stačí tedy rozparsovat dokument se seznamem, tak si ho uložíme na disk
  \end{itemize}
\end{frame}


\begin{frame}[fragile]{Stahování dat}
  Součást standardní knihovny -- \texttt{open-uri}
  \scriptsize
\begin{verbatim}
  require 'open-uri'
  File.open(local_filename, 'wb') do |f2|
    open(remote_url, 'rb') do |f1|
      f2.write f1.read
    end
  end
\end{verbatim}
\end{frame}

\begin{frame}{Relativní URL}
  \begin{itemize}
    \item http://www.redmeat.com/redmeat/meatlocker/index.html
    \item relativní \texttt{../2013-12-03/index.html}
    \item $\to$ http://www.redmeat.com/redmeat/2013-12-03/index.html
    \item relativní \texttt{index-1.gif}
    \item $\to$ http://www.redmeat.com/redmeat/2013-12-03/index-1.gif
  \end{itemize}
\end{frame}

\begin{frame}{Problémy s formáty}
  \begin{itemize}
    \item stažené obrázky jsou ve formátu GIF
    \item LaTeX umí z rastrů jen PNG a JPG
    \item potřeba převést -- pro automatizaci je vhodný balík \texttt{ImageMagick}
    \item ... a jeho příkaz \texttt{convert}
  \end{itemize}
\end{frame}

\begin{frame}{Vygenerovat PDF}
  \begin{itemize}
    \item zase triviální, už to umíme, rychlá akce na deset minut!
    \item ERb šablona, použít \texttt{erb\_compiler}
    \item seznam souborů vzít z \texttt{Dir["redmeat\_*.png"]}
    \item co a jak s LaTeXem viz např. šestou přednášku
  \end{itemize}
\end{frame}

\begin{frame}{A to je vše, přátelé!}
  \begin{center}
    \includegraphics[width=0.8\textwidth]{looney_tunes}
  \end{center}
\end{frame}

\end{document}
