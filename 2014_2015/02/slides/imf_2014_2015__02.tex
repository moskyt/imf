\documentclass{beamer}

\def\Tiny{\fontsize{6pt}{6pt}\selectfont}
\def\supertiny{\fontsize{4pt}{4pt}\selectfont}

\mode<presentation>
{
  \usetheme{Warsaw}
  % \setbeamercovered{transparent}
  \usecolortheme{crane}
}


\usepackage{graphicx, ifthen, listings}

\usepackage[czech]{babel}
% \usefonttheme{professionalfonts}
\usepackage{times}
\usepackage{amsmath}
\usepackage[utf8]{inputenc}
\usepackage{wrapfig}

\usepackage[T1]{fontenc}

\lstset{ basicstyle=\tiny, stringstyle=\ttfamily, showstringspaces=false }

\everymath{\displaystyle}

\setbeamerfont{frametitle}{size=\large}
\setbeamerfont{subsection in toc}{size=\scriptsize}

\makeatletter\newenvironment{blackbox}{%
   \begin{lrbox}{\@tempboxa}\begin{minipage}{0.95\columnwidth}}{\end{minipage}\end{lrbox}%
   \colorbox{black}{\usebox{\@tempboxa}}
}\makeatother

\title[IMF (2)]{Informatika pro moderní fyziky (2)\\základy Ruby, zpracování textu}

\author[Franti\v{s}ek HAVL\r{U}J, ORF ÚJV Řež]{Franti\v{s}ek HAVL\r{U}J\\{\scriptsize \emph{e-mail: haf@ujv.cz}}}

\date{akademický rok 2014/2015\\1. října 2014}

\institute[ORF ÚJV Řež]
{ÚJV Řež\\oddělení Reaktorové fyziky a podpory palivového cyklu}

\AtBeginSection[]
{
\begin{frame}<beamer>
\frametitle{Obsah}
\tableofcontents[currentsection,hideothersubsections]
\end{frame}
}

\begin{document}

\begin{frame}
  \titlepage
\end{frame}

\begin{frame}
  \tableofcontents
\end{frame}

\section{Co jsme se naučili minule}

\begin{frame}{}
  \begin{itemize}
    \item základní principy automatizace
    \item CSV soubory a Gnuplot
    \item příkazový řádek / terminál
    \item dávkové (BAT) soubory
    \item představení skriptovacích jazyků
    \item interpret Ruby a IRb
    \item letem světem Ruby 
  \end{itemize}
\end{frame}


\section{Úvod do jazyka Ruby}

\subsection{Ještě chvilku v IRb}

\begin{frame}[fragile]{OOP - volání metod}
  Klasickým příkladem je například počet znaků v řetězci.
  \begin{block}{procedurální jazyky}
    \begin{verbatim}
      strlen("retezec")
    \end{verbatim}
  \end{block}
  \pause
  Můžeme místo toho nahlížet na řetězec jako na objekt:
  \pause
  \begin{block}{objektově orientované jazyky}
    \begin{verbatim}
      "retezec".length
    \end{verbatim}
  \end{block}
\end{frame}

\begin{frame}[fragile]{Hrátky s řetězci}
  \begin{block}{Délka řetězce}
    \begin{verbatim}
      "krabice".length
      "kocour".size
    \end{verbatim}
  \end{block}
  \pause
  \begin{block}{Ořez mezer}
    \begin{verbatim}
      "   hromada ".strip
      " koleso   ".lstrip
    \end{verbatim}
  \end{block}
\end{frame}

\begin{frame}[fragile]{Hrátky s řetězci}
  \begin{block}{Hledání}
    \begin{verbatim}
      "koleno na kole".include?("kole")
      "koleno na kole".count("kole")
    \end{verbatim}
  \end{block}
  \pause
  \begin{block}{Nahrazení}
    \begin{verbatim}
      "volej kolej".sub("olej", "yber")
      "baba a deda".gsub("ba", "ta")
    \end{verbatim}
  \end{block}
\end{frame}

\begin{frame}{Dokumentace}
  \begin{block}{GIYF: Google is your friend}
    \texttt{ruby api string}
  \end{block}
  \begin{block}{API dokumentace}
    \texttt{http://www.ruby-doc.org/core-1.9.3/String.html}
  \end{block}
\end{frame}

\subsection{Pole}

\begin{frame}[fragile]{Ztracen v poli}
  \begin{block}{Literál, přiřazení}
    \begin{verbatim}
      a = []
      a << 1
      a << "string"
      b = []
    \end{verbatim}
  \end{block}
  \pause
  \begin{block}{Délka, řazení, vypletí, převracení}
    \begin{verbatim}
      [4, 2, 6].sort
      [2, 5, 3, 3, 4, 1, 2, 1].uniq.sort
      [4, 2, 6].reverse
    \end{verbatim}
  \end{block}
\end{frame}

\begin{frame}[fragile]{Ztracen v poli}
  \begin{block}{Indexace}
    \begin{verbatim}
      a = [1,2,3]
      a[1]
      a[3]
    \end{verbatim}
  \end{block}
  \pause
  \begin{block}{Do mínusu, odkud kam}
    \begin{verbatim}
      a = [1,2,3,4,5,6]
      a[-1]
      a[-2]
      a[0..3]
    \end{verbatim}
  \end{block}
\end{frame}

\begin{frame}[fragile]{Pole z řetězů}
  \begin{block}{Řetězec, pole znaků}
    \begin{verbatim}
      "kopr"[2]
      "mikroskop"[0..4]
    \end{verbatim}
  \end{block}
  \pause
  \begin{block}{Leccos funguje!}
    \begin{verbatim}
      "abcd".reverse
      [1,2,3].size
    \end{verbatim}
  \end{block}
  \begin{block}{Sekáček na maso}
    \begin{verbatim}
      "a b  c d".split
      "a b,c d".split(",")
    \end{verbatim}
  \end{block}
\end{frame}

\begin{frame}[fragile]{Operátor a operatér}
  \begin{block}{Malé bezvýznamné plus}
    \begin{verbatim}
      "alfa" + "beta"
      [1, 2] + [3, 4]
    \end{verbatim}
  \end{block}
  \pause
  \begin{block}{Násobilka}
    \begin{verbatim}
      "kolo" * 5
      [1, 2, 3] * 3
      ["a", "b", "c"] * ","
    \end{verbatim}
  \end{block}
\end{frame}

\begin{frame}[fragile]{}
  \begin{block}{Převádět přes ulici}
    \begin{verbatim}
      "123".to_i
      1250.to_s
      "0.6".to_f
    \end{verbatim}
  \end{block}
  \pause
  % \begin{block}{Hash / slovník}
  %   \begin{verbatim}
  %     a = {}
  %     a["xyz"] = 555
  %     b = {'a' => 4, 'c' => 6}
  %   \end{verbatim}
  % \end{block}
\end{frame}

\begin{frame}[fragile]{}
  \begin{block}{Vocaď pocaď}
    \begin{verbatim}
      (1..4)
      (0...10)
      (1..5).to_a
    \end{verbatim}
  \end{block}
  \pause
  \begin{block}{Symbolika}
    \begin{verbatim}
      "letadlo"
      :letadlo
    \end{verbatim}
  \end{block}
\end{frame}

\begin{frame}[fragile]{Boolean nebolí}
  \begin{block}{Jednoduchá porovnání}
    \begin{verbatim}
      2 + 2 < 5
      "alfa" != "beta"
      (x == y) and (y == z)
    \end{verbatim}
  \end{block}
  (pozor na \texttt{=} versus \texttt{==})
  \pause
  \begin{block}{Chytré metody}
    \begin{verbatim}
      [1, 2, 3].include?(3)
      "abc".include?("bc")
    \end{verbatim}
  \end{block}
\end{frame}


\begin{frame}[fragile]{Úlohy}
  \begin{block}{Konverze II}
    \begin{itemize}
      \item vyzkoumejte, jak se chová \texttt{to\_f} a \texttt{to\_i} pro řetězce, které nejsou tak úplně číslo
    \end{itemize}
  \end{block}
  \pause
  \begin{block}{Palindrom}
    \begin{itemize}
      \item z libovolného řetězce vyrobte palindrom (osel $\to$ oselleso)
      \item z libovolného řetězce vyrobte palindrom s lichým počtem znaků (osel $\to$ oseleso)
    \end{itemize}
  \end{block}
\end{frame}

\begin{frame}[fragile]{Úlohy}
  \begin{block}{Palindrom / řešení}
\begin{verbatim}
  s = "osel"

  puts s + s.reverse
  puts s[0..-2] + s.reverse
\end{verbatim}
  \end{block}
\end{frame}

\subsection{Vstup a výstup}

\begin{frame}[fragile]{Výpis z účtu}
  \begin{block}{Tiskem}
    \begin{verbatim}
      print "jedna"
      puts "dve"
    \end{verbatim}
  \end{block}
  \pause
  \begin{block}{Inspektor Clouseau}
    \begin{verbatim}
      puts "2 + 2 = #{2+2}"
      puts [1,2,3].inspect
    \end{verbatim}
  \end{block}
\end{frame}

\begin{frame}[fragile]{Cyklistika}
  \begin{block}{Jednoduchý rozsah}
    \begin{verbatim}
      (1..5).each do
        puts "Cislo"
      end
    \end{verbatim}
  \end{block}
  \pause
  \begin{block}{S polem a proměnnou}
    \begin{verbatim}
      [1, 2, 3].each do |i|
        puts "Cislo #{i}"
      end
    \end{verbatim}
  \end{block}
\end{frame}

\begin{frame}{Úlohy}
  \begin{itemize}
    \item vypište prvních deset druhých mocnin (1 * 1 = 1, 2 * 2 = 4 atd.)
    \item vypište malou násobilku
    \item vypište prvních N členů Fibonacciho posloupnosti (1, 1, 2, 3, 5, 8 ...)
    \item metodou Erathostenova síta nalezněte prvočísla menší než N
  \end{itemize}
\end{frame}

\begin{frame}[fragile]{Úlohy}
  \begin{block}{Mocniny / řešení}
\begin{verbatim}
  (1..10).each do |x|
    print x
    print " * "
    print x
    print " = "
    puts x*x
  end
\end{verbatim}
  \end{block}
  \pause
  \begin{block}{Mocniny / lepší řešení}
\begin{verbatim}
  (1..10).each do |x|
    puts "#{x} * #{x} = #{x*x}"
  end
\end{verbatim}
  \end{block}
\end{frame}

\begin{frame}[fragile]{Úlohy}
  \begin{block}{Násobilka / řešení}
\begin{verbatim}
(1..10).each do |a|
  (1..10).each do |b|
    puts "#{b} * #{a} = #{a*b}"
  end
end
\end{verbatim}
  \end{block}
  \pause
  \begin{block}{Násobilka / jiné řešení}
\begin{verbatim}
(1..10).each do |a|
  (1..10).each do |b|
    puts "%2d * %2d = %3d" % [b, a, a * b]
  end
end
\end{verbatim}
  \end{block}
\end{frame}

\begin{frame}[fragile]{Úlohy}
  \begin{block}{Fibonacci / řešení}
\begin{verbatim}
a, b = 1, 1

20.times do
  c = a + b
  puts a
  a = b
  b = c
end
\end{verbatim}
  \end{block}
\end{frame}

\begin{frame}[fragile]{Úlohy}
  \begin{block}{Erathostenes / řešení}
\begin{verbatim}
n = 100

ary = (2..n).to_a
ary.each do |x|
  y = x
  while y <= n
    y += x
    ary.delete(y)
  end
end

puts ary.inspect
\end{verbatim}
  \end{block}
\end{frame}

\begin{frame}[fragile]{Česko čte dětem}
  \begin{block}{Šikovný iterátor}
    \begin{verbatim}
      IO.foreach("data.txt") do |line|
        ...
      end
    \end{verbatim}
  \end{block}
  \pause
  \begin{block}{V kuse}
    \begin{verbatim}
      string = IO.read("data.txt")
    \end{verbatim}
  \end{block}
\end{frame}

\begin{frame}[fragile]{V podmínce}
  \begin{block}{If nebo Unless}
    \scriptsize
    \begin{verbatim}
      if "velikost".include?("kost")
        puts "s kosti"
      end
      unless 7 > 8
        puts "poporadku"
      end
    \end{verbatim}
  \end{block}
  \pause
  \begin{block}{Přirozený jazyk}
    \scriptsize
    \begin{verbatim}
      puts "je tam!" if "podvodnik".include? "vodnik"
      puts "pocty" unless 2 + 2 == 5
      a = [1]
      a << a.last * 2 while a.size < 10
    \end{verbatim}
  \end{block}
\end{frame}

\begin{frame}{Úlohy}
  V souboru \texttt{data/text\_1.txt}:
  \begin{itemize}
    \item spočítejte všechny řádky
    \item spočítejte všechny řádky s výskytem slova kapr
    \item spočítejte počet výskytů slova kapr (po řádcích i v kuse)
  \end{itemize}
\end{frame}

\begin{frame}[fragile]{Úlohy}
  \begin{block}{Kapři / řešení}
    \scriptsize
\begin{verbatim}
  n, n_kapr, nn_kapr = 0, 0, 0
  IO.foreach("../data/text_1.txt") do |line|
    n += 1
    n_kapr += 1 if line.include?("kapr")
    nn_kapr += line.count("kapr")
  end

  nn_kapr_bis = IO.read("../data/text_1.txt").count("kapr")

  puts "Celkem radku: #{n}"
  puts "Radku s kaprem: #{n_kapr}"
  puts "Celkem kapru: #{nn_kapr}"
  puts "        nebo: #{nn_kapr_bis}"
\end{verbatim}
  \end{block}
\end{frame}

\begin{frame}[fragile]{Zápis do katastru}
  \begin{block}{Soubor se otevře a pak už to známe}
    \begin{verbatim}
      f = File.open("text.txt", 'w')
      f.puts "Nazdar!"
      f.close
    \end{verbatim}
  \end{block}
  \pause
  \begin{block}{The Ruby way}
    \begin{verbatim}
      File.open("text.txt", 'w') do |f|
        f.puts "Nazdar!"
      end
    \end{verbatim}
  \end{block}
\end{frame}

\begin{frame}{Úlohy}
  Z dat v souboru \texttt{data/data\_two\_1.csv}:
  \begin{itemize}
    \item vyberte pouze druhý sloupec
    \item sečtěte oba sloupce do jednoho
    \item vypočtěte součet obou sloupců
    \item vypočtěte průměr a RMS druhého sloupce
  \end{itemize}
  S hvězdičkou:
  \begin{itemize}
    \item použijte soubory \texttt{*multi*}
    \item proveďte pro všechny čtyři CSV soubory
  \end{itemize}
\end{frame}

\begin{frame}[fragile]{Úlohy}
  \begin{block}{CSV(1) / řešení}
    \scriptsize
\begin{verbatim}
  File.open("druhy_sloupec.csv", 'w') do |f|
    IO.foreach("../data/data_two_1.csv") do |line|
      f.puts line.strip.split[1]
    end
  end
\end{verbatim}
  \end{block}
\end{frame}

\begin{frame}[fragile]{Úlohy}
  \begin{block}{CSV(2) / řešení}
    \scriptsize
\begin{verbatim}
  File.open("sectene_sloupce.csv", 'w') do |f|
    IO.foreach("../data/data_two_1.csv") do |line|
      f.puts line.strip.split[0].to_f + line.strip.split[1].to_f
    end
  end
\end{verbatim}
  \end{block}
\end{frame}

\begin{frame}[fragile]{Úlohy}
  \begin{block}{CSV(3) / řešení}
    \scriptsize
\begin{verbatim}
  x0 = 0
  x1 = 0
  n = 0
  IO.foreach("../data/data_two_1.csv") do |line|
    x0 += line.strip.split[0].to_f
    x1 += line.strip.split[1].to_f
    n += 1
  end
  puts "Prvni sloupec: soucet #{x0}"
  puts "Druhy sloupec: soucet #{x1}"
\end{verbatim}
  \end{block}
\end{frame}

\begin{frame}[fragile]{Úlohy}
  \begin{block}{CSV(4) / řešení}
    \scriptsize
\begin{verbatim}
...
a0 = x0 / n
a1 = x1 / n

rms0 = 0
rms1 = 1
IO.foreach("../data/data_two_1.csv") do |line|
  rms0 += (line.strip.split[0].to_f - a0) ** 2
  rms1 += (line.strip.split[1].to_f - a1) ** 2
  n += 1
end
rms0 = (rms0 / n) ** 0.5
rms1 = (rms1 / n) ** 0.5
puts "Prvni sloupec: RMS #{rms0}"
puts "Druhy sloupec: RMS #{rms1}"
\end{verbatim}
  \end{block}
\end{frame}


\subsection{Problém č. 2: jehla v kupce sena}

\begin{frame}{Zadání}
  \begin{block}{\# 2}
    Adresář plný CSV souborů (stovky souborů) obsahuje data, která jsou záznamy signálů s lineární závislostí.
    
    V pěti z nich jsou ale poruchy - data ležící zcela mimo přímku.
    
    Kde?
  \end{block}
\end{frame}

\begin{frame}{Příklad - dobrý signál}
  \begin{center}
      \includegraphics[width=0.6\columnwidth]{search_good}
      \end{center}
\end{frame}
\begin{frame}{Příklad - špatný signál}
  \begin{center}
      \includegraphics[width=0.6\columnwidth]{search_bad}
      \end{center}
\end{frame}

\begin{frame}{Řešení}
  \begin{itemize}
    \item stačí vykreslit grafy pro všechny
    \item \texttt{Dir} pro najití souborů
    \item připravit a spustit \texttt{gnuplot}  
  \end{itemize}
\end{frame}

\begin{frame}{A to je vše, přátelé!}
  \begin{center}
    \includegraphics[width=0.8\textwidth]{looney_tunes}
  \end{center}
\end{frame}

\end{document}
