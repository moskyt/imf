\documentclass{beamer}

\def\Tiny{\fontsize{6pt}{6pt}\selectfont}
\def\supertiny{\fontsize{4pt}{4pt}\selectfont}

\mode<presentation>
{
  \usetheme{Warsaw}
  % \setbeamercovered{transparent}
  \usecolortheme{crane}
}

\usepackage{graphicx, ifthen, listings, fancyvrb}

\usepackage[czech]{babel}
% \usefonttheme{professionalfonts}
\usepackage{times}
\usepackage{amsmath}
\usepackage[utf8]{inputenc}
\usepackage{wrapfig}

\usepackage[T1]{fontenc}

\lstset{ basicstyle=\tiny, stringstyle=\ttfamily, showstringspaces=false }

\everymath{\displaystyle}

\setbeamerfont{frametitle}{size=\large}
\setbeamerfont{subsection in toc}{size=\scriptsize}

\makeatletter\newenvironment{blackbox}{%
   \begin{lrbox}{\@tempboxa}\begin{minipage}{0.95\columnwidth}}{\end{minipage}\end{lrbox}%
   \colorbox{black}{\usebox{\@tempboxa}}
}\makeatother

\title[IMF (11)]{Informatika pro moderní fyziky (11)\\ web scraping; serializace; zadání zápočtových úloh}

\author[Franti\v{s}ek HAVL\r{U}J, ORF ÚJV Řež]{Franti\v{s}ek HAVL\r{U}J\\{\scriptsize \emph{e-mail: haf@ujv.cz}}}

\date{akademický rok 2014/2015\\3. prosince 2014}

\institute[ORF ÚJV Řež]
{ÚJV Řež\\oddělení Reaktorové fyziky a podpory palivového cyklu}

\AtBeginSection[]
{
\begin{frame}<beamer>
\frametitle{Obsah}
\tableofcontents[currentsection,hideothersubsections]
\end{frame}
}

\begin{document}

\begin{frame}
  \titlepage
\end{frame}

\begin{frame}
  \tableofcontents
\end{frame}

\section{Navážeme na předminulou hodinu}

\begin{frame}{HTML scraping}
  \begin{itemize}
    \item získávání 
    \item 
    \item pozor, možné úskalí:
  \end{itemize}
\end{frame}

\begin{frame}{Připomenutí obecného postupu}
  \begin{itemize}
    \item 
    \item 
    \item 
  \end{itemize}
\end{frame}

\begin{frame}{Nalezení vhodného selektoru}
  \begin{itemize}
    \item 
    \item 
    \item 
  \end{itemize}
\end{frame}


\begin{frame}[fragile]{Práce s XML / HTML}
  Knihovna \texttt{nokogiri}
  \scriptsize
\begin{verbatim}
  doc = Nokogiri::HTML(File.open("redmeat.html"))
  doc.css("li.archiveImage a").each do |x|
    url = x.attributes['href']
    ...
  end
\end{verbatim}
\end{frame}

\begin{frame}[fragile]{Knihovna open-uri}
umožňuje otevírat URL jako soubory
  \scriptsize
\begin{verbatim}
require 'open-uri'
doc = Nokogiri::HTML(File.open("http://redmeat.com"))
\end{verbatim}
\end{frame}

\begin{frame}[fragile]{Stahování dat}
  Součást standardní knihovny -- \texttt{open-uri}
  \scriptsize
\begin{verbatim}
  require 'open-uri'
  File.open(local_filename, 'wb') do |f2|
    open(remote_url, 'rb') do |f1|
      f2.write f1.read
    end
  end
\end{verbatim}
\end{frame}


\begin{frame}{Serializace dat}
  \begin{itemize}
    \item
    \item
    \item
  \end{itemize}
\end{frame}






\begin{frame}{A to je vše, přátelé!}
  \begin{center}
    \includegraphics[width=0.8\textwidth]{looney_tunes}
  \end{center}
\end{frame}

\end{document}
