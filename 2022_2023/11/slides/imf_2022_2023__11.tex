\documentclass{beamer}

\def\Tiny{\fontsize{6pt}{6pt}\selectfont}
\def\supertiny{\fontsize{4pt}{4pt}\selectfont}

\mode<presentation>
{
  \usetheme{Warsaw}
  % \setbeamercovered{transparent}
  \usecolortheme{crane}
}

\usepackage{graphicx, ifthen, listings, fancyvrb}

\usepackage[czech]{babel}
% \usefonttheme{professionalfonts}
\usepackage{times}
\usepackage{amsmath}
\usepackage[utf8]{inputenc}
\usepackage{wrapfig}

\usepackage[T1]{fontenc}

\lstset{ basicstyle=\tiny, stringstyle=\ttfamily, showstringspaces=false }

\everymath{\displaystyle}

\setbeamerfont{frametitle}{size=\large}
\setbeamerfont{subsection in toc}{size=\scriptsize}

\makeatletter\newenvironment{blackbox}{%
   \begin{lrbox}{\@tempboxa}\begin{minipage}{0.95\columnwidth}}{\end{minipage}\end{lrbox}%
   \colorbox{black}{\usebox{\@tempboxa}}
}\makeatother

\title[IMF (11)]{Informatika pro moderní fyziky (11)\\ Použití cizích API, interaktivní mapa}

\author[Franti\v{s}ek HAVL\r{U}J, ORF ÚJV Řež]{Franti\v{s}ek HAVL\r{U}J\\{\scriptsize \emph{e-mail: haf@ujv.cz}}}

\date{akademický rok 2022/2023\\14. prosince 2022}

\institute[ORF ÚJV Řež]
{ÚJV Řež\\oddělení Reaktorové fyziky a podpory palivového cyklu}

\AtBeginSection[]
{
\begin{frame}<beamer>
\frametitle{Obsah}
\tableofcontents[currentsection,hideothersubsections]
\end{frame}
}

\begin{document}

\begin{frame}
  \titlepage
\end{frame}

\begin{frame}
  \tableofcontents
\end{frame}

\section{Co jsme se naučili minule}

\begin{frame}{}
  \begin{itemize}
    \item základy tvorby obrázků v SVG
    \item využití JSON a YAML souborů
    \item základy interaktivní mapy
  \end{itemize}
\end{frame}


\section{Použití cizích API}

\begin{frame}{K čemu to?}
  \begin{itemize}
    \item spousta informací na webu je poskytována ve strojově čitelné formě
    \item API -- rozhraní mezi aplikacemi
    \item s využitím webových služeb naše možnosti exponenciálně rostou (počasí, doprava, mapy, atd atd.)
    \item spousta věcí se dá udělat jako \emph{mashup} -- sice nic neumím, ale umím to dát dohromady
  \end{itemize}
\end{frame}


\begin{frame}{Typy / formáty}
  \begin{itemize}
    \item URL -- rovnou dostanu např. obrázek po zadání správného URL
    \item XML -- velmi obecný, ale komplikovaný formát (``vypadá jako HTML'')
    \item JSON -- velmi jednoduchý a kompaktní formát, vyvinutý pro JS (v podstatě jen číslo, řetězec, pole, hash)
  \end{itemize}
\end{frame}


\begin{frame}{URL API}
  \begin{itemize}
    \item stačí správně vymyslet URL a je to tím vyřešené
    \item pozor na usage limits (v produkci je nutné lokální cache...)
    \item QR platba: \\
    {\tiny \url{http://qr-platba.cz/pro-vyvojare/restful-api/\#generator-czech-image}}
    \item cestu a dotaz odděluje otazník, dotaz je tvaru \texttt{a=b\&c=d}
    \item zkuste vygenerovat URL pro platbu na svůj účet s poznámkou
    \item \texttt{URI.encode\_www\_form}
    \item nebo Google Maps static API
  \end{itemize}
\end{frame}

\begin{frame}[containsverbatim]{Jednoduchý mashup: mapa o-závodů}
  \begin{itemize}
    \item ORIS API -- http://oris.orientacnisporty.cz/API
    \item úkol: vypišme kalendář MTBO závodů v roce 2022
    \item {\tiny \url{http://oris.orientacnisporty.cz/API/?format=json\&method=getEventList\&sport=3\&datefrom=2022-01-01\&dateto=2022-12-31}}
  \end{itemize}

  \begin{verbatim}
    require "open-uri"
    require "json"

    url = "http://oris.orientacnisporty.cz/API/?format=json&method=getEventList&sport=3&datefrom=2022-01-01&dateto=2022-12-31"
    s = URI.open(url).read
    data = JSON[s]
  \end{verbatim}
\end{frame}

\section{Interaktivní mapa}

\begin{frame}{}
  \begin{itemize}
    \item vykresleme klikací mapu MTBO závodů
    \item použijeme mapy.cz
    \item chceme to ovšem trochu dotáhnout k dokonalosti – např. slučování závodů na stejném místě do jednoho bodu
    \item vygenerujeme mapy pro různé roky (třeba 2018 až 2022)
    \item když to budeme mít, jako bonus budeme vypisovat i vítězku ženské elitní kategorie (W21E)
  \end{itemize}
\end{frame}

\begin{frame}{Zdroj dat}
  \begin{itemize}
    \item JSON API systému ORIS je na \url{http://oris.orientacnisporty.cz/API/?format=json\&method=getEventList\&sport=3\&datefrom=2017-01-01\&dateto=2017-12-31}
    \item API pro výsledky: \url{https://oris.orientacnisporty.cz/API/?format=json&method=getEventResults&eventid=2077}
    \item hodí se použít nějaký add-on do prohlížeče na JSON, je s tím pak míň práce
    \item alternativně si vyrobíme svůj hezkovypisovač s použitím \texttt{JSON.pretty\_generate}
  \end{itemize}
\end{frame}

\begin{frame}{Lokální cache}
  \begin{itemize}
    \item je pomalé a nešikovné stahovat pokaždé data, takže chceme lokální cache pro stažená data
    \item uvědomíme si, že vlastně nemusíme řešit JSON a stačí jen uložit data
    \item případně můžeme vytahat z dat to, co potřebujeme a uložit si to do JSONu už zpracované
  \end{itemize}
\end{frame}

\section{Jak na to}

\begin{frame}{Mapa - HTML - JS - Erb}
  \begin{itemize}
    \item použijeme příklad \texttt{mapa.html}
    \item ideální postup: nejdřív si potřebnou funkcionalitu (značka, vizitka ...) vytvořím ručně
    \item až pak to začnu řešit ve skriptu
    \item -- protože když neumím JavaScript, tak je potřeba s tím zacházet trochu opatrně
    \item rozhodně použijme erb
  \end{itemize}
\end{frame}

\begin{frame}{Příklady JS pro to, co potřebujeme}
  \begin{itemize}
    \item značky: \url{https://api.mapy.cz/view?page=markerlayer}
    \item s vizitkou: \url{https://api.mapy.cz/view?page=markercard}
    \item komplexnější vizitky: \url{https://api.mapy.cz/view?page=card}
  \end{itemize}
\end{frame}

\begin{frame}{Na co nezapomenout při zpracování dat}
  \begin{itemize}
    \item budeme chtít sdružovat závody podle místa
    \item vzpomeneme si: pokud seskupuju cokoliv podle jednoznačného klíče, použiju hash
    \item získání výsledků: pomocí ID závodu \url{https://oris.orientacnisporty.cz/API/?format=json&method=getEventResults&eventid=2077}
  \end{itemize}
\end{frame}

\begin{frame}{A to je vše, přátelé!}
  \begin{center}
    \includegraphics[width=0.8\textwidth]{looney_tunes}
  \end{center}
\end{frame}

\end{document}
