\documentclass{beamer}

\def\Tiny{\fontsize{6pt}{6pt}\selectfont}
\def\supertiny{\fontsize{4pt}{4pt}\selectfont}

\mode<presentation>
{
  \usetheme{Warsaw}
  % \setbeamercovered{transparent}
  \usecolortheme{crane}
}


\usepackage{graphicx, ifthen, listings}

\usepackage[czech]{babel}
% \usefonttheme{professionalfonts}
\usepackage{times}
\usepackage{amsmath}
\usepackage[utf8]{inputenc}
\usepackage{wrapfig}

\usepackage[T1]{fontenc}

\lstset{ basicstyle=\tiny, stringstyle=\ttfamily, showstringspaces=false }

\everymath{\displaystyle}

\setbeamerfont{frametitle}{size=\large}
\setbeamerfont{subsection in toc}{size=\scriptsize}

\makeatletter\newenvironment{blackbox}{%
   \begin{lrbox}{\@tempboxa}\begin{minipage}{0.95\columnwidth}}{\end{minipage}\end{lrbox}%
   \colorbox{black}{\usebox{\@tempboxa}}
}\makeatother

\title[IMF (5)]{Informatika pro moderní fyziky (5)\\výstupní a vstupní soubory pro výpočetní programy, datové struktury}

\author[Franti\v{s}ek HAVL\r{U}J, ORF ÚJV Řež]{Franti\v{s}ek HAVL\r{U}J\\{\scriptsize \emph{e-mail: haf@ujv.cz}}}

\date{akademický rok 2022/2023, 26. října 2022}

\institute[ORF ÚJV Řež]
{ÚJV Řež\\oddělení Reaktorové fyziky a podpory palivového cyklu}

\AtBeginSection[]
{
\begin{frame}<beamer>
\frametitle{Obsah}
\tableofcontents[currentsection,hideothersubsections]
\end{frame}
}

\begin{document}

\begin{frame}
  \titlepage
\end{frame}

\begin{frame}
  \tableofcontents
\end{frame}

\section{Co jsme se naučili minule}

\begin{frame}{}
  \begin{itemize}
    \item práci s datovými soubory
    \item zpracování většího množství dat
    \item rozšíření RubyGems, vytvoření tabulky v Excelu
    \item základ získání dat z výstupního souboru
    \item definice vlastní metody
  \end{itemize}
\end{frame}

\section{Načítání výstupních dat – dokončení}

\begin{frame}[fragile]{Skončili jsme u takovéto tabulky}
  něco jak toto:
  {\scriptsize
  \begin{verbatim}
    0   0  0.94800
    0 640  0.99800
    0  64  0.94850
    0 128  0.95000
    0 192  0.95250
    0 256  0.95600
    ...
  \end{verbatim}
  }
  -- kdo nemá, tak je v souboru \texttt{mcnp/keff.csv}
\end{frame}

\begin{frame}[fragile]{A protože přehlednost je nade vše}
  rádi bychom měli něco jak toto:
  {\scriptsize
  \begin{verbatim}
    keff           0           64          128    ...
      0      0.94800      0.94900      0.95000    ...
     64      0.94850      0.94950      0.95050    ...
    128      0.95000      0.95100      0.95200    ...
    192      0.95250      0.95350      0.95450    ...
    256      0.95600      0.95700      0.95800    ...
    320      0.96050      0.96150      0.96250    ...
    384      0.96600      0.96700      0.96800    ...
    448      0.97250      0.97350      0.97450    ...
    512      0.98000      0.98100      0.98200    ...
    576      0.98850      0.98950      0.99050    ...
    640      0.99800      0.99900      1.00000    ...
  \end{verbatim}
  }
  -- ale jak nejlépe uložit data do 2D struktury?
\end{frame}

\begin{frame}[fragile]{1D a 2D pole}
  \textbf{První možnost} je mít normálně 1D pole a spočítat si index:\\
  \texttt{a[i + j * 11]}
  \pause


  \textbf{Druhá možnost} je `2D pole' -- ve skutečnosti v Ruby nic takového není, ale můžete mít pochopitelne pole polí (tj. každý prvek pole může být klidně pole, proč ne):
  \texttt{a[i][j]}
  \pause

  Problém je, že data nemám seřazená, takže moc nemáme dobrý způsob, jak je postupně do matice nastrkat.

  Jedna možnost je nejdřív si to pole vyrobit (prvky budou nil nebo 0 nebo tak něco) a pak do něj teprv zapisovat.
  Druhá možnost je zapomenout na pole a udělat si to jednodušší.
\end{frame}

\begin{frame}[fragile]{Hash to the rescue}
  Možná se bude velmi hodit hash!
  {\scriptsize
  \begin{verbatim}
    data = {}
    data[3] = 0.99
  \end{verbatim}
  }
  \pause
  no a dokonce, protože klíč může být cokoliv:
  {\scriptsize
  \begin{verbatim}
    data = {}
    data[[1,2]] = 0.99
    data[[7,8]] = 1.05
    puts data[[1,2]]
  \end{verbatim}
  }
  Takže nic nebrání tomu si naskládat data do hashe (= jakobymatice s volnou strukturou) a vypsat krásnou tabulku.
\end{frame}

\begin{frame}{Navážeme na úspěchy z minulých týdnů}
  Zopakujeme, prohloubíme, rozšíříme, nelenivíme, nezapomínáme.
  \begin{itemize}
    \item data jsou v `keff.csv'
    \item vykreslit graf! pro každou z 11 poloh R1 jedna čára (závislost keff na R2)
    \item (= csv soubor, gnuplot, znáte to)
    \item najít automaticky kritickou polohu R2 pro každou z 11 poloh R1
  \end{itemize}
\end{frame}


\section{Načítání složitějšího výstupu}

\begin{frame}[fragile]{HELIOS}
  Tabulka výstupů:
  \scriptsize
  \begin{verbatim}
List name       : list
List Title(s)  1) This is a table
               2) of some data
               3) in many columns
               4) and has a long title!

               bup        kinf          ab          ab        u235        u238       pu239
 0001     0.00E+00     1.16949  9.7053E-03  7.6469E-02  1.8806E-04  7.5509E-03  1.0000E-20
 0002     0.00E+00     1.13213  9.7478E-03  7.9058E-02  1.8806E-04  7.5509E-03  1.0000E-20
 0003     1.00E+01     1.13149  9.7488E-03  7.9070E-02  1.8797E-04  7.5509E-03  2.3647E-09
 0004     5.00E+01     1.13004  9.7521E-03  7.9093E-02  1.8760E-04  7.5506E-03  5.3144E-08
 0005     1.00E+02     1.12826  9.7559E-03  7.9218E-02  1.8714E-04  7.5503E-03  1.8746E-07
 0006     1.50E+02     1.12664  9.7594E-03  7.9407E-02  1.8668E-04  7.5500E-03  3.7495E-07
 0007     2.50E+02     1.12399  9.7657E-03  7.9869E-02  1.8577E-04  7.5493E-03  8.4139E-07
 0008     5.00E+02     1.12007  9.7812E-03  8.1065E-02  1.8351E-04  7.5476E-03  2.1882E-06
 0009     1.00E+03     1.11561  9.8203E-03  8.3169E-02  1.7914E-04  7.5443E-03  4.8723E-06
 0010     2.00E+03     1.10542  9.9329E-03  8.6731E-02  1.7088E-04  7.5376E-03  9.6873E-06
 0011     3.00E+03     1.09354  1.0067E-02  8.9717E-02  1.6316E-04  7.5309E-03  1.3879E-05
 0012     4.00E+03     1.08126  1.0207E-02  9.2299E-02  1.5591E-04  7.5242E-03  1.7571E-05
 0013     6.00E+03     1.05755  1.0474E-02  9.6562E-02  1.4258E-04  7.5105E-03  2.3760E-05
  \end{verbatim}
\end{frame}

\begin{frame}{Co bychom chtěli}
  \begin{itemize}
    \item mít načtené jednotlivé tabulky (zatím jen jednu, ale bude jich víc)
    \item asi po jednotlivých sloupcích, sloupec = pole (hodnot po řádcích)
    \item sloupce se nějak jmenují, tedy použijeme \texttt{Hash}
    \item \texttt{table["kinf"]}
    \item pozor na \texttt{ab}, asi budeme muset vyrobit něco jako \texttt{ab1}, \texttt{ab2} (ale to až za chvíli)
  \end{itemize}
\end{frame}

\begin{frame}{Nástrahy, chytáky a podobně}
  \begin{itemize}
    \item tabulka skládající se z více bloků
    \item více tabulek
    \item tabulky mají jméno \texttt{list name} a popisek \texttt{list title(s)}
  \end{itemize}
\end{frame}

\begin{frame}[fragile]{Jak uspořádat data?}
  \begin{itemize}
    \item pole s tabulkami + pole s názvy + pole s titulky?
    \pause
    \item co hashe tabulky[název] a titulky[název]?
    \pause
    \item nejchytřeji: \verb!{"a"=>{:title => "Table title",! \\ \verb!:data => {"kinf"=>...}}}!
    \pause
    \item ``nová'' syntaxe: \verb!{"a"=>{title: "Table title",! \\ \verb!data: {"kinf"=>...}}}!
  \end{itemize}
\end{frame}

\begin{frame}{Z příkazové řádky}
  \begin{itemize}
    \item a co takhle z toho udělat skript, který lze pustit s~argumentem = univerzální
    \item \texttt{ruby read\_helios.rb helios1.out}
    \item vypíše seznam všech tabulek, seznam jejich sloupců, počet řádků
    \item pole \texttt{ARGV} -- seznam všech argumentů
    \item vylepšení – provede pro všechny zadané soubory: \texttt{ruby read\_helios.rb helios1.out helios2.out} (tip: využívejte vlastní metody, kde to jen jde)
  \end{itemize}
\end{frame}

\section{Automatizace tvorby vstupů}

\begin{frame}[fragile]{Určení poloh tyčí}
  Ve vstupním souboru si najdeme relevantní část:
  \scriptsize
  \begin{verbatim}
    c ---------------------------------
    c polohy tyci (z-plochy)
    c ---------------------------------
    c
    67 pz 47.6000    $ dolni hranice absoberu r1
    68 pz 40.4980    $ dolni hranice hlavice r1
    69 pz 44.8000    $ dolni hranice absoberu r2
    70 pz 37.6980    $ dolni hranice hlavice r2
  \end{verbatim}
\end{frame}

\begin{frame}[fragile]{Výroba šablon}
  Jak dostat polohy tyčí do vstupního souboru? Vyrobíme šablonu, tzn nahradíme
  \begin{verbatim}
    67 pz 47.6000    $ dolni hranice absoberu r1
  \end{verbatim}
  \pause
  nějakou značkou (\emph{placeholder}):
  \begin{verbatim}
    67 pz %r1%    $ dolni hranice absoberu r1
  \end{verbatim}
\end{frame}

\begin{frame}{Chytáky a zádrhele}
  \begin{itemize}
    \item kromě samotné plochy konce absorbéru je nutno správně umístit i z-plochu konce hlavice o 7,102 cm níže
    \item obecně je na místě ohlídat si, že placeholder nebude kolidovat s ničím jiným
  \end{itemize}
  Doporučené nástroje jsou:
  \begin{itemize}
    \item již známá funkce \texttt{sub} pro nahrazení jednoho řetězce jiným
    \item pro pragmatické lenochy funkce \texttt{File.read} načítající celý soubor do řetězce
    \item možno ovšem použít i \texttt{File.readlines} (v čem je to lepší?)
  \end{itemize}
\end{frame}

\begin{frame}[fragile]{Realizace}
  \scriptsize
  \begin{verbatim}
    delta = 44.8000 - 37.6980

    template = File.read("template")
    (0..10).each do |i1|
      (0..10).each do |i2|
        r1 = i1 * 50
        r2 = i2 * 50
        s = template.sub("%r1%", r1.to_s)
        s = s.sub("%r1_%", (r1 - delta).to_s)
        s = s.sub("%r2%", r2.to_s)
        s = s.sub("%r2_%", (r2 - delta).to_s)
        File.write("inputs/c_#{i1}_#{i2}", s)
      end
    end
  \end{verbatim}
\end{frame}

\section{Automatizace tvorby vstupů -- zobecnění}

\begin{frame}{A co takhle trocha zobecnění?}
  \begin{itemize}
    \item když budu chtít přidat další tyče nebo jiné parametry, bude to děsně bobtnat
    \item funkce \texttt{process(``template'', ``inputs/c\_\#\{i1\}\_\#\{i2\}'', \{``r1'' => r1, ``r2'' => r2, .....\})}
    \item všechno víme, známe, umíme...
    \item rozšiřte tak, že třeba tyč B1 bude mezi R1 a R2, B2 mezi R1 a B1, B3 mezi R2 a dolní hranicí palivového článku (Z = 1 cm)
  \end{itemize}
\end{frame}

\begin{frame}{A to je vše, přátelé!}
  \begin{center}
    \includegraphics[width=0.8\textwidth]{looney_tunes}
  \end{center}
\end{frame}

\end{document}
