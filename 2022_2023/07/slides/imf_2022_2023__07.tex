\documentclass{beamer}

\def\Tiny{\fontsize{6pt}{6pt}\selectfont}
\def\supertiny{\fontsize{4pt}{4pt}\selectfont}

\mode<presentation>
{
  \usetheme{Warsaw}
  % \setbeamercovered{transparent}
  \usecolortheme{crane}
}


\usepackage{graphicx, ifthen, listings, fancyvrb}

\usepackage[czech]{babel}
% \usefonttheme{professionalfonts}
\usepackage{times}
\usepackage{amsmath}
\usepackage[utf8]{inputenc}
\usepackage{wrapfig}

\usepackage[T1]{fontenc}

\lstset{ basicstyle=\tiny, stringstyle=\ttfamily, showstringspaces=false }

\everymath{\displaystyle}

\setbeamerfont{frametitle}{size=\large}
\setbeamerfont{subsection in toc}{size=\scriptsize}

\makeatletter\newenvironment{blackbox}{%
   \begin{lrbox}{\@tempboxa}\begin{minipage}{0.95\columnwidth}}{\end{minipage}\end{lrbox}%
   \colorbox{black}{\usebox{\@tempboxa}}
}\makeatother

\title[IMF (7)]{Informatika pro moderní fyziky (7)\\vstupní soubory pro výpočetní programy, tvorba dokumentů}

\author[Franti\v{s}ek HAVL\r{U}J, ORF ÚJV Řež]{Franti\v{s}ek HAVL\r{U}J\\{\scriptsize \emph{e-mail: haf@ujv.cz}}}

\date{akademický rok 2022/2023, 9. listopadu 2022}

\institute[ORF ÚJV Řež]
{ÚJV Řež\\oddělení Reaktorové fyziky a podpory palivového cyklu}

\AtBeginSection[]
{
\begin{frame}<beamer>
\frametitle{Obsah}
\tableofcontents[currentsection,hideothersubsections]
\end{frame}
}

\begin{document}

\begin{frame}
  \titlepage
\end{frame}

\begin{frame}
  \tableofcontents
\end{frame}

\section{Co jsme se naučili minule}

\begin{frame}{}
  \begin{itemize}
    \item základy OOP
    \item načítání komplexního výstupu
    \item čtení souboru
  \end{itemize}
\end{frame}

\section{Automatizace tvorby vstupů}


\begin{frame}[fragile]{Určení poloh tyčí}
  V souboru \texttt{data/mcnp/c1\_1} máme příklad vstupního souboru.

  Po jeho vzoru budeme chtít vyrobit vstupní soubor, ale chceme do něj vložit polohy regulačních tyčí podle potřeby.
\end{frame}

\begin{frame}[fragile]{Určení poloh tyčí}
  Ve vstupním souboru si najdeme relevantní část:
  \scriptsize
  \begin{verbatim}
    c ---------------------------------
    c polohy tyci (z-plochy)
    c ---------------------------------
    c
    67 pz 47.6000    $ dolni hranice absoberu r1
    68 pz 40.4980    $ dolni hranice hlavice r1
    69 pz 44.8000    $ dolni hranice absoberu r2
    70 pz 37.6980    $ dolni hranice hlavice r2
  \end{verbatim}
\end{frame}

\begin{frame}[fragile]{Výroba šablon}
  Jak dostat polohy tyčí do vstupního souboru? Vyrobíme šablonu, tzn nahradíme
  \begin{verbatim}
    67 pz 47.6000    $ dolni hranice absoberu r1
  \end{verbatim}
  \pause
  nějakou značkou (\emph{placeholder}):
  \begin{verbatim}
    67 pz %{r1}    $ dolni hranice absoberu r1
  \end{verbatim}
\end{frame}

\begin{frame}{Chytáky a zádrhele}
  \begin{itemize}
    \item kromě samotné plochy konce absorbéru je nutno správně umístit i z-plochu konce hlavice o 7,102 cm níže
    \item obecně je na místě ohlídat si, že placeholder nebude kolidovat s ničím jiným
  \end{itemize}
  Doporučené nástroje jsou:
  \begin{itemize}
    \item již známá funkce \texttt{sub} pro nahrazení jednoho řetězce jiným
    \item pro pragmatické lenochy funkce \texttt{File.read} načítající celý soubor do řetězce
    \item možno ovšem použít i \texttt{File.readlines} (v čem je to lepší?)
  \end{itemize}
\end{frame}

\begin{frame}[fragile]{Realizace}
  \scriptsize
  \begin{verbatim}
    delta = 44.8000 - 37.6980

    template = File.read("template")
    (0..10).each do |i1|
      (0..10).each do |i2|
        r1 = i1 * 50
        r2 = i2 * 50
        s = template.sub("%{r1}", r1.to_s)
        s = s.sub("%{r1_}", (r1 - delta).to_s)
        s = s.sub("%{r2}", r2.to_s)
        s = s.sub("%{r2_}", (r2 - delta).to_s)
        File.write("inputs/c_#{i1}_#{i2}", s)
      end
    end
  \end{verbatim}
\end{frame}

\section{Automatizace tvorby vstupů -- zobecnění}

\begin{frame}[fragile]{Ale my už to od minula umíme objektově...}
  \scriptsize
  \begin{verbatim}
    r1 = 500
    r2 = 300
    t = Template.new("template", r1, r2)
    t.write("inputs/input_#{r1}_#{r2}")
  \end{verbatim}
\end{frame}

\begin{frame}[fragile]{A co kdyby to bylo ještě obecnější}
  \scriptsize
  \begin{verbatim}
    t = Template.new("template")
    t.r1 = 500
    t.r2 = 300
    t.b1 = 650
    t.b2 = 650
    t.b3 = 650
    t.write("inputs/input_#{r1}_#{r2}")
  \end{verbatim}
\end{frame}

% \begin{frame}{A co takhle trocha zobecnění?}
%   \begin{itemize}
%     \item když budu chtít přidat další tyče nebo jiné parametry, bude to děsně bobtnat
%     \item funkce \texttt{process(``template'', ``inputs/c\_\#\{i1\}\_\#\{i2\}'', \{``r1'' => r1, ``r2'' => r2, .....\})}
%     \item všechno víme, známe, umíme...
%     \item rozšiřte tak, že třeba tyč B1 bude mezi R1 a R2, B2 mezi R1 a B1, B3 mezi R2 a dolní hranicí palivového článku (Z = 1 cm)
%   \end{itemize}
% \end{frame}

\section{Závěrečná zpráva}

\begin{frame}{Jak vyrobit zprávu?}
  \begin{itemize}
    \item potřebujeme udělat hezké PDF shrnující výsledky našich výpočtů
    \item takže úvod, popis toho co jsme dělali a pak přehled výsledků
    \item tabulka s hodnotami, 11+1 graf
    \item co by znamenalo to dělat ve Wordu ..... ?
    \item hodilo by se to zautomatizovat!
  \end{itemize}
\end{frame}

\begin{frame}{Jak vygenerovat text?}
  \begin{itemize}
    \item zase potřebujeme lepší nástroj na text, než jsou WYSIWYG (What You See Is What You Get) editory
    \item ideálně něco, co bude mít plain-text vstup (který můžeme s úspěchem generovat v Ruby) a co se pak
    \item odpověď zní \LaTeX
    \item nejvíc nejlepší text-processor na světě
  \end{itemize}
\end{frame}

\begin{frame}{Koncepce oddělení obsahu a formy}
  \begin{itemize}
    \item když píšu, nechci říct, že je text tučně a o dva body větší, ale že je to nadpis kapitoly
    \item ideálně chci popsat někde, jak bude dokument vypadat a nemíchat vzhled s obsahem
    \item styly ve Wordu se tomu vzdáleně blíží
    \item v LaTeXu vlastně píšu jen obsah a o formu se musím starat jen hodně málo
    \item je samozřejmostí zadarmo obsah, rejstřík atd.
    \item kdo píše diplomku v něčem jiném, kazí si život
  \end{itemize}
\end{frame}

\begin{frame}{Příklad jednoduchého dokumentu}
  \begin{itemize}
    \item viz \texttt{document.tex}
    \item není potřeba úplně všemu rozumět, zatím to jen budeme upravovat v mezích zákona
    \item všechny příkazy začínají zpětným lomítkem, parametry jsou ve složených závorkách
    \item napíšu \texttt{pdflatex document.tex} a dostanu pdfko!
  \end{itemize}
\end{frame}

\begin{frame}{Text}
  \begin{itemize}
    \item konce řádku nejsou důležité
    \item nový odstavec se dělá prázdným řádkem
    \item nové (pod)kapitoly pomocí příkazů \texttt{section}, \texttt{subsection}
  \end{itemize}
\end{frame}

\begin{frame}[fragile]{Tabulka}
  \begin{itemize}
    \item prostředí (= \texttt{begin} ... \texttt{end})
    \item sloupce se
  \end{itemize}
      \scriptsize
  \begin{verbatim}
\begin{tabular}{ll}
a & b \\
c & d \\
\end{tabular}
  \end{verbatim}

\end{frame}

\begin{frame}{Pozor na backslash}
  \begin{itemize}
    \item v Ruby se v řetězci backslash \texttt{\textbackslash} používá jako escape character
    \item např konec řádky je \texttt{\textbackslash n}
    \item pokud chci vytiskout zpětné lomítko (což bude asi pro LaTeX potřeba), musím ho zdvojit: \texttt{\textbackslash \textbackslash}
  \end{itemize}
\end{frame}

\begin{frame}{Složitější práce}
  \begin{itemize}
    \item LaTeX je ideální pro rozsáhlé texty
    \item výzkumák, diplomka, disertace se naformátuje sama a všechno funguje bez trápení a snažení
    \item odkazy, reference, citace .. všechno bez starostí
  \end{itemize}
\end{frame}

\begin{frame}{Prezentace}
  \begin{itemize}
    \item balíček \texttt{beamer}
    \item bez nutnosti se uklikat to samo od sebe vypadá slušně
    \item viz tahle prezentace
    \item nevýhoda(?): obtížnost přizpůsobit rozmístění textu apod., ale na druhou stranu to aspoň drží jednotný styl
  \end{itemize}
\end{frame}

\section{Výroba dokumentu v praxi}

\begin{frame}{Úkol na dnešek}
  \begin{itemize}
    \item pro jeden blok JE mám provozní data - v určitých dnech hodnotu koncentrace kyseliny borité a axiálního ofsetu - pro několik kampaní (blíže neurčený počet)
    \item chci vyrobit přehledové PDF, které bude hezky prezentovat grafy obou veličin pro každou kampaň a k tomu i tabulky
    \item data pro jednotlivé kampaně mám v CSV souborech, každý má tři sloupce (datum, cB, AO)
  \end{itemize}
\end{frame}

\begin{frame}{Rozbor}
  \begin{itemize}
    \item načíst tabulky a vykreslit grafy umíme
    \item převést tabulky v CSV na tabulky v LaTeXu se záhy naučíme
    \item vložit obrázek do latexu taky umíme
    \item předem neznámý počet souborů nás netrápí (\texttt{Dir["*.csv"]})
  \end{itemize}
\end{frame}

\begin{frame}{Tak nejdřív ty grafy}
  \begin{itemize}
    \item to už je vážně obehraná písnička, ale tady je aspoň trochu změna
    \item potřebujeme vybrat, které dva sloupce použít - parametr \texttt{using}
    \item \texttt{plot ``data.csv'' using 1:2}
    \item datum na vodorovné ose -- potřeba načíst ve správném formátu atd.
    \item \texttt{set xdata time}
    \item \texttt{set timefmt ``\%m/\%d/\%Y''}
    \item s hvězdičkou: nastavit nadpis a popisky os
  \end{itemize}
\end{frame}

\begin{frame}[fragile]{Ne tak úplně chytře}
  \tiny
\begin{verbatim}
  Dir["*.csv"].each do |fn|
    base = fn.split(".").first

    File.open("#{base}_bc.gp", "w") do |f|
      f.puts "set terminal png"
      f.puts "set xdata time"
      f.puts "set timefmt \"%m/%d/%Y\""
      f.puts "set output \"#{base}_bc.png\""
      f.puts "plot \"#{base}.csv\" using 1:2"
    end
    `gnuplot #{base}_bc.gp`

    File.open("#{base}_ao.gp", "w") do |f|
      f.puts "set terminal png"
      f.puts "set output \"#{base}_ao.png\""
      f.puts "set xdata time"
      f.puts "set timefmt \"%m/%d/%Y\""
      f.puts "plot \"#{base}.csv\" using 1:3"
    end
    `gnuplot #{base}_ao.gp`
  end
\end{verbatim}
\end{frame}

\begin{frame}[fragile]{DRY}
  \begin{itemize}
    \item základní paradigma: DRY = don't repeat yourself
    \item nemá cenu psát dvě věci stejně
    \item použití \texttt{copy and paste} při programování je varovný signál
    \item pokud nejsou stejně, ale skoro stejně, je potřeba trochu chytrosti
    \item připomeňme si hash: \texttt{\{''a'' => 1, ''b'' => 2\}}
    \item přes hash se dá iterovat: \\ \texttt{\{''a'' => 1, ''b'' => 2\}.each do |key, value|}
  \end{itemize}
\end{frame}

\begin{frame}[fragile]{Grafy -- DRY}
  \tiny
  \begin{verbatim}
    Dir["*.csv"].each do |fn|
      base = fn.split(".").first
      {:bc => 2, :ao => 3}.each do |var, col|
        File.open("#{base}_#{var}.gp", "w") do |f|
          f.puts "set terminal png"
          f.puts "set xdata time"
          f.puts "set timefmt \"%m/%d/%Y\""
          f.puts "set output \"#{base}_#{var}.png\""
          f.puts "plot \"#{base}.csv\" using 1:#{col}"
        end
        `gnuplot #{base}_#{var}.gp`
      end
    end
  \end{verbatim}
\end{frame}

\begin{frame}[fragile]{Jak na tabulky}
  \begin{itemize}
    \item tabulky budou dost rozsáhlé a montovat je přímo nějak do latexových vstupů je asi spíš nepraktické, naštěstí to jde i jinak
    \item naštěstí má LaTex příkaz \texttt{\textbackslash{}input}, kterým můžeme prostě vložit do dokumentu nějaký externí soubor
    \item takže si nejdřív přichystáme soubory s tabulkami a pak se na ně budeme už jenom odkazovat
  \end{itemize}
\end{frame}

\begin{frame}[fragile]{Jak na tabulky v LaTeXu (1)}
  \begin{block}{ }
    Základem tabulky je prostředí \texttt{tabular} s definicí počtu a zarovnání sloupců:
    \scriptsize
    \begin{verbatim}
\begin{tabular}{lrr}
  ...
\end{tabular}
    \end{verbatim}
  \end{block}
\end{frame}

\begin{frame}[fragile]{Jak na tabulky v LaTeXu (2)}
  \begin{block}{ }
    Uvnitř tabulky se sloupce oddělují ampersandem a řádky dvojitým backslashem:
    \scriptsize
    \begin{verbatim}
\begin{tabular}{lrr}
  Data 1 & a & 1.0 \\
  Data 2 & b & 2.0 \\
  Data 3 & c & 3.0 \\
\end{tabular}
    \end{verbatim}
  \end{block}
\end{frame}

\begin{frame}[fragile]{Jak na tabulky v LaTeXu (3)}
  \begin{block}{ }
    Přidání mřížky je nesnadné, leč proveditelné a vlastně docela dobře vymyšlené - přidáváme jednotlivé čáry po sloupcích a řádcích:
    \scriptsize
    \begin{verbatim}
\begin{tabular}{|l|r|r|}
  \hline
  Data 1 & a & 1.0 \\
  \hline
  Data 2 & b & 2.0 \\
  \hline
  Data 3 & c & 3.0 \\
  \hline
\end{tabular}
    \end{verbatim}
  \end{block}
\end{frame}

\begin{frame}{Úkol na teď: výroba tabulek}
  \begin{itemize}
    \item vyrobit z CSV souboru (tři sloupce) dvě LaTeX tabulky (po dvou sloupcích)
    \item postarat se, aby byly hezké
    \item chytré je vyrobit tabulku třeba o šesti sloupcích (jakože tři dvousloupce), pak už se to na stránku v klidu vejde
  \end{itemize}
\end{frame}

% ukol ted: vytahnout z trisloupcoveho CSV dvousloupcovou latex tabulku
%
% (ukazat, s vysledkem)
%
% pak: do tri dvousloupcu
%
% (vysledek)
%
% (ukazat)


\section{Na šablony chytře}

\begin{frame}{Úskalí šablon}
  \begin{itemize}
    \item snadno umíme nahradit jeden řetězec druhým
    \item trochu méně pohodlné pro větší bloky textu
    \item navíc by se hodila nějaká logika (cyklus) přímo v šabloně
    \item naštěstí jsou na to postupy
  \end{itemize}
\end{frame}

\begin{frame}{ERb (Embedded Ruby)}
  \begin{itemize}
    \item lepší šablona - ``aktivní text''
    \item používá se například ve webových aplikacích
    \item hodí se ale i na generování latexových dokumentů, resp. všude, kde nám nesejde na whitespace
    \item poměrně jednoduchá syntax, zvládne skoro všechno
  \end{itemize}
\end{frame}


\begin{frame}[fragile]{Základní syntaxe ERb (1)}
  \begin{block}{ }
    Jakýkoli Ruby příkaz, přiřazení, výpočet ...
    \scriptsize
    \begin{verbatim}
      <% a = b + 5 %>
      <% list = ary * ", " %>
    \end{verbatim}
  \end{block}
\end{frame}

\begin{frame}[fragile]{Základní syntaxe ERb (2)}
  \begin{block}{ }
    Pokud chci něco vložit, stačí přidat rovnítko
    \scriptsize
    \begin{verbatim}
      <%= a %>
      <%= ary[1] %>
      <%= b + 5 %>
    \end{verbatim}
  \end{block}
\end{frame}

\begin{frame}[fragile]{Základní syntaxe ERb (3)}
  \begin{block}{ }
    Radost je možnost použít bloky a tedy i iterátory apod. v propojení s vkládaným textem:
    \scriptsize
    \begin{verbatim}
      <% (1..5).each do |i| %>
      Number <%= i %>
      <% end %>
      <% ary.each do |x| %>
      Array contains <%= x %>
      <% end %>
    \end{verbatim}
  \end{block}
\end{frame}

\begin{frame}{ERb -- shrnutí}
  \begin{itemize}
    \item dobrý sluha, ale špatný pán
    \item můžu s tím vyrobit hromadu užitečných věcí na malém prostoru
    \item daň je velké riziko zamotaného kódu a nízké přehlednosti (struktura naprosto není patrná na první pohled, proto je namístě ji držet maximálně jednoduchou)
  \end{itemize}
\end{frame}

\begin{frame}{Důležité upozornění}
  \begin{itemize}
    \item oddělení modelu a view
    \item přestože lze provádět zpracování dat a výpočty přímo v ERb, je to nejvíc nejhorší nápad
    \item je chytré si všechno připravit v modelu (tj. v Ruby skriptu, kterým data chystáme)
    \item a kód ve view (tj. v ERb šabloně) omezit na naprosté minimum
  \end{itemize}
\end{frame}

\begin{frame}[fragile]{Jak ze šablony udělat výsledek}
  \scriptsize
  \begin{block}{Příklad překladu ERb}
    \scriptsize
    \begin{verbatim}
require_relative "lib/erb_compiler.rb"

erb(template, filename, {:x => 1, :y => 2})
    \end{verbatim}
  \end{block}
\end{frame}

\begin{frame}[fragile]{Příklad -- kreslení grafůa}
  \begin{block}{template.gp}
    \scriptsize
    \begin{verbatim}
set terminal png
set output "plot_<%=n%>.png"
plot "data_<%=n%>.csv"
    \end{verbatim}
  \end{block}
  \begin{block}{}
    \scriptsize
    \begin{verbatim}
(1..10).each do |i|
  erb("template.gp", "plot_#{i}.gp", {:n => i})
end
    \end{verbatim}
  \end{block}
\end{frame}

\begin{frame}[fragile]{Takže v latexu třeba}
\scriptsize
\begin{verbatim}
  \subsection{Koncentrace kyseliny borité}

  <% files.each do |f| %>

  \subsubsection{Kampaň <%= f.split("_").last %>}
  \begin{center}
  \includegraphics[width=0.8\textwidth]{<%= f %>_bc.eps}
  \end{center}
  <% end %>
\end{verbatim}
\end{frame}

\begin{frame}{A teď už to jenom dejte dohromady...}
  \begin{enumerate}
    \item připravit si základní kostru dokumentu v latexu
    \item převést na šablonu: mít seznam souborů, správně generovat kapitoly
    \item vyrobit grafy
    \item vložit grafy do šablony
    \item vyrobit tabulky
    \item vložit tabulky do šablony
    \item A JE TO!
  \end{enumerate}
\end{frame}

\begin{frame}{A to je vše, přátelé!}
  \begin{center}
    \includegraphics[width=0.8\textwidth]{looney_tunes}
  \end{center}
\end{frame}

\end{document}
