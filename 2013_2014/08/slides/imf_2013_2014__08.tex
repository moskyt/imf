/ kodovani
/ ukol: signaly z detektoru a obrazky
/ SVG - vyklad, rucni generovani, samotne generovani obrazku
/ generovat soustavu html souboru
/ ???? sinatra (aspon uvod)

\documentclass{beamer}

\def\Tiny{\fontsize{6pt}{6pt}\selectfont}
\def\supertiny{\fontsize{4pt}{4pt}\selectfont}

\mode<presentation>
{
  \usetheme{Warsaw}
  % \setbeamercovered{transparent}
  \usecolortheme{crane}
}

\usepackage{graphicx, ifthen, listings, fancyvrb}

\usepackage[czech]{babel}
% \usefonttheme{professionalfonts}
\usepackage{times}
\usepackage{amsmath}
\usepackage[utf8]{inputenc}
\usepackage{wrapfig}

\usepackage[T1]{fontenc}

\lstset{ basicstyle=\tiny, stringstyle=\ttfamily, showstringspaces=false }

\everymath{\displaystyle}

\setbeamerfont{frametitle}{size=\large}
\setbeamerfont{subsection in toc}{size=\scriptsize}

\makeatletter\newenvironment{blackbox}{%
   \begin{lrbox}{\@tempboxa}\begin{minipage}{0.95\columnwidth}}{\end{minipage}\end{lrbox}%
   \colorbox{black}{\usebox{\@tempboxa}}
}\makeatother

\title[IMF (8)]{Informatika pro moderní fyziky (8)\\ CSS - stylování dokumentů, jednoduché webové aplikace}

\author[Franti\v{s}ek HAVL\r{U}J, ORF ÚJV Řež]{Franti\v{s}ek HAVL\r{U}J\\{\scriptsize \emph{e-mail: haf@ujv.cz}}}

\date{akademický rok 2013/2014\\3. prosince 2013}

\institute[ORF ÚJV Řež]
{ÚJV Řež\\oddělení Reaktorové fyziky a podpory palivového cyklu}

\AtBeginSection[]
{
\begin{frame}<beamer>
\frametitle{Obsah}
\tableofcontents[currentsection,hideothersubsections]
\end{frame}
}

\begin{document}

\begin{frame}
  \titlepage
\end{frame}

\begin{frame}
  \tableofcontents
\end{frame}

\section{Co jsme se naučili minule}

\begin{frame}{}
  \begin{itemize}
    \item HTML pro běžné použití
    \item procvičení ERb
    \item základy použití JS
  \end{itemize}
\end{frame}

\section{Něco o kódování}

\begin{frame}{Kódování}
  \begin{itemize}
    \item znaky v počítači: 1 znak = 1 byte = 256 možností
    \item cca polovina je ``normální text'', zbytek jsou tak trochu speciální znaky
    \item jsou tam ale jen `západní' znaky, na středoevropské se nedostalo
    \item co teprv azbuky, japonština, čínština, ...
  \end{itemize}
\end{frame}

\begin{frame}{Kódování}
  \begin{itemize}
    \item varianta 1: nahrazovat druhou polovinu znaků tím, co zrovna potřebuju (ěščř...)
    \item výhoda: nezabírá místo, pořád platí znak=byte, jednoduché řešení
    \item nevýhoda: pro různé jazyky různá kódování
    \item další nevýhoda: na leckterý jazyk (neevropský) to naprosto nestačí
  \end{itemize}
\end{frame}

\begin{frame}{Kódování}
  \begin{itemize}
    \item varianta 2: přejít na reprezentaci jednoho znaku více byty
    \item výhoda: podstatně se rozšíří počet znaků, takže není nutné `přepínat'
    \item nevýhoda: zvětšuje velikost textu
    \item další nevýhoda: míra rozsypanosti čaje při špatné interpretaci se zvyšuje (je možné způsobit i nečitelnost `obyčejných' znaků)
  \end{itemize}
\end{frame}

\begin{frame}{Kódování - obecné problémy}
  \begin{itemize}
    \item soubory neobsahují informaci o tom, v jakém kódování jsou, takže je nutné dodávat metadata
    \item manipulace s kódováním je otravná, obtížná a snadno způsobí problémy
    \item je nutné synchronizovat/řešit kódování na všech vrstvách aplikace (soubory / databáze / databázový klient / aplikace / server / klient)
    \item ... pokud existuje 100 různých variant, řeší se to obvykle tím, že se vymyslí nějaká 101. ...
  \end{itemize}
\end{frame}

\begin{frame}{Kódování}
  \begin{itemize}
    \item naštěstí to (minimálně v euroamerickém prostředí) už dneska nemusíme řešit, existuje jednotný standard -- Unicode (UTF-8)
    \item starší systémy občas poskytují data v jiných kódováních, je dobré vědět, že pro češtinu se můžete setkat s \texttt{Windows/CP1250} a \texttt{ISO-9981-2}
    \item pokud budete chtít pracovat s exotickými jazyky (Afrika, Oceánie), budete asi muset využít UTF-16/32 (naneštěstí opět více variant)
    \item z historických důvodů je nejednotnost také pro japonská kódování
    \item to hlavní: používejte UTF-8. Tečka. Nezapomeňte: v editoru, v hlavičce HTML, v hlavičce LaTeXu, všude...
  \end{itemize}
\end{frame}

\section{Stylování dokumentů}

\begin{frame}{CSS}
  \begin{itemize}
    \item jazyk pro popis vzhledu HTML dokumentu
    \item v nejjednodušším přístupu definuje styly pro jednotlivé typy elementů
    \item dále umožňuje definovat tzv. třídy (skupiny elementů, pomocí atributu \texttt{class} v HTML) a také styly pro konkrétní elementy (id)
    \item složitější selekory -- vnoření, souslednost atd.
  \end{itemize}
\end{frame}

\begin{frame}[fragile]{CSS - jednoduchý příklad}
  \begin{itemize}
    \item základní selektory: \emph{element}, \emph{.třída} a \emph{\#id}
    \item základní syntaxe \emph{vlastnost: hodnota;}
  \end{itemize}
  \scriptsize
  \begin{verbatim}
    a {
      color: blue;
      text-decoration: none;
      font-weight: bold;
    }
    #core_map { float:left; }
  \end{verbatim}
\end{frame}

\begin{frame}{CSS - kam s ním}
  \begin{itemize}
    \item přímo k tagu (\texttt{<div style=``display:none''>}) -- možná dobré na rychlé ladění/patlání, ale vesměs vždy špatně; jedinou výjimkou je \texttt{display:none} pro elementy, které mají být vidět až později, tam to jinak nejde
    \item v hlavičce (\texttt{head}) HTML dokumentu: \texttt{<style type=``text/css''>...</style>}
    \item v externím souboru (obvykle jediné správné řešení) -- pomocí tagu \texttt{link} v hlavičce
    \item my vystačíme se \texttt{style} tagem v hlavičcce
  \end{itemize}
\end{frame}

\begin{frame}{Rychlé procvičení}
  \begin{itemize}
    \item vezměte si svoje krásné HTML z minula a doplňte do něj trochu toho stylování
    \item minimum: změnit font (\texttt{font-family}), nastavit rozumné velikosti písma a barvy
    \item zrušit podtrhávání odkazů (\texttt{text-decoration})
    \item bonus: jiná barva odkazů na přepínání tabulka/graf
  \end{itemize}
\end{frame}

\section{Tvorba obrázků}

\begin{frame}{Zadání dnešní úlohy}
  \begin{itemize}
    \item každý den data z 1-9 detektorů
    \item detektor má svoji polohu v AZ (VR-1 Vrabec, 8x8 čtvercových pozic)
    \item je potřeba hezky zobrazit na každý den mapu AZ a grafy signálů z detektorů
    \item viz \texttt{html/document.html}
  \end{itemize}
\end{frame}

\begin{frame}{}
  \begin{itemize}
    \item
  \end{itemize}
\end{frame}

\begin{frame}{A to je vše, přátelé!}
  \begin{center}
    \includegraphics[width=0.8\textwidth]{looney_tunes}
  \end{center}
\end{frame}

\end{document}
  %
  %
  % kodovani
  % stylovani - vytunit tu vec z minula (? nebude to automatizace a tak dal, zejo)
  % serverside - prezentace zaznamu z reaktoru ...
  % ?generovani obrazku
  %
  %
  % zaznamy z detektoru, polohy, kresleni obrazku s polohou, mazce!
  %
  %



% \section{Server-side interaktivita}
%
% \begin{frame}{Principy webových aplikací}
%   \begin{itemize}
%     \item poskytování HTML (a dalšího) obsahu pomocí protokolu HTTP
%     \item webový prohlížeč pošle dotaz (request), který se skládá z adresy (URL) a případných parametrů
%     \item interakce je v podstatě bezstavová = na stejný dotaz dostanu stejnou odpověď
%     \item stav lze udržovat v tzv. session (rozumné např. pro autentizaci), ale měl by se používat jen minimálně
%   \end{itemize}
% \end{frame}
%
% \begin{frame}{Jak se píše webová aplikace}
%   \begin{itemize}
%     \item MVC - model/view/controller
%     \item model obsahuje tzv. business logic -- všechnu chytrost s daty, bez ohledu na to, jak se k nim přistupuje
%     \item controller zpracovává interakci uživatele s aplikací -- analyzuje request a rozhodne, co se má zobrazit
%     \item view je například ERb šablona
%   \end{itemize}
% \end{frame}
