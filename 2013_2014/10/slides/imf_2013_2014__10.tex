\documentclass{beamer}

\def\Tiny{\fontsize{6pt}{6pt}\selectfont}
\def\supertiny{\fontsize{4pt}{4pt}\selectfont}

\mode<presentation>
{
  \usetheme{Warsaw}
  % \setbeamercovered{transparent}
  \usecolortheme{crane}
}

\usepackage{graphicx, ifthen, listings, fancyvrb}

\usepackage[czech]{babel}
% \usefonttheme{professionalfonts}
\usepackage{times}
\usepackage{amsmath}
\usepackage[utf8]{inputenc}
\usepackage{wrapfig}

\usepackage[T1]{fontenc}

\lstset{ basicstyle=\tiny, stringstyle=\ttfamily, showstringspaces=false }

\everymath{\displaystyle}

\setbeamerfont{frametitle}{size=\large}
\setbeamerfont{subsection in toc}{size=\scriptsize}

\makeatletter\newenvironment{blackbox}{%
   \begin{lrbox}{\@tempboxa}\begin{minipage}{0.95\columnwidth}}{\end{minipage}\end{lrbox}%
   \colorbox{black}{\usebox{\@tempboxa}}
}\makeatother

\title[IMF (10)]{Informatika pro moderní fyziky (9)\\ složitější interaktivní dokument, správa zdrojového kódu}

\author[Franti\v{s}ek HAVL\r{U}J, ORF ÚJV Řež]{Franti\v{s}ek HAVL\r{U}J\\{\scriptsize \emph{e-mail: haf@ujv.cz}}}

\date{akademický rok 2013/2014\\10. prosince 2013}

\institute[ORF ÚJV Řež]
{ÚJV Řež\\oddělení Reaktorové fyziky a podpory palivového cyklu}

\AtBeginSection[]
{
\begin{frame}<beamer>
\frametitle{Obsah}
\tableofcontents[currentsection,hideothersubsections]
\end{frame}
}

\begin{document}

\begin{frame}
  \titlepage
\end{frame}

\begin{frame}
  \tableofcontents
\end{frame}

\section{Co jsme se naučili minule}

\begin{frame}{}
  \begin{itemize}
    \item procvičení zpracování dat
    \item stylování dokumentů s CSS
    \item vektorové obrázky na webu -- SVG
    \item procvičení zpracování dat
  \end{itemize}
\end{frame}

\section{Vzájemně provázaná data}

\begin{frame}{Zadání}
  \begin{itemize}
    \item zpracovat data o docházce do práce z ledna 2013
    \item soubor lide.csv obsahuje čísla průkazů a jména
    \item soubor dochazka.csv obsahuje datum, číslo průkazu, čas příchodu, čas odchodu
    \item kolik hodin byl kdo v lednu v práci?
    \item které dny kdo chyběl?
  \end{itemize}
\end{frame}

\begin{frame}{Inspirace}
  \begin{itemize}
    \item mám tady vztah číslo-člověk a pak záznamy číslo+časy
    \item mapa číslo-člověk je typický příklad použití hashe!
    \item pro agregaci záznamů ze dnů k lidem je také chytré použít hash (číslo průkazu jako klíč)
    \item pomněte paradigma \texttt{hash[key] ||= []}
  \end{itemize}
\end{frame}

\section{Navážeme na předminulou hodinu}

\begin{frame}{Zadání -- připomenutí}
  \begin{itemize}
    \item každý den data z 1-9 detektorů (\texttt{data/*.csv})
    \item detektor má svoji polohu v AZ (VR-1 Vrabec, 8x8 čtvercových pozic) -- včetně data je uvedena na prvním řádku CSV souboru
    \item je potřeba hezky zobrazit na každý den mapu AZ a grafy signálů z detektorů
    \item viz \texttt{html/document.html}
  \end{itemize}
\end{frame}

\begin{frame}{Co už máme}
  \begin{itemize}
    \item grafy pro jednotlivé detektory ve formátu PNG
    \item mapy zóny, ale zatím neklikací (SVG)
    \item umíme načíst data o jednotlivých detektorech -- víme, které pozice jsou v jednotlivé dny obsazeny
  \end{itemize}
\end{frame}

\begin{frame}{Co nám chybí}
  \begin{itemize}
    \item mít schované i názvy souborů pro jednotlivé detektory (hash! hash!)
    \item vyrobit si HTML dokument, který by zobrazoval jednotlivé mapy
    \item doplnit interaktivitu do SVG obrázků
    \item zobrazovat po kliknutí do mapy správný graf
  \end{itemize}
\end{frame}

\begin{frame}{Krok číslo jedna}
  \begin{itemize}
    \item z datových souborů si vyrobit takovou datovou strukturu, která mi bude říkat, pro který den a na které poloze je který soubor
    \item tj. radím například takovýto hash: {datum => {poloha => soubor}}
  \end{itemize}
\end{frame}

\begin{frame}{Další JS chytrosti}
  \begin{itemize}
    \item v jQuery už známe \texttt{\$(`\#id')}
    \item ale ve skutečnosti jde použít jakýkoli CSS selektor, takže třeba \texttt{\$(`p')}
    \item pokročilý CSS selektor -- vnoření: \texttt{\#my\_list img} vybere všechny obrázky (\texttt{img}) které jsou uvnitř elementu s id \texttt{my\_list}
    \item ... použiju v situacích, kdy chci schovat nějakou množinu elementů a pak jeden z nich zobrazit (tj. když mám hromadu obrázků a chci, aby byl vidět jen jeden)
  \end{itemize}
\end{frame}

\begin{frame}{Další CSS chytrosti}
  \begin{itemize}
    \item normálně se jednotlivé elementy řadí pod sebe
    \item můžu místo toho použít tzv. floating, kdy se začnou elementy řadit nalevo nebo napravo
    \item efekt znáte např. z webových galerií, kde mi fotky vyplní celou šířku okna a jdou po řádcích
    \item \texttt{float:left}
    \item barvu pozadí nastavím např. \texttt{background-color:red} nebo \texttt{background-color:\#ddddff}
  \end{itemize}
\end{frame}


\begin{frame}{Další SVG chytrosti}
  \begin{itemize}
    \item kromě \texttt{rect} se bude hodit také \texttt{text}
    \item jako text se zobrazí obsah příslušného elementu
    \item opět použiju atributy \texttt{x}, \texttt{y} (levý dolní roh) a můžu přihodit \texttt{text-anchor=``middle''}, aby to byl dolní prostředek
    \item pozor, text mi překryje čtvereček, takže budu muset zopakovat onclick! (musí být na textu i na čtverečku)
  \end{itemize}
\end{frame}

\section{Správa zdrojového kódu}

\begin{frame}{Jak mít uložený zdrojový kód}
  \begin{itemize}
    \item to, že mám na disku hromadu souborů a `jen tak' je edituju, přináší spoustu problémů
    \item
    \item
    \item
  \end{itemize}
\end{frame}

\begin{frame}{Práce v týmu}
  \begin{itemize}
    \item pokud na projektu pracuje víc lidí, stává se z problému noční můra
    \item jak dát dohromady
    \item jak zajistit, že všichni používají stejnou verzi?
    \item pro dokumenty už jsou různá více či méně dobrá řešení (slučování změn v MS Office, online nástroje jako Google Documents)
  \end{itemize}
\end{frame}

\begin{frame}{}
  \begin{itemize}
    \item
    \item
    \item
    \item
  \end{itemize}
\end{frame}

\begin{frame}{}
  \begin{itemize}
    \item
    \item
    \item
    \item
  \end{itemize}
\end{frame}

\begin{frame}{}
  \begin{itemize}
    \item
    \item
    \item
    \item
  \end{itemize}
\end{frame}


\begin{frame}{A to je vše, přátelé!}
  \begin{center}
    \includegraphics[width=0.8\textwidth]{looney_tunes}
  \end{center}
\end{frame}

\end{document}
