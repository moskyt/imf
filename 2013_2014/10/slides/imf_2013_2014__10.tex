\documentclass{beamer}

\def\Tiny{\fontsize{6pt}{6pt}\selectfont}
\def\supertiny{\fontsize{4pt}{4pt}\selectfont}

\mode<presentation>
{
  \usetheme{Warsaw}
  % \setbeamercovered{transparent}
  \usecolortheme{crane}
}

\usepackage{graphicx, ifthen, listings, fancyvrb}

\usepackage[czech]{babel}
% \usefonttheme{professionalfonts}
\usepackage{times}
\usepackage{amsmath}
\usepackage[utf8]{inputenc}
\usepackage{wrapfig}

\usepackage[T1]{fontenc}

\lstset{ basicstyle=\tiny, stringstyle=\ttfamily, showstringspaces=false }

\everymath{\displaystyle}

\setbeamerfont{frametitle}{size=\large}
\setbeamerfont{subsection in toc}{size=\scriptsize}

\makeatletter\newenvironment{blackbox}{%
   \begin{lrbox}{\@tempboxa}\begin{minipage}{0.95\columnwidth}}{\end{minipage}\end{lrbox}%
   \colorbox{black}{\usebox{\@tempboxa}}
}\makeatother

\title[IMF (10)]{Informatika pro moderní fyziky (10)\\ složitější interaktivní dokument, správa zdrojového kódu}

\author[Franti\v{s}ek HAVL\r{U}J, ORF ÚJV Řež]{Franti\v{s}ek HAVL\r{U}J\\{\scriptsize \emph{e-mail: haf@ujv.cz}}}

\date{akademický rok 2013/2014\\17. prosince 2013}

\institute[ORF ÚJV Řež]
{ÚJV Řež\\oddělení Reaktorové fyziky a podpory palivového cyklu}

\AtBeginSection[]
{
\begin{frame}<beamer>
\frametitle{Obsah}
\tableofcontents[currentsection,hideothersubsections]
\end{frame}
}

\begin{document}

\begin{frame}
  \titlepage
\end{frame}

\begin{frame}
  \tableofcontents
\end{frame}

\section{Co jsme se naučili minule}

\begin{frame}{}
  \begin{itemize}
    \item další procvičení zpracování dat - data s vazbami
    \item CSS selektory
    \item získávání informací z webu
    \item LaTex a ERb -- opakování
  \end{itemize}
\end{frame}

\section{Navážeme na předminulou hodinu}

\begin{frame}{Zadání -- připomenutí}
  \begin{itemize}
    \item každý den data z 1-9 detektorů (\texttt{data/*.csv})
    \item detektor má svoji polohu v AZ (VR-1 Vrabec, 8x8 čtvercových pozic) -- včetně data je uvedena na prvním řádku CSV souboru
    \item je potřeba hezky zobrazit na každý den mapu AZ a grafy signálů z detektorů
    \item viz \texttt{html/document.html}
  \end{itemize}
\end{frame}

\begin{frame}{Co už máme}
  \begin{itemize}
    \item grafy pro jednotlivé detektory ve formátu PNG
    \item mapy zóny, ale zatím neklikací (SVG)
    \item umíme načíst data o jednotlivých detektorech -- víme, které pozice jsou v jednotlivé dny obsazeny
  \end{itemize}
\end{frame}

\begin{frame}{Co nám chybí}
  \begin{itemize}
    \item mít schované i názvy souborů pro jednotlivé detektory (hash! hash!)
    \item vyrobit si HTML dokument, který by zobrazoval jednotlivé mapy
    \item doplnit interaktivitu do SVG obrázků
    \item zobrazovat po kliknutí do mapy správný graf
  \end{itemize}
\end{frame}

\begin{frame}{Krok číslo jedna}
  \begin{itemize}
    \item z datových souborů si vyrobit takovou datovou strukturu, která mi bude říkat, pro který den a na které poloze je který soubor
    \item tj. radím například takovýto hash: {datum => {poloha => soubor}}
  \end{itemize}
\end{frame}

\begin{frame}{Další kroky}
  \begin{itemize}
    \item dopsat do mapy detektorů text - polohu
    \item vyrobit zatím verzi, kde si budu moci prohlížet jednotlivé mapy (zatím bez grafů)
    \item zobrazit grafy detektorů - zatím všechny
    \item dokončit -- zobrazovat grafy
  \end{itemize}
\end{frame}

\begin{frame}{Další JS chytrosti}
  \begin{itemize}
    \item v jQuery už známe \texttt{\$(`\#id')}
    \item ale ve skutečnosti jde použít jakýkoli CSS selektor, takže třeba \texttt{\$(`p')}
    \item pokročilý CSS selektor -- vnoření: \texttt{\#my\_list img} vybere všechny obrázky (\texttt{img}) které jsou uvnitř elementu s id \texttt{my\_list}
    \item ... použiju v situacích, kdy chci schovat nějakou množinu elementů a pak jeden z nich zobrazit (tj. když mám hromadu obrázků a chci, aby byl vidět jen jeden)
  \end{itemize}
\end{frame}

\begin{frame}{Další CSS chytrosti}
  \begin{itemize}
    \item normálně se jednotlivé elementy řadí pod sebe
    \item můžu místo toho použít tzv. floating, kdy se začnou elementy řadit nalevo nebo napravo
    \item efekt znáte např. z webových galerií, kde mi fotky vyplní celou šířku okna a jdou po řádcích
    \item \texttt{float:left}
    \item barvu pozadí nastavím např. \texttt{background-color:red} nebo \texttt{background-color:\#ddddff}
  \end{itemize}
\end{frame}


\begin{frame}{Další SVG chytrosti}
  \begin{itemize}
    \item kromě \texttt{rect} se bude hodit také \texttt{text}
    \item jako text se zobrazí obsah příslušného elementu
    \item opět použiju atributy \texttt{x}, \texttt{y} (levý dolní roh) a můžu přihodit \texttt{text-anchor=``middle''}, aby to byl dolní prostředek
    \item pozor, text mi překryje čtvereček, takže budu muset zopakovat onclick! (musí být na textu i na čtverečku)
  \end{itemize}
\end{frame}

\section{Správa zdrojového kódu}

\begin{frame}{Jak mít uložený zdrojový kód}
  \begin{itemize}
    \item to, že mám na disku hromadu souborů a `jen tak' je edituju, přináší spoustu problémů
    \item pokud udělám chybu/něco se rozbije, nemám se jak vrátit, nemám přehled o průběhu změn
    \item pokud chci mít víc verzí najednou (stabilní / vývojová), nemám jak zajistit konzistenci
    \item není jak se vracet ke starším verzím, problémy de facto nelze reproduktovat
  \end{itemize}
\end{frame}

\begin{frame}{Práce v týmu}
  \begin{itemize}
    \item pokud na projektu pracuje víc lidí, stává se z problému noční můra
    \item jak dát dohromady
    \item jak zajistit, že všichni používají stejnou verzi?
    \item pro dokumenty už jsou různá více či méně dobrá řešení (slučování změn v MS Office, online nástroje jako Google Documents)
  \end{itemize}
\end{frame}

\begin{frame}{Jaké jsou možnosti}
  \begin{itemize}
    \item nabízí se schovávat si každý týden verzi a pokusit se v tom udržet pořádek
    \item dbát na správná jména složek/souborů
    \item za žádnou cenu nemodifikovat verzovanou složku (read-only...)
    \item vést záznam o změnách, všechno podrobně dokumentovat
  \end{itemize}
\end{frame}

\begin{frame}{Jak se to dělá správně}
  \begin{itemize}
    \item existují takzvané version control systems -- VCS, které se o řešení tohoto problému postarají za nás
    \item základní princip: neukládám aktuální verze souborů, ale pouze \emph{změny}
    \item k čemu to je -- změny je možné mezi sebou \textbf{sčítat}, což je (po lehkém zamyšlení) naprosto geniální a řeší to všechno
    \item pokud mám paralelní změny (dva různí lidé něco změní), není to obvykle vůbec žádný problém, pokud
    \item můžu rekonstruovat libovolný bod v historii
  \end{itemize}
\end{frame}

\begin{frame}{Řešení konfliktů -- merging}
  \begin{itemize}
    \item pokud dva uživatelé udělají změnu zároveň, co se stane?
    \item když konflikt nenastal na stejném řádku stejného souboru, tak je to v pořádku
    \item v opačném případě to prostě musí vyřešit uživatel ručně (ale není to častá situace)
    \item spojení změn se ale vždycky musí dělat lokálně (na straně uživatele), aby se zamezilo sémantickému konfliktu
  \end{itemize}
\end{frame}

\begin{frame}{Paralelní větve -- branches}
  \begin{itemize}
    \item často se hodí mít víc verzí najednou
    \item jednak různé stupně experimentálnosti (stabilní, testovací, vývojová)
    \item jednak pokud chci přidat větší feature, tak to nechci míchat s hlavní větví vývoje
    \item VCS to většinou umí řešit -- strom vývoje nemusí být lineární, ale může se větvit
  \end{itemize}
\end{frame}

\begin{frame}{Přehled VCS}
  \begin{itemize}
    \item CVS -- první standardní VCS, dnes již zcela překonaný
    \item SVN -- moderní centralizovaný VCS, donedávna nejpoužívanější řešení
    \item Git -- dnešní de facto standard, pokročilý distribuovaný VCS s geniální podporou branches
    \item Mercurial, Bazaar -- méně obvyklé alternativy Gitu
  \end{itemize}
\end{frame}

\begin{frame}{Centralizovaný vs. distribuovaný VCS}
  \begin{itemize}
    \item centralizovaný: mám jeden centrální repozitář, do kterého ukládají změny všichni uživatelé
    \item distribuovaný: každý uživatel má celý repozitář u sebe
    \item výhody -- lokální commity (bez spojení se serverem!), lokální branche, bezpečnost (vysoká redundance)
    \item nevýhody -- někomu to dělá problém pochopit a přijmout
  \end{itemize}
\end{frame}

\begin{frame}{Kdy je dobré používat VCS}
  \begin{itemize}
    \item vlastně vždycky, pokud pracuju s plaintextovými soubory (tj. zejména zdrojové kódy, skripty)
    \item využití pro binární data (což se týká de facto i wordovského dokumentu...) je velice omezené
    \item -- další důvod používat LaTeX! tam je samozřejmě všechno v pořádku
    \item výhody verzování (oproti tomu to ``jen někde mít'') jsou nedozírné
    \item za sebe můžu říct, že asi úplně všechnu svoji práci mám v nějakém repo
  \end{itemize}
\end{frame}

\begin{frame}{Ukázka a příklad veřejných repozitářů}
  \begin{itemize}
    \item \texttt{http://github.com}
    \item \texttt{http://github.com/jquery/jquery}
    \item vhodné zejména pro open source projekty, ale umožňuje i soukromé repo (za poplatek)
    \item velmi hezké GUI (přehledy změn, grafy větvení, issue tracker, komunikace mezi vývojáři)
    \item samozřejmě není problém si nainstalovat repozitář (např. \texttt{gitolite}) někde u sebe na serveru
  \end{itemize}
\end{frame}



\begin{frame}{A to je vše, přátelé!}
  \begin{center}
    \includegraphics[width=0.8\textwidth]{looney_tunes}
  \end{center}
\end{frame}

\end{document}
