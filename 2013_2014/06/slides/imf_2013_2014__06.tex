\documentclass{beamer}

\def\Tiny{\fontsize{6pt}{6pt}\selectfont}
\def\supertiny{\fontsize{4pt}{4pt}\selectfont}

\mode<presentation>
{
  \usetheme{Warsaw}
  % \setbeamercovered{transparent}
  \usecolortheme{crane}
}

\usepackage{graphicx, ifthen, listings, fancyvrb}

\usepackage[czech]{babel}
% \usefonttheme{professionalfonts}
\usepackage{times}
\usepackage{amsmath}
\usepackage[utf8]{inputenc}
\usepackage{wrapfig}

\usepackage[T1]{fontenc}

\lstset{ basicstyle=\tiny, stringstyle=\ttfamily, showstringspaces=false }

\everymath{\displaystyle}

\setbeamerfont{frametitle}{size=\large}
\setbeamerfont{subsection in toc}{size=\scriptsize}

\makeatletter\newenvironment{blackbox}{%
   \begin{lrbox}{\@tempboxa}\begin{minipage}{0.95\columnwidth}}{\end{minipage}\end{lrbox}%
   \colorbox{black}{\usebox{\@tempboxa}}
}\makeatother

\title[IMF (6)]{Informatika pro moderní fyziky (6)\\ Chytré šablony a interaktivní dokumenty}

\author[Franti\v{s}ek HAVL\r{U}J, ORF ÚJV Řež]{Franti\v{s}ek HAVL\r{U}J\\{\scriptsize \emph{e-mail: haf@ujv.cz}}}

\date{akademický rok 2013/2014\\19. listopadu 2013}

\institute[ORF ÚJV Řež]
{ÚJV Řež\\oddělení Reaktorové fyziky a podpory palivového cyklu}

\AtBeginSection[]
{
\begin{frame}<beamer>
\frametitle{Obsah}
\tableofcontents[currentsection,hideothersubsections]
\end{frame}
}

\begin{document}

\begin{frame}
  \titlepage
\end{frame}

\begin{frame}
  \tableofcontents
\end{frame}

\section{Co jsme se naučili minule}

\begin{frame}{}
  \begin{itemize}
    \item používání klávesových zkratek
    \item generování většího množství grafů, práce se všemi soubory v adresáři
    \item tvorba tabulek v LaTeXu, jejich vložení do dokumentu
  \end{itemize}
\end{frame}

\begin{frame}[fragile]{Řetězec vs. proměnná}
  V čem se liší tyto příkazy a co znamenají?
  \begin{verbatim}
puts a
puts "a"
puts "#{a}"
  \end{verbatim}
\end{frame}

\section{Tvorba dokumentu -- dokončení}

\begin{frame}{Co už máme}
  \begin{itemize}
    \item soubory \texttt{data\_*.csv} - data ve třech sloupcích (datum, koncentrace kyseliny borité v chladivu, axiální ofset)
    \item vyrobené tabulky (jistě si všichni zpracovali doma) (\texttt{ao\_c??.tex}, \texttt{bc\_c??.tex})
    \item hezké grafy (\texttt{ao\_c??.eps}, \texttt{bc\_c??.eps})
  \end{itemize}
\end{frame}

\begin{frame}{Co nám ještě chybí}
  \begin{itemize}
    \item slepit všechno do jednoho dokumentu -- s hezkou strukturou, s vloženými
    \item de facto potřebujeme vyrobit celý LaTeX dokument automaticky (kromě hlavičky) vyrobit automaticky
    \item jaké máme možnosti?
  \end{itemize}
\end{frame}

\section{Na šablony chytře}

\begin{frame}{Úskalí šablon}
  \begin{itemize}
    \item snadno umíme nahradit jeden řetězec druhým
    \item trochu méně pohodlné pro větší bloky textu
    \item navíc by se hodila nějaká logika (cyklus) přímo v šabloně
    \item naštěstí jsou na to postupy
  \end{itemize}
\end{frame}

\begin{frame}{ERb (Embedded Ruby)}
  \begin{itemize}
    \item lepší šablona - ``aktivní text''
    \item používá se například ve webových aplikacích
    \item hodí se ale i na generování latexových dokumentů, resp. všude, kde nám nesejde na whitespace
    \item poměrně jednoduchá syntax, zvládne skoro všechno (viz předmět MAA3)
  \end{itemize}
\end{frame}


\begin{frame}[fragile]{Základní syntaxe ERb (1)}
  \begin{block}{ }
    Jakýkoli Ruby příkaz, přiřazení, výpočet ...
    \scriptsize
    \begin{verbatim}
      <% a = b + 5 %>
      <% list = ary * ", " %>
    \end{verbatim}
  \end{block}
\end{frame}

\begin{frame}[fragile]{Základní syntaxe ERb (2)}
  \begin{block}{ }
    Pokud chci něco vložit, stačí přidat rovnítko
    \scriptsize
    \begin{verbatim}
      <%= a %>
      <%= ary[1] %>
      <%= b + 5 %>
    \end{verbatim}
  \end{block}
\end{frame}

\begin{frame}[fragile]{Základní syntaxe ERb (3)}
  \begin{block}{ }
    Radost je možnost použít bloky a tedy i iterátory apod. v propojení s vkládaným textem:
    \scriptsize
    \begin{verbatim}
      <% (1..5).each do |i| %>
      Number <%= i %>
      <% end %>
      <% ary.each do |x| %>
      Array contains <%= x %>
      <% end %>
    \end{verbatim}
  \end{block}
\end{frame}

\begin{frame}{ERb -- shrnutí}
  \begin{itemize}
    \item dobrý sluha, ale špatný pán
    \item můžu s tím vyrobit hromadu užitečných věcí na malém prostoru
    \item daň je velké riziko zamotaného kódu a nízké přehlednosti (struktura naprosto není patrná na první pohled, proto je namístě ji držet maximálně jednoduchou)
  \end{itemize}
\end{frame}

\begin{frame}{Důležité upozornění}
  \begin{itemize}
    \item oddělení modelu a view
    \item přestože lze provádět zpracování dat a výpočty přímo v ERb, je to nejvíc nejhorší nápad
    \item je chytré si všechno připravit v modelu (tj. v Ruby skriptu, kterým data chystáme)
    \item a kód ve view (tj. v ERb šabloně) omezit na naprosté minimum
  \end{itemize}
\end{frame}

\begin{frame}[fragile]{Jak ze šablony udělat výsledek}
  \scriptsize
  \begin{block}{Příklad překladu ERb}
    \scriptsize
    \begin{verbatim}
require 'erb_compiler'

erb(template, filename, {:x => 1, :y => 2})
    \end{verbatim}
  \end{block}
\end{frame}

\begin{frame}[fragile]{Příklad -- kreslení grafů z minula}
  \begin{block}{template.gp}
    \scriptsize
    \begin{verbatim}
set terminal png
set output "plot_<%=n%>.png"
plot "data_<%=n%>.csv"
    \end{verbatim}
  \end{block}
  \begin{block}{}
    \scriptsize
    \begin{verbatim}
(1..10).each do |i|
  erb("template.gp", "plot_#{i}.gp", {:n => i})
end
    \end{verbatim}
  \end{block}
\end{frame}

\begin{frame}[fragile]{Takže v latexu třeba}
\scriptsize
\begin{verbatim}
  \subsection{Koncentrace kyseliny borité}

  <% files.each do |f| %>

  \subsubsection{Kampaň <%= f.split('_').last %>}
  \begin{center}
  \includegraphics[width=0.8\textwidth]{<%= f %>_bc.eps}
  \end{center}
  <% end %>
\end{verbatim}
\end{frame}

\begin{frame}{A teď už to jenom dejte dohromady...}
  \begin{enumerate}
    \item připravit si základní kostru dokumentu v latexu
    \item převést na šablonu: mít seznam souborů, správně generovat kapitoly
    \item vyrobit grafy
    \item vložit grafy do šablony
    \item vyrobit tabulky
    \item vložit tabulky do šablony
    \item A JE TO!
  \end{enumerate}
\end{frame}

\section{HTML jako prezentační nástroj}

\begin{frame}{Výhody HTML+JS}
  \begin{itemize}
    \item PDF zpráva je fajn, ale umožňuje jen ``plochou'' strukturu
    \item často je potřeba prezentovat tak velké množství informací, že to nejde jenom nalepit za sebe
    \item hodilo by se něco klikacího
    \item příklad -- ASTRID (výstup z programu ANDREA)
  \end{itemize}
\end{frame}

\begin{frame}{HTML}
  \begin{itemize}
    \item
  \end{itemize}
\end{frame}

\begin{frame}{javaScript}
  \begin{itemize}
    \item trochu divný jazyk, historicky vzniknul jako ideový koncept a omylem se z toho stala definitivní verze
    \item potvora s nejasnou syntaxí a dost netradiční
    \item naštěstí na většinu základního využití
  \end{itemize}
\end{frame}

\begin{frame}{}
  \begin{itemize}
    \item
  \end{itemize}
\end{frame}

\begin{frame}{}
  \begin{itemize}
    \item
  \end{itemize}
\end{frame}

\begin{frame}{A to je vše, přátelé!}
  \begin{center}
    \includegraphics[width=0.8\textwidth]{looney_tunes}
  \end{center}
\end{frame}

\end{document}

% 7 - dodelavka z minula (asi JS)
%   -
%
%
%! VCS, SVG, rubygems, bundler, import/export exl



% 1   gnuplot
% 2   ruby
% 3   z mcnp
% 4   rake, latex
%
%
% 5   erb, more latex, vcs intro
%   - jak funguje erb
%   - load, require, def
%   - tyce revisited
%   - vyrobit z latexu tabulku
%   - about vcs
%   - github -- clone imf repo
%

%
% 6
% 7
% 8
% 9
% 10
%
%
% html, zase generovani
%
% webove aplikace
%
% yaml a xml
%
% pustit vypocet na linuxu
%
% generovani svg
%
% version control
%
% ---
% git@win?
% konto na lenochodech?
