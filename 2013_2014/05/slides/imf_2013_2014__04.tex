\documentclass{beamer}

\def\Tiny{\fontsize{6pt}{6pt}\selectfont}
\def\supertiny{\fontsize{4pt}{4pt}\selectfont}

\mode<presentation>
{
  \usetheme{Warsaw}
  % \setbeamercovered{transparent}
  \usecolortheme{crane}
}


\usepackage{graphicx, ifthen, listings, fancyvrb}

\usepackage[czech]{babel}
% \usefonttheme{professionalfonts}
\usepackage{times}
\usepackage{amsmath}
\usepackage[utf8]{inputenc}
\usepackage{wrapfig}

\usepackage[T1]{fontenc}

\lstset{ basicstyle=\tiny, stringstyle=\ttfamily, showstringspaces=false }

\everymath{\displaystyle}

\setbeamerfont{frametitle}{size=\large}
\setbeamerfont{subsection in toc}{size=\scriptsize}

\makeatletter\newenvironment{blackbox}{%
   \begin{lrbox}{\@tempboxa}\begin{minipage}{0.95\columnwidth}}{\end{minipage}\end{lrbox}%
   \colorbox{black}{\usebox{\@tempboxa}}
}\makeatother

\title[IMF (5)]{Informatika pro moderní fyziky (5)\\ XXX}

\author[Franti\v{s}ek HAVL\r{U}J, ORF ÚJV Řež]{Franti\v{s}ek HAVL\r{U}J\\{\scriptsize \emph{e-mail: haf@ujv.cz}}}

\date{akademický rok 2013/2014\\12. listopadu 2013}

\institute[ORF ÚJV Řež]
{ÚJV Řež\\oddělení Reaktorové fyziky a podpory palivového cyklu}

\AtBeginSection[]
{
\begin{frame}<beamer>
\frametitle{Obsah}
\tableofcontents[currentsection,hideothersubsections]
\end{frame}
}

\begin{document}

\begin{frame}
  \titlepage
\end{frame}

\begin{frame}
  \tableofcontents
\end{frame}

\section{Co jsme se naučili minule}

\begin{frame}{}
  \begin{itemize}
    \item komplexnější řešení úlohy na zpracování dat, vykreslování sady grafů
    \item organizace jednoduchých skriptů - rake
    \item rychlý úvod do LaTeXu
  \end{itemize}
\end{frame}

\section{Výroba dokumentu v praxi}

\begin{frame}{Úkol na dnešek}
  \begin{itemize}
    \item pro jeden blok JE mám provozní data - v určitých dnech hodnotu koncentrace kyseliny borité a axiálního ofsetu - pro několik kampaní (blíže neurčený počet)
    \item chci vyrobit přehledové PDF, které bude hezky prezentovat grafy obou veličin pro každou kampaň a k tomu i tabulky
    \item data pro jednotlivé kampaně mám v CSV souborech, každý má tři sloupce (datum, cB, AO)
  \end{itemize}
\end{frame}

\begin{frame}{Rozbor}
  \begin{itemize}
    \item načíst tabulky a vykreslit grafy umíme
    \item převést tabulky v CSV na tabulky v LaTeXu se záhy naučíme
    \item vložit obrázek do latexu taky umíme
    \item předem neznámý počet souborů nás netrápí (\texttt{Dir["*.csv"]})
  \end{itemize}
\end{frame}

\section{Na šablony chytře}

\begin{frame}{Úskalí šablon}
  \begin{itemize}
    \item snadno umíme nahradit jeden řetězec druhým
    \item trochu méně pohodlné pro větší bloky textu
    \item navíc by se hodila nějaká logika (cyklus) přímo v šabloně
    \item naštěstí jsou na to postupy
  \end{itemize}
\end{frame}

\begin{frame}{ERb (XXXX Ruby)}
  \begin{itemize}
    \item lepší šablona - ``aktivní text''
    \item používá se například ve webových aplikacích
    \item hodí se ale i na generování latexových dokumentů, resp. všude, kde nám nesejde na whitespace
    \item poměrně jednoduchá syntax, zvládne skoro všechno (viz předmět MAA3)
  \end{itemize}
\end{frame}



\begin{frame}[fragile]{ }
  \scriptsize
  \begin{block}{ }
    \scriptsize
    \begin{verbatim}

    \end{verbatim}
  \end{block}
\end{frame}

Základní syntaxe ERb (1)
Jakýkoli Ruby příkaz, přiřazení, výpočet ...
<% a = b + 5 %>
<% list = ary * ", " %>

Základní syntaxe ERb (2)
Pokud chci něco vložit, stačí přidat rovnítko
<%= a %>
<%= ary[1] %>
<%= b + 5 %>

Základní syntaxe ERb (3)
Radost je možnost použít bloky a tedy i iterátory apod.:
<% (1..5).each do |i| %>
Number <%= i %>
<% end %>
<% ary.each do |x| %>
Array contains <%= x %>
<% end %>

Důležité upozornění

- oddělení modelu a view
- přestože lze provádět zpracování dat a výpočty přímo v ERb, je to nejvíc nejhorší nápad
- je chytré si všechno připravit v modelu (tj. v Ruby skriptu, kterým data chystáme)
- a kód ve view (tj. v ERb šabloně) omezit na naprosté minimum






\begin{frame}[fragile]{Jak ze šablony udělat výsledek}
  \scriptsize
  \begin{block}{Příklad překladu ERb}
    \scriptsize
    \begin{verbatim}
require 'erb'

erb(template, filename, {:x => 1, :y => 2})
    \end{verbatim}
  \end{block}
\end{frame}




\begin{frame}[fragile]{Příklad -- kreslení grafů z minula}
  \begin{block}{template.gp}
    \scriptsize
    \begin{verbatim}
set terminal png
set output "plot_<%=n%>.png"
plot "data_<%=n%>.csv"
    \end{verbatim}
  \end{block}
  \begin{block}{}
    \scriptsize
    \begin{verbatim}
(1..10).each do |i|
  erb("template.gp", "plot_#{i}.gp", {:n => i})
end
    \end{verbatim}
  \end{block}
\end{frame}


\begin{frame}{A to je vše, přátelé!}
  \begin{center}
    \includegraphics[width=0.8\textwidth]{looney_tunes}
  \end{center}
\end{frame}

\end{document}

1   gnuplot
2   ruby
3   z mcnp
4   rake, latex


5   erb, more latex, vcs intro
  - jak funguje erb
  - load, require, def
  - tyce revisited
  - vyrobit z latexu tabulku
  - about vcs
  - github -- clone imf repo


%
% 6
% 7
% 8
% 9
% 10
%
%
% html, zase generovani
%
% webove aplikace
%
% yaml a xml
%
% pustit vypocet na linuxu
%
% generovani svg
%
% version control
%
% ---
% git@win?
% konto na lenochodech?\documentclass{beamer}

\def\Tiny{\fontsize{6pt}{6pt}\selectfont}
\def\supertiny{\fontsize{4pt}{4pt}\selectfont}

\mode<presentation>
{
  \usetheme{Warsaw}
  % \setbeamercovered{transparent}
  \usecolortheme{crane}
}


\usepackage{graphicx, ifthen, listings, fancyvrb}

\usepackage[czech]{babel}
% \usefonttheme{professionalfonts}
\usepackage{times}
\usepackage{amsmath}
\usepackage[utf8]{inputenc}
\usepackage{wrapfig}

\usepackage[T1]{fontenc}

\lstset{ basicstyle=\tiny, stringstyle=\ttfamily, showstringspaces=false }

\everymath{\displaystyle}

\setbeamerfont{frametitle}{size=\large}
\setbeamerfont{subsection in toc}{size=\scriptsize}

\makeatletter\newenvironment{blackbox}{%
   \begin{lrbox}{\@tempboxa}\begin{minipage}{0.95\columnwidth}}{\end{minipage}\end{lrbox}%
   \colorbox{black}{\usebox{\@tempboxa}}
}\makeatother

\title[IMF (5)]{Informatika pro moderní fyziky (5)\\ XXX}

\author[Franti\v{s}ek HAVL\r{U}J, ORF ÚJV Řež]{Franti\v{s}ek HAVL\r{U}J\\{\scriptsize \emph{e-mail: haf@ujv.cz}}}

\date{akademický rok 2013/2014\\12. listopadu 2013}

\institute[ORF ÚJV Řež]
{ÚJV Řež\\oddělení Reaktorové fyziky a podpory palivového cyklu}

\AtBeginSection[]
{
\begin{frame}<beamer>
\frametitle{Obsah}
\tableofcontents[currentsection,hideothersubsections]
\end{frame}
}

\begin{document}

\begin{frame}
  \titlepage
\end{frame}

\begin{frame}
  \tableofcontents
\end{frame}

\section{Co jsme se naučili minule}

\begin{frame}{}
  \begin{itemize}
    \item komplexnější řešení úlohy na zpracování dat, vykreslování sady grafů
    \item organizace jednoduchých skriptů - rake
    \item rychlý úvod do LaTeXu
  \end{itemize}
\end{frame}

\section{Výroba dokumentu v praxi}

\begin{frame}{Úkol na dnešek}
  \begin{itemize}
    \item pro jeden blok JE mám provozní data - v určitých dnech hodnotu koncentrace kyseliny borité a axiálního ofsetu - pro několik kampaní (blíže neurčený počet)
    \item chci vyrobit přehledové PDF, které bude hezky prezentovat grafy obou veličin pro každou kampaň a k tomu i tabulky
    \item data pro jednotlivé kampaně mám v CSV souborech, každý má tři sloupce (datum, cB, AO)
  \end{itemize}
\end{frame}

\begin{frame}{Rozbor}
  \begin{itemize}
    \item načíst tabulky a vykreslit grafy umíme
    \item převést tabulky v CSV na tabulky v LaTeXu se záhy naučíme
    \item vložit obrázek do latexu taky umíme
    \item předem neznámý počet souborů nás netrápí (\texttt{Dir["*.csv"]})
  \end{itemize}
\end{frame}

\section{Na šablony chytře}

\begin{frame}{Úskalí šablon}
  \begin{itemize}
    \item snadno umíme nahradit jeden řetězec druhým
    \item trochu méně pohodlné pro větší bloky textu
    \item navíc by se hodila nějaká logika (cyklus) přímo v šabloně
    \item naštěstí jsou na to postupy
  \end{itemize}
\end{frame}

\begin{frame}{ERb (XXXX Ruby)}
  \begin{itemize}
    \item lepší šablona - ``aktivní text''
    \item používá se například ve webových aplikacích
    \item hodí se ale i na generování latexových dokumentů, resp. všude, kde nám nesejde na whitespace
    \item poměrně jednoduchá syntax, zvládne skoro všechno (viz předmět MAA3)
  \end{itemize}
\end{frame}





Základní syntaxe ERb (1)
Jakýkoli Ruby příkaz, přiřazení, výpočet ...
<% a = b + 5 %>
<% list = ary * ", " %>

Základní syntaxe ERb (2)
Pokud chci něco vložit, stačí přidat rovnítko
<%= a %>
<%= ary[1] %>
<%= b + 5 %>

Základní syntaxe ERb (3)
Radost je možnost použít bloky a tedy i iterátory apod.:
<% (1..5).each do |i| %>
Number <%= i %>
<% end %>
<% ary.each do |x| %>
Array contains <%= x %>
<% end %>

Důležité upozornění

- oddělení modelu a view
- přestože lze provádět zpracování dat a výpočty přímo v ERb, je to nejvíc nejhorší nápad
- je chytré si všechno připravit v modelu (tj. v Ruby skriptu, kterým data chystáme)
- a kód ve view (tj. v ERb šabloně) omezit na naprosté minimum






\begin{frame}[fragile]{Jak ze šablony udělat výsledek}
  \scriptsize
  \begin{block}{Příklad překladu ERb}
    \scriptsize
    \begin{verbatim}
require 'erb'

erb(template, filename, {:x => 1, :y => 2})
    \end{verbatim}
  \end{block}
\end{frame}




\begin{frame}[fragile]{Příklad -- kreslení grafů z minula}
  \begin{block}{template.gp}
    \scriptsize
    \begin{verbatim}
set terminal png
set output "plot_<%=n%>.png"
plot "data_<%=n%>.csv"
    \end{verbatim}
  \end{block}
  \begin{block}{}
    \scriptsize
    \begin{verbatim}
(1..10).each do |i|
  erb("template.gp", "plot_#{i}.gp", {:n => i})
end
    \end{verbatim}
  \end{block}
\end{frame}


\begin{frame}{A to je vše, přátelé!}
  \begin{center}
    \includegraphics[width=0.8\textwidth]{looney_tunes}
  \end{center}
\end{frame}

\end{document}

1   gnuplot
2   ruby
3   z mcnp
4   rake, latex


5   erb, more latex, vcs intro
  - jak funguje erb
  - load, require, def
  - tyce revisited
  - vyrobit z latexu tabulku
  - about vcs
  - github -- clone imf repo


%
% 6
% 7
% 8
% 9
% 10
%
%
% html, zase generovani
%
% webove aplikace
%
% yaml a xml
%
% pustit vypocet na linuxu
%
% generovani svg
%
% version control
%
% ---
% git@win?
% konto na lenochodech?\documentclass{beamer}

\def\Tiny{\fontsize{6pt}{6pt}\selectfont}
\def\supertiny{\fontsize{4pt}{4pt}\selectfont}

\mode<presentation>
{
  \usetheme{Warsaw}
  % \setbeamercovered{transparent}
  \usecolortheme{crane}
}


\usepackage{graphicx, ifthen, listings, fancyvrb}

\usepackage[czech]{babel}
% \usefonttheme{professionalfonts}
\usepackage{times}
\usepackage{amsmath}
\usepackage[utf8]{inputenc}
\usepackage{wrapfig}

\usepackage[T1]{fontenc}

\lstset{ basicstyle=\tiny, stringstyle=\ttfamily, showstringspaces=false }

\everymath{\displaystyle}

\setbeamerfont{frametitle}{size=\large}
\setbeamerfont{subsection in toc}{size=\scriptsize}

\makeatletter\newenvironment{blackbox}{%
   \begin{lrbox}{\@tempboxa}\begin{minipage}{0.95\columnwidth}}{\end{minipage}\end{lrbox}%
   \colorbox{black}{\usebox{\@tempboxa}}
}\makeatother

\title[IMF (5)]{Informatika pro moderní fyziky (5)\\ XXX}

\author[Franti\v{s}ek HAVL\r{U}J, ORF ÚJV Řež]{Franti\v{s}ek HAVL\r{U}J\\{\scriptsize \emph{e-mail: haf@ujv.cz}}}

\date{akademický rok 2013/2014\\12. listopadu 2013}

\institute[ORF ÚJV Řež]
{ÚJV Řež\\oddělení Reaktorové fyziky a podpory palivového cyklu}

\AtBeginSection[]
{
\begin{frame}<beamer>
\frametitle{Obsah}
\tableofcontents[currentsection,hideothersubsections]
\end{frame}
}

\begin{document}

\begin{frame}
  \titlepage
\end{frame}

\begin{frame}
  \tableofcontents
\end{frame}

\section{Co jsme se naučili minule}

\begin{frame}{}
  \begin{itemize}
    \item komplexnější řešení úlohy na zpracování dat, vykreslování sady grafů
    \item organizace jednoduchých skriptů - rake
    \item rychlý úvod do LaTeXu
  \end{itemize}
\end{frame}

\section{Výroba dokumentu v praxi}

\begin{frame}{Úkol na dnešek}
  \begin{itemize}
    \item pro jeden blok JE mám provozní data - v určitých dnech hodnotu koncentrace kyseliny borité a axiálního ofsetu - pro několik kampaní (blíže neurčený počet)
    \item chci vyrobit přehledové PDF, které bude hezky prezentovat grafy obou veličin pro každou kampaň a k tomu i tabulky
    \item data pro jednotlivé kampaně mám v CSV souborech, každý má tři sloupce (datum, cB, AO)
  \end{itemize}
\end{frame}

\begin{frame}{Rozbor}
  \begin{itemize}
    \item načíst tabulky a vykreslit grafy umíme
    \item převést tabulky v CSV na tabulky v LaTeXu se záhy naučíme
    \item vložit obrázek do latexu taky umíme
    \item předem neznámý počet souborů nás netrápí (\texttt{Dir["*.csv"]})
  \end{itemize}
\end{frame}

\section{Na šablony chytře}

\begin{frame}{Úskalí šablon}
  \begin{itemize}
    \item snadno umíme nahradit jeden řetězec druhým
    \item trochu méně pohodlné pro větší bloky textu
    \item navíc by se hodila nějaká logika (cyklus) přímo v šabloně
    \item naštěstí jsou na to postupy
  \end{itemize}
\end{frame}

\begin{frame}{ERb (XXXX Ruby)}
  \begin{itemize}
    \item lepší šablona - ``aktivní text''
    \item používá se například ve webových aplikacích
    \item hodí se ale i na generování latexových dokumentů, resp. všude, kde nám nesejde na whitespace
    \item poměrně jednoduchá syntax, zvládne skoro všechno (viz předmět MAA3)
  \end{itemize}
\end{frame}





Základní syntaxe ERb (1)
Jakýkoli Ruby příkaz, přiřazení, výpočet ...
<% a = b + 5 %>
<% list = ary * ", " %>

Základní syntaxe ERb (2)
Pokud chci něco vložit, stačí přidat rovnítko
<%= a %>
<%= ary[1] %>
<%= b + 5 %>

Základní syntaxe ERb (3)
Radost je možnost použít bloky a tedy i iterátory apod.:
<% (1..5).each do |i| %>
Number <%= i %>
<% end %>
<% ary.each do |x| %>
Array contains <%= x %>
<% end %>

Důležité upozornění

- oddělení modelu a view
- přestože lze provádět zpracování dat a výpočty přímo v ERb, je to nejvíc nejhorší nápad
- je chytré si všechno připravit v modelu (tj. v Ruby skriptu, kterým data chystáme)
- a kód ve view (tj. v ERb šabloně) omezit na naprosté minimum






\begin{frame}[fragile]{Jak ze šablony udělat výsledek}
  \scriptsize
  \begin{block}{Příklad překladu ERb}
    \scriptsize
    \begin{verbatim}
require 'erb'

erb(template, filename, {:x => 1, :y => 2})
    \end{verbatim}
  \end{block}
\end{frame}




\begin{frame}[fragile]{Příklad -- kreslení grafů z minula}
  \begin{block}{template.gp}
    \scriptsize
    \begin{verbatim}
set terminal png
set output "plot_<%=n%>.png"
plot "data_<%=n%>.csv"
    \end{verbatim}
  \end{block}
  \begin{block}{}
    \scriptsize
    \begin{verbatim}
(1..10).each do |i|
  erb("template.gp", "plot_#{i}.gp", {:n => i})
end
    \end{verbatim}
  \end{block}
\end{frame}


\begin{frame}{A to je vše, přátelé!}
  \begin{center}
    \includegraphics[width=0.8\textwidth]{looney_tunes}
  \end{center}
\end{frame}

\end{document}

1   gnuplot
2   ruby
3   z mcnp
4   rake, latex


5   erb, more latex, vcs intro
  - jak funguje erb
  - load, require, def
  - tyce revisited
  - vyrobit z latexu tabulku
  - about vcs
  - github -- clone imf repo


%
% 6
% 7
% 8
% 9
% 10
%
%
% html, zase generovani
%
% webove aplikace
%
% yaml a xml
%
% pustit vypocet na linuxu
%
% generovani svg
%
% version control
%
% ---
% git@win?
% konto na lenochodech?\documentclass{beamer}

\def\Tiny{\fontsize{6pt}{6pt}\selectfont}
\def\supertiny{\fontsize{4pt}{4pt}\selectfont}

\mode<presentation>
{
  \usetheme{Warsaw}
  % \setbeamercovered{transparent}
  \usecolortheme{crane}
}


\usepackage{graphicx, ifthen, listings, fancyvrb}

\usepackage[czech]{babel}
% \usefonttheme{professionalfonts}
\usepackage{times}
\usepackage{amsmath}
\usepackage[utf8]{inputenc}
\usepackage{wrapfig}

\usepackage[T1]{fontenc}

\lstset{ basicstyle=\tiny, stringstyle=\ttfamily, showstringspaces=false }

\everymath{\displaystyle}

\setbeamerfont{frametitle}{size=\large}
\setbeamerfont{subsection in toc}{size=\scriptsize}

\makeatletter\newenvironment{blackbox}{%
   \begin{lrbox}{\@tempboxa}\begin{minipage}{0.95\columnwidth}}{\end{minipage}\end{lrbox}%
   \colorbox{black}{\usebox{\@tempboxa}}
}\makeatother

\title[IMF (5)]{Informatika pro moderní fyziky (5)\\ XXX}

\author[Franti\v{s}ek HAVL\r{U}J, ORF ÚJV Řež]{Franti\v{s}ek HAVL\r{U}J\\{\scriptsize \emph{e-mail: haf@ujv.cz}}}

\date{akademický rok 2013/2014\\12. listopadu 2013}

\institute[ORF ÚJV Řež]
{ÚJV Řež\\oddělení Reaktorové fyziky a podpory palivového cyklu}

\AtBeginSection[]
{
\begin{frame}<beamer>
\frametitle{Obsah}
\tableofcontents[currentsection,hideothersubsections]
\end{frame}
}

\begin{document}

\begin{frame}
  \titlepage
\end{frame}

\begin{frame}
  \tableofcontents
\end{frame}

\section{Co jsme se naučili minule}

\begin{frame}{}
  \begin{itemize}
    \item komplexnější řešení úlohy na zpracování dat, vykreslování sady grafů
    \item organizace jednoduchých skriptů - rake
    \item rychlý úvod do LaTeXu
  \end{itemize}
\end{frame}

\section{Výroba dokumentu v praxi}

\begin{frame}{Úkol na dnešek}
  \begin{itemize}
    \item pro jeden blok JE mám provozní data - v určitých dnech hodnotu koncentrace kyseliny borité a axiálního ofsetu - pro několik kampaní (blíže neurčený počet)
    \item chci vyrobit přehledové PDF, které bude hezky prezentovat grafy obou veličin pro každou kampaň a k tomu i tabulky
    \item data pro jednotlivé kampaně mám v CSV souborech, každý má tři sloupce (datum, cB, AO)
  \end{itemize}
\end{frame}

\begin{frame}{Rozbor}
  \begin{itemize}
    \item načíst tabulky a vykreslit grafy umíme
    \item převést tabulky v CSV na tabulky v LaTeXu se záhy naučíme
    \item vložit obrázek do latexu taky umíme
    \item předem neznámý počet souborů nás netrápí (\texttt{Dir["*.csv"]})
  \end{itemize}
\end{frame}

\section{Na šablony chytře}

\begin{frame}{Úskalí šablon}
  \begin{itemize}
    \item snadno umíme nahradit jeden řetězec druhým
    \item trochu méně pohodlné pro větší bloky textu
    \item navíc by se hodila nějaká logika (cyklus) přímo v šabloně
    \item naštěstí jsou na to postupy
  \end{itemize}
\end{frame}

\begin{frame}{ERb (XXXX Ruby)}
  \begin{itemize}
    \item lepší šablona - ``aktivní text''
    \item používá se například ve webových aplikacích
    \item hodí se ale i na generování latexových dokumentů, resp. všude, kde nám nesejde na whitespace
    \item poměrně jednoduchá syntax, zvládne skoro všechno (viz předmět MAA3)
  \end{itemize}
\end{frame}





Základní syntaxe ERb (1)
Jakýkoli Ruby příkaz, přiřazení, výpočet ...
<% a = b + 5 %>
<% list = ary * ", " %>

Základní syntaxe ERb (2)
Pokud chci něco vložit, stačí přidat rovnítko
<%= a %>
<%= ary[1] %>
<%= b + 5 %>

Základní syntaxe ERb (3)
Radost je možnost použít bloky a tedy i iterátory apod.:
<% (1..5).each do |i| %>
Number <%= i %>
<% end %>
<% ary.each do |x| %>
Array contains <%= x %>
<% end %>

Důležité upozornění

- oddělení modelu a view
- přestože lze provádět zpracování dat a výpočty přímo v ERb, je to nejvíc nejhorší nápad
- je chytré si všechno připravit v modelu (tj. v Ruby skriptu, kterým data chystáme)
- a kód ve view (tj. v ERb šabloně) omezit na naprosté minimum






\begin{frame}[fragile]{Jak ze šablony udělat výsledek}
  \scriptsize
  \begin{block}{Příklad překladu ERb}
    \scriptsize
    \begin{verbatim}
require 'erb'

erb(template, filename, {:x => 1, :y => 2})
    \end{verbatim}
  \end{block}
\end{frame}




\begin{frame}[fragile]{Příklad -- kreslení grafů z minula}
  \begin{block}{template.gp}
    \scriptsize
    \begin{verbatim}
set terminal png
set output "plot_<%=n%>.png"
plot "data_<%=n%>.csv"
    \end{verbatim}
  \end{block}
  \begin{block}{}
    \scriptsize
    \begin{verbatim}
(1..10).each do |i|
  erb("template.gp", "plot_#{i}.gp", {:n => i})
end
    \end{verbatim}
  \end{block}
\end{frame}


\begin{frame}{A to je vše, přátelé!}
  \begin{center}
    \includegraphics[width=0.8\textwidth]{looney_tunes}
  \end{center}
\end{frame}

\end{document}

1   gnuplot
2   ruby
3   z mcnp
4   rake, latex


5   erb, more latex, vcs intro
  - jak funguje erb
  - load, require, def
  - tyce revisited
  - vyrobit z latexu tabulku
  - about vcs
  - github -- clone imf repo


%
% 6
% 7
% 8
% 9
% 10
%
%
% html, zase generovani
%
% webove aplikace
%
% yaml a xml
%
% pustit vypocet na linuxu
%
% generovani svg
%
% version control
%
% ---
% git@win?
% konto na lenochodech?