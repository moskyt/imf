\documentclass{beamer}

\def\Tiny{\fontsize{6pt}{6pt}\selectfont}
\def\supertiny{\fontsize{4pt}{4pt}\selectfont}

\mode<presentation>
{
  \usetheme{Warsaw}
  % \setbeamercovered{transparent}
  \usecolortheme{crane}
}

\usepackage{graphicx, ifthen, listings, fancyvrb}

\usepackage[czech]{babel}
% \usefonttheme{professionalfonts}
\usepackage{times}
\usepackage{amsmath}
\usepackage[utf8]{inputenc}
\usepackage{wrapfig}

\usepackage[T1]{fontenc}

\lstset{ basicstyle=\tiny, stringstyle=\ttfamily, showstringspaces=false }

\everymath{\displaystyle}

\setbeamerfont{frametitle}{size=\large}
\setbeamerfont{subsection in toc}{size=\scriptsize}

\makeatletter\newenvironment{blackbox}{%
   \begin{lrbox}{\@tempboxa}\begin{minipage}{0.95\columnwidth}}{\end{minipage}\end{lrbox}%
   \colorbox{black}{\usebox{\@tempboxa}}
}\makeatother

\title[IMF (5)]{Informatika pro moderní fyziky (5)\\ Tvorba textových dokumentů}

\author[Franti\v{s}ek HAVL\r{U}J, ORF ÚJV Řež]{Franti\v{s}ek HAVL\r{U}J\\{\scriptsize \emph{e-mail: haf@ujv.cz}}}

\date{akademický rok 2013/2014\\12. listopadu 2013}

\institute[ORF ÚJV Řež]
{ÚJV Řež\\oddělení Reaktorové fyziky a podpory palivového cyklu}

\AtBeginSection[]
{
\begin{frame}<beamer>
\frametitle{Obsah}
\tableofcontents[currentsection,hideothersubsections]
\end{frame}
}

\begin{document}

\begin{frame}
  \titlepage
\end{frame}

\begin{frame}
  \tableofcontents
\end{frame}

\section{Co jsme se naučili minule}

\begin{frame}{}
  \begin{itemize}
    \item komplexnější řešení úlohy na zpracování dat, vykreslování sady grafů
    \item organizace jednoduchých skriptů - rake
    \item rychlý úvod do LaTeXu
  \end{itemize}
\end{frame}

\section{Pracky pryč, padouchu!}

\begin{frame}{Klávesnice a myš}
  \begin{itemize}
    \item myš je dobrá na grafiku a jako alternativa k tabletu
    \item taky se hodí tam, kde se potřebuju přesouvat mezi položkami, které nemají jednoznačné pořadí nebo prostorový vztah (neseřazené ikony na ploše, rozhraní s mrakem oken a
    \item případně ještě na použití menu pro úkony, které dělám jednou za uherský rok
    \item naopak na programování je nejlepší na myš vůbec nešahat a používat skoro jenom klávesnici
    \item extrémní školy dokonce brojí proti kurzorovým šipkám, protože (na velké klávesnici) nutí měnit polohu rukou, což je pomalé
  \end{itemize}
\end{frame}

\begin{frame}{Přepínání jazyků}
  \begin{itemize}
    \item je dobré se mu vyhnout, protože to opravdu trochu otravuje (i když se s tím dá docela dobře žít, pokud máte dobrou klávesovou zkratku)
    \item rozhodně stojí za to zjistit -- například pro psaní v LaTeXu -- kde na české klávesnici máte potřebné speciální znaky (v tomto případě zejména backslash a složené závorky)
    \item chytré editory mají různé pochystávky a makra, která vám umožní se těmto speciálním znakům defacto vyhnout
  \end{itemize}
\end{frame}

\begin{frame}{Klávesové zkratky}
  \begin{itemize}
    \item jako s programováním -- musím se něco naučit / zapamatovat, ale pak mi to ušetří hromadu času
    \item minimálně základní sadu stojí za to se naučit
    \item často jdou ručně editovat, ale většinou to není nutné (a je to stejně na houby, pokud zrovna nesedíte u svého počítače)
    \item jako s hudebním nástrojem -- za čas už neznáte ty zkratky, ale prostě je umíte zmáčknout bez přemýšlení
    \item hodně jich je sdílených napříč programy a editory
  \end{itemize}
\end{frame}

\begin{frame}{Klávesové zkratky - MS Windows}
  \begin{itemize}
    \item copy-paste
    \item undo
    \item přepínání aplikací
    \item přepínání oken v rámci aplikace
  \end{itemize}
\end{frame}

\begin{frame}{Klávesové zkratky -- Notepad++}
  \begin{itemize}
    \item pohyb v textu po slovech a stránkách
    \item uložení, otevření, zavření
    \item změna odsazení bloku \texttt{Tab / Shift+Tab}
    \item přepínání mezi soubory
    \item zakomentovat/odkomentovat \texttt{Ctrl+Q}
  \end{itemize}
\end{frame}

\section{Výroba dokumentu v praxi}

\begin{frame}{Úkol na dnešek}
  \begin{itemize}
    \item pro jeden blok JE mám provozní data - v určitých dnech hodnotu koncentrace kyseliny borité a axiálního ofsetu - pro několik kampaní (blíže neurčený počet)
    \item chci vyrobit přehledové PDF, které bude hezky prezentovat grafy obou veličin pro každou kampaň a k tomu i tabulky
    \item data pro jednotlivé kampaně mám v CSV souborech, každý má tři sloupce (datum, cB, AO)
  \end{itemize}
\end{frame}

\begin{frame}{Rozbor}
  \begin{itemize}
    \item načíst tabulky a vykreslit grafy umíme
    \item převést tabulky v CSV na tabulky v LaTeXu se záhy naučíme
    \item vložit obrázek do latexu taky umíme
    \item předem neznámý počet souborů nás netrápí (\texttt{Dir["*.csv"]})
  \end{itemize}
\end{frame}

1) jak na ty tabulky
- chytre: vyrobit tabulku v souboru zvlast a pak input
- zaklad tabulky: prostredi tabular, radky oddelene \\ a sloupce &

2,3,4) tabulka v prikladech

ukol ted: vytahnout z trisloupcoveho CSV dvousloupcovou latex tabulku

(ukazat, s vysledkem)

pak: do tri dvousloupcu

(vysledek)

(ukazat)

\begin{frame}[fragile]{Jak na tabulky}
  \begin{itemize}
    \item tabulky budou dost rozsáhlé a montovat
    \item naštěstí má LaTex příkaz \texttt{\input}, kterým můžeme prostě vložit do dokumentu nějaký externí soubor
    \item takže si nejdřív přichystáme soubory s tabulkami a pak se na ně budeme už jenom odkazovat
  \end{itemize}
\end{frame}

\begin{frame}[fragile]{Jak na tabulky v LaTeXu (1)}
  \begin{block}{ }
    Základem tabulky je prostředí \texttt{tabular} s definicí počtu a zarovnání sloupců:
    \scriptsize
    \begin{verbatim}
\begin{tabular}
\end{tabular}
    \end{verbatim}
  \end{block}
\end{frame}

\begin{frame}[fragile]{Jak na tabulky v LaTeXu (2)}
  \begin{block}{ }
    Základem tabulky je prostředí \texttt{tabular} s definicí počtu a zarovnání sloupců:
    \scriptsize
    \begin{verbatim}
\begin{tabular}
\end{tabular}
    \end{verbatim}
  \end{block}
\end{frame}

\begin{frame}[fragile]{Jak na tabulky v LaTeXu (3)}
  \begin{block}{ }
    Základem tabulky je prostředí \texttt{tabular} s definicí počtu a zarovnání sloupců:
    \scriptsize
    \begin{verbatim}
\begin{tabular}
\end{tabular}
    \end{verbatim}
  \end{block}
\end{frame}



\section{Na šablony chytře}

\begin{frame}{Úskalí šablon}
  \begin{itemize}
    \item snadno umíme nahradit jeden řetězec druhým
    \item trochu méně pohodlné pro větší bloky textu
    \item navíc by se hodila nějaká logika (cyklus) přímo v šabloně
    \item naštěstí jsou na to postupy
  \end{itemize}
\end{frame}

\begin{frame}{ERb (Embedded Ruby)}
  \begin{itemize}
    \item lepší šablona - ``aktivní text''
    \item používá se například ve webových aplikacích
    \item hodí se ale i na generování latexových dokumentů, resp. všude, kde nám nesejde na whitespace
    \item poměrně jednoduchá syntax, zvládne skoro všechno (viz předmět MAA3)
  \end{itemize}
\end{frame}


\begin{frame}[fragile]{Základní syntaxe ERb (1)}
  \begin{block}{ }
    Jakýkoli Ruby příkaz, přiřazení, výpočet ...
    \scriptsize
    \begin{verbatim}
      <% a = b + 5 %>
      <% list = ary * ", " %>
    \end{verbatim}
  \end{block}
\end{frame}

\begin{frame}[fragile]{Základní syntaxe ERb (2)}
  \begin{block}{ }
    Pokud chci něco vložit, stačí přidat rovnítko
    \scriptsize
    \begin{verbatim}
      <%= a %>
      <%= ary[1] %>
      <%= b + 5 %>
    \end{verbatim}
  \end{block}
\end{frame}

\begin{frame}[fragile]{Základní syntaxe ERb (3)}
  \begin{block}{ }
    Radost je možnost použít bloky a tedy i iterátory apod. v propojení s vkládaným textem:
    \scriptsize
    \begin{verbatim}
      <% (1..5).each do |i| %>
      Number <%= i %>
      <% end %>
      <% ary.each do |x| %>
      Array contains <%= x %>
      <% end %>
    \end{verbatim}
  \end{block}
\end{frame}

\begin{frame}{ERb -- shrnutí}
  \begin{itemize}
    \item dobrý sluha, ale špatný pán
    \item můžu s tím vyrobit hromadu užitečných věcí na malém prostoru
    \item daň je velké riziko zamotaného kódu a nízké přehlednosti (struktura naprosto není patrná na první pohled, proto je namístě ji držet maximálně jednoduchou)
  \end{itemize}
\end{frame}

\begin{frame}{Důležité upozornění}
  \begin{itemize}
    \item oddělení modelu a view
    \item přestože lze provádět zpracování dat a výpočty přímo v ERb, je to nejvíc nejhorší nápad
    \item je chytré si všechno připravit v modelu (tj. v Ruby skriptu, kterým data chystáme)
    \item a kód ve view (tj. v ERb šabloně) omezit na naprosté minimum
  \end{itemize}
\end{frame}

\begin{frame}[fragile]{Jak ze šablony udělat výsledek}
  \scriptsize
  \begin{block}{Příklad překladu ERb}
    \scriptsize
    \begin{verbatim}
require 'erb'

erb(template, filename, {:x => 1, :y => 2})
    \end{verbatim}
  \end{block}
\end{frame}

\begin{frame}[fragile]{Příklad -- kreslení grafů z minula}
  \begin{block}{template.gp}
    \scriptsize
    \begin{verbatim}
set terminal png
set output "plot_<%=n%>.png"
plot "data_<%=n%>.csv"
    \end{verbatim}
  \end{block}
  \begin{block}{}
    \scriptsize
    \begin{verbatim}
(1..10).each do |i|
  erb("template.gp", "plot_#{i}.gp", {:n => i})
end
    \end{verbatim}
  \end{block}
\end{frame}


\begin{frame}{A to je vše, přátelé!}
  \begin{center}
    \includegraphics[width=0.8\textwidth]{looney_tunes}
  \end{center}
\end{frame}

\end{document}

% 1   gnuplot
% 2   ruby
% 3   z mcnp
% 4   rake, latex
%
%
% 5   erb, more latex, vcs intro
%   - jak funguje erb
%   - load, require, def
%   - tyce revisited
%   - vyrobit z latexu tabulku
%   - about vcs
%   - github -- clone imf repo
%

%
% 6
% 7
% 8
% 9
% 10
%
%
% html, zase generovani
%
% webove aplikace
%
% yaml a xml
%
% pustit vypocet na linuxu
%
% generovani svg
%
% version control
%
% ---
% git@win?
% konto na lenochodech?
