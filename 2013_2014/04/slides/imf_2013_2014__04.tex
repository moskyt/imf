\documentclass{beamer}

\def\Tiny{\fontsize{6pt}{6pt}\selectfont}
\def\supertiny{\fontsize{4pt}{4pt}\selectfont}

\mode<presentation>
{
  \usetheme{Warsaw}
  % \setbeamercovered{transparent}
  \usecolortheme{crane}
}


\usepackage{graphicx, ifthen, listings}

\usepackage[czech]{babel}
% \usefonttheme{professionalfonts}
\usepackage{times}
\usepackage{amsmath}
\usepackage[utf8]{inputenc}
\usepackage{wrapfig}

\usepackage[T1]{fontenc}

\lstset{ basicstyle=\tiny, stringstyle=\ttfamily, showstringspaces=false }

\everymath{\displaystyle}

\setbeamerfont{frametitle}{size=\large}
\setbeamerfont{subsection in toc}{size=\scriptsize}

\makeatletter\newenvironment{blackbox}{%
   \begin{lrbox}{\@tempboxa}\begin{minipage}{0.95\columnwidth}}{\end{minipage}\end{lrbox}%
   \colorbox{black}{\usebox{\@tempboxa}}
}\makeatother

\title[IMF (3)]{Informatika pro moderní fyziky (3)\\ladění programů a zpracování textu}

\author[Franti\v{s}ek HAVL\r{U}J, ORF ÚJV Řež]{Franti\v{s}ek HAVL\r{U}J\\{\scriptsize \emph{e-mail: haf@ujv.cz}}}

\date{akademický rok 2013/2014\\29. října 2013}

\institute[ORF ÚJV Řež]
{ÚJV Řež\\oddělení Reaktorové fyziky a podpory palivového cyklu}

\AtBeginSection[]
{
\begin{frame}<beamer>
\frametitle{Obsah}
\tableofcontents[currentsection,hideothersubsections]
\end{frame}
}

\begin{document}

\begin{frame}
  \titlepage
\end{frame}

\begin{frame}
  \tableofcontents
\end{frame}

\section{Co jsme se naučili minule}

\begin{frame}{}
  \begin{itemize}
    \item čtení ze souboru, automatické vyhledání výrazu
    \item jak dostat keff z výstupu MCNP
    \item výroba vstupních souborů pomocí šablon
  \end{itemize}
\end{frame}

\section{Opakování a dotažení}

\begin{frame}{Co dál}
  \begin{itemize}
    \item máme soubor s daty pro polohy tyčí (data/keff.csv)
    \item vykreslit graf! pro každou z 11 poloh R1 jedna čára (závislost keff na R2)
    \item (= csv soubor, gnuplot, znáte to)
    \item najít automaticky kritickou polohu R2 pro každou z 11 poloh R1
    \item a zase graf... (kritická poloha R2 v závislosti na R1)
  \end{itemize}
\end{frame}

\begin{frame}{Ruční hledání}
  \begin{itemize}
    \item v takových datech se hrozně špatně hledá
    \item pokud bych měl na 11 řádcích (pro každou polohu R1) 11 hodnot (pro každou polohu R2), tak už ``kouknu a vidím''
    \item na to se ale snadno napíše skript ...
  \end{itemize}
\end{frame}

\begin{frame}[fragile]{Úloha 2}
    \scriptsize
\begin{verbatim}


\end{verbatim}
\end{frame}


\begin{frame}[fragile]{Úloha 2}
    \scriptsize
\begin{verbatim}
0.94800 0.94850 0.95000 0.95250 0.95600 0.96050 0.96600 0.97250 0.98000 0.98850 0.99800
0.94900 0.94950 0.95100 0.95350 0.95700 0.96150 0.96700 0.97350 0.98100 0.98950 0.99900
0.95000 0.95050 0.95200 0.95450 0.95800 0.96250 0.96800 0.97450 0.98200 0.99050 1.00000
0.95100 0.95150 0.95300 0.95550 0.95900 0.96350 0.96900 0.97550 0.98300 0.99150 1.00100
0.95200 0.95250 0.95400 0.95650 0.96000 0.96450 0.97000 0.97650 0.98400 0.99250 1.00200
0.95300 0.95350 0.95500 0.95750 0.96100 0.96550 0.97100 0.97750 0.98500 0.99350 1.00300
0.95400 0.95450 0.95600 0.95850 0.96200 0.96650 0.97200 0.97850 0.98600 0.99450 1.00400
0.95500 0.95550 0.95700 0.95950 0.96300 0.96750 0.97300 0.97950 0.98700 0.99550 1.00500
0.95600 0.95650 0.95800 0.96050 0.96400 0.96850 0.97400 0.98050 0.98800 0.99650 1.00600
0.95700 0.95750 0.95900 0.96150 0.96500 0.96950 0.97500 0.98150 0.98900 0.99750 1.00700
0.95800 0.95850 0.96000 0.96250 0.96600 0.97050 0.97600 0.98250 0.99000 0.99850 1.00800
\end{verbatim}
\end{frame}


\begin{frame}{A to je vše, přátelé!}
  \begin{center}
    \includegraphics[width=0.8\textwidth]{looney_tunes}
  \end{center}
\end{frame}

\end{document}

% z minula - dotahnout generovani grafu, * hledani kriticke polohy (aspon walkthrough)
% - vic skriptu = Rakefile
% - vygenerovat zpravu - jak? plaintext blbej
% - HTML? mozno, ale potreba js/jquery/bootstrap...
% - idealne nejaky metaformat nebo neco.....
% - oddeleni obsahu a formy
% - LaTeX
% - zakladni principy
% - pouzit sablonu a jet
% - erb
% - require, knihovny ...
% - > ???
