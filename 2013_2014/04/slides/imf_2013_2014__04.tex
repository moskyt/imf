\documentclass{beamer}

\def\Tiny{\fontsize{6pt}{6pt}\selectfont}
\def\supertiny{\fontsize{4pt}{4pt}\selectfont}

\mode<presentation>
{
  \usetheme{Warsaw}
  % \setbeamercovered{transparent}
  \usecolortheme{crane}
}


\usepackage{graphicx, ifthen, listings}

\usepackage[czech]{babel}
% \usefonttheme{professionalfonts}
\usepackage{times}
\usepackage{amsmath}
\usepackage[utf8]{inputenc}
\usepackage{wrapfig}

\usepackage[T1]{fontenc}

\lstset{ basicstyle=\tiny, stringstyle=\ttfamily, showstringspaces=false }

\everymath{\displaystyle}

\setbeamerfont{frametitle}{size=\large}
\setbeamerfont{subsection in toc}{size=\scriptsize}

\makeatletter\newenvironment{blackbox}{%
   \begin{lrbox}{\@tempboxa}\begin{minipage}{0.95\columnwidth}}{\end{minipage}\end{lrbox}%
   \colorbox{black}{\usebox{\@tempboxa}}
}\makeatother

\title[IMF (3)]{Informatika pro moderní fyziky (3)\\ladění programů a zpracování textu}

\author[Franti\v{s}ek HAVL\r{U}J, ORF ÚJV Řež]{Franti\v{s}ek HAVL\r{U}J\\{\scriptsize \emph{e-mail: haf@ujv.cz}}}

\date{akademický rok 2013/2014\\29. října 2013}

\institute[ORF ÚJV Řež]
{ÚJV Řež\\oddělení Reaktorové fyziky a podpory palivového cyklu}

\AtBeginSection[]
{
\begin{frame}<beamer>
\frametitle{Obsah}
\tableofcontents[currentsection,hideothersubsections]
\end{frame}
}

\begin{document}

\begin{frame}
  \titlepage
\end{frame}

\begin{frame}
  \tableofcontents
\end{frame}

\section{Co jsme se naučili minule}

\begin{frame}{}
  \begin{itemize}
    \item čtení ze souboru, automatické vyhledání výrazu
    \item jak dostat keff z výstupu MCNP
    \item výroba vstupních souborů pomocí šablon
  \end{itemize}
\end{frame}

\section{Opakování a dotažení}

\begin{frame}{Co dál}
  \begin{itemize}
    \item máme soubor s daty pro polohy tyčí (data/keff.csv)
    \item vykreslit graf! pro každou z 11 poloh R1 jedna čára (závislost keff na R2)
    \item (= csv soubor, gnuplot, znáte to)
    \item najít automaticky kritickou polohu R2 pro každou z 11 poloh R1
    \item a zase graf... (kritická poloha R2 v závislosti na R1)
  \end{itemize}
\end{frame}

\begin{frame}{Ruční hledání}
  \begin{itemize}
    \item v takových datech se hrozně špatně hledá
    \item pokud bych měl na 11 řádcích (pro každou polohu R1) 11 hodnot (pro každou polohu R2), tak už ``kouknu a vidím''
    \item na to se ale snadno napíše skript ...
  \end{itemize}
\end{frame}

\begin{frame}[fragile]{Úloha 2}
    \scriptsize
\begin{verbatim}
  keff = {}
  IO.foreach("keff.csv") do |line|
    ary = line.strip.split
    r1, r2, k = ary[0].to_i, ary[1].to_i, ary[2].to_f
    keff[[r1,r2]] = k
  end

  (0..10).each do |i|
    (0..10).each do |j|
      print "%8.5f" % keff[[i*10, j*10]]
    end
    puts
  end
\end{verbatim}
\end{frame}


\begin{frame}[fragile]{Úloha 2}
    \scriptsize
\begin{verbatim}
0.94800 0.94850 0.95000 0.95250 0.95600 0.96050 0.96600 0.97250 0.98000 0.98850 0.99800
0.94900 0.94950 0.95100 0.95350 0.95700 0.96150 0.96700 0.97350 0.98100 0.98950 0.99900
0.95000 0.95050 0.95200 0.95450 0.95800 0.96250 0.96800 0.97450 0.98200 0.99050 1.00000
0.95100 0.95150 0.95300 0.95550 0.95900 0.96350 0.96900 0.97550 0.98300 0.99150 1.00100
0.95200 0.95250 0.95400 0.95650 0.96000 0.96450 0.97000 0.97650 0.98400 0.99250 1.00200
0.95300 0.95350 0.95500 0.95750 0.96100 0.96550 0.97100 0.97750 0.98500 0.99350 1.00300
0.95400 0.95450 0.95600 0.95850 0.96200 0.96650 0.97200 0.97850 0.98600 0.99450 1.00400
0.95500 0.95550 0.95700 0.95950 0.96300 0.96750 0.97300 0.97950 0.98700 0.99550 1.00500
0.95600 0.95650 0.95800 0.96050 0.96400 0.96850 0.97400 0.98050 0.98800 0.99650 1.00600
0.95700 0.95750 0.95900 0.96150 0.96500 0.96950 0.97500 0.98150 0.98900 0.99750 1.00700
0.95800 0.95850 0.96000 0.96250 0.96600 0.97050 0.97600 0.98250 0.99000 0.99850 1.00800
\end{verbatim}
\end{frame}

\begin{frame}{Najít kritickou polohu automaticky}
  \begin{itemize}
    \item pro každou polohu R1 snadno najdu odpovídající R2
    \item jak? najdu nejdřív interval, ve kterém leží keff=1
    \item a potom už je to jen obyčejná lineární interpolace
    \item (nevzdávat, vymyslet!)
    \item a nakonec zase nakreslit graf
  \end{itemize}
\end{frame}

\section{Hrabání listí dělá pořádek (Rake)}

\begin{frame}{Spousta skriptů, spousta zmatku}
  \begin{itemize}
    \item mám jeden projekt/práci a potřebuju udělat víc věcí
    \item zatím jsme měli jeden skript na jednu věc
    \item což skončí hromadou .rb souborů, kde nebudu vědět co dělá který a budu v tom mít trochu zmatek
    \item nehledě na to, že bych mohl chtít sdílet nějakou konfiguraci (jména souborů atd.)
  \end{itemize}
\end{frame}

\begin{frame}{Nástroj Rake}
  \begin{itemize}
    \item alternativa k unixovému MAKE, ale v Ruby (Ruby MAKE = Rake)
    \item nejjednodušší -- nastrkám si do jednoho Rakefile víc úloh (\texttt{task}) a ty pak snadno spustím
    \item složitější -- můžu specifikovat závislosti
  \end{itemize}
\end{frame}

\begin{frame}[fragile]{Rakefile - příklad}
  \begin{block}{obsah Rakefile}
        \scriptsize
    \begin{verbatim}
    desc 'rearrange keff into a nice table'
    task :rearrange do
    ...
    end

    desc ''
    task :find do
    ...
    end
    \end{verbatim}
  \end{block}
  \begin{block}{spuštění}
        \scriptsize
    \begin{verbatim}
rake find
rake -T
    \end{verbatim}
  \end{block}
\end{frame}

\section{Závěrečná zpráva}

\begin{frame}{Jak vyrobit zprávu?}
  \begin{itemize}
    \item potřebujeme udělat hezké PDF shrnující výsledky našich výpočtů
    \item takže úvod, popis toho co jsme dělali a pak přehled výsledků
    \item tabulka s hodnotami, 11+1 graf
    \item co by znamenalo to dělat ve Wordu ..... ?
    \item hodilo by se to zautomatizovat!
  \end{itemize}
\end{frame}

\begin{frame}{Jak vygenerovat text?}
  \begin{itemize}
    \item zase potřebujeme lepší nástroj na text, než jsou WYSIWYG (What You See Is What You Get) editory
    \item ideálně něco, co bude mít plain-text vstup (který můžeme s úspěchem generovat v Ruby) a co se pak 
    \item odpověď zní LaTeX (\LaTeX) /ˈleɪtɛk/
    \item nejvíc nejlepší text-processor na světě
  \end{itemize}
\end{frame}

\begin{frame}{Koncepce oddělení obsahu a formy}
  \begin{itemize}
    \item když píšu, nechci říct, že je text tučně a o dva body větší, ale že je to nadpis kapitoly
    \item ideálně chci popsat někde, jak bude dokument
    \item styly ve Wordu se tomu vzdáleně blíží
    \item v LaTeXu vlastně píšu jen obsah a o formu se musím starat jen hodně málo
    \item je samozřejmostí zadarmo obsah, rejstřík atd.
    \item kdo píše diplomku v něčem jiném, kazí si život
  \end{itemize}
\end{frame}

\begin{frame}{Příklad jednoduchého dokumentu}
  \begin{itemize}
    \item viz (document.tex)
    \item není potřeba úplně všemu rozumět, zatím to jen budeme upravovat v mezích zákona
    \item všechny příkazy začínají zpětným lomítkem
    \item napíšu \texttt{pdflatex document.tex} a dostanu pdfko!
  \end{itemize}
\end{frame}

\begin{frame}{A to je vše, přátelé!}
  \begin{center}
    \includegraphics[width=0.8\textwidth]{looney_tunes}
  \end{center}
\end{frame}

\end{document}

% z minula - dotahnout generovani grafu, * hledani kriticke polohy (aspon walkthrough)
% - vic skriptu = Rakefile
% - vygenerovat zpravu - jak? plaintext blbej
% - HTML? mozno, ale potreba js/jquery/bootstrap...
% - idealne nejaky metaformat nebo neco.....
% - oddeleni obsahu a formy
% - LaTeX
% - zakladni principy
% - pouzit sablonu a jet
% - erb
% - require, knihovny ...
% - > ???
