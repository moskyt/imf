\documentclass{beamer}

\def\Tiny{\fontsize{6pt}{6pt}\selectfont}
\def\supertiny{\fontsize{4pt}{4pt}\selectfont}

\mode<presentation>
{
  \usetheme{Warsaw}
  % \setbeamercovered{transparent}
  \usecolortheme{crane}
}

\usepackage{graphicx, ifthen, listings, fancyvrb}

\usepackage[czech]{babel}
% \usefonttheme{professionalfonts}
\usepackage{times}
\usepackage{amsmath}
\usepackage[utf8]{inputenc}
\usepackage{wrapfig}

\usepackage[T1]{fontenc}

\lstset{ basicstyle=\tiny, stringstyle=\ttfamily, showstringspaces=false }

\everymath{\displaystyle}

\setbeamerfont{frametitle}{size=\large}
\setbeamerfont{subsection in toc}{size=\scriptsize}

\makeatletter\newenvironment{blackbox}{%
   \begin{lrbox}{\@tempboxa}\begin{minipage}{0.95\columnwidth}}{\end{minipage}\end{lrbox}%
   \colorbox{black}{\usebox{\@tempboxa}}
}\makeatother

\title[IMF (10)]{Informatika pro moderní fyziky (10)\\ konfigurační soubory YML, formát JSON, použití cizích API, mash-up aplikace}

\author[Franti\v{s}ek HAVL\r{U}J, ORF ÚJV Řež]{Franti\v{s}ek HAVL\r{U}J\\{\scriptsize \emph{e-mail: haf@ujv.cz}}}

\date{akademický rok 2019/2020\\27. listopadu 2019}

\institute[ORF ÚJV Řež]
{ÚJV Řež\\oddělení Reaktorové fyziky a podpory palivového cyklu}

\AtBeginSection[]
{
\begin{frame}<beamer>
\frametitle{Obsah}
\tableofcontents[currentsection,hideothersubsections]
\end{frame}
}

\begin{document}

\begin{frame}
  \titlepage
\end{frame}

\begin{frame}
  \tableofcontents
\end{frame}

\section{Tvorba obrázků}

\begin{frame}[fragile]{Dokončení z minula}
  \begin{itemize}
    \item v posledním souboru jsou navíc neznámé typy XX a YY -- vhodně vyřešte:
    \item a) náhodná barva \texttt{rand}, \texttt{rand(123)}
    \item b) seznam barev pro neznámé typy
    \item vylepšení: načítejme barvy ze souboru! mohlo by to vypadat takto: (alepoužijeme YAML - viz dále)
  \end{itemize}
\begin{verbatim}
F4  ff0000
R1  00ffff
R2  00eeee
\end{verbatim}
\end{frame}

\begin{frame}[fragile]{YAML - kamarád pro konfigurační soubory}
  \begin{itemize}
    \item už je trochu nuda pořád ručně načítat soubory a parsovat je, normálně se to tak nedělá – použiju standardizovaný formát souboru
    \item na konfigurační soubory je skvělý formát YAML (YML) – jednoduchý hash formátovaný odsazením
    \item načtu standardní knihovnu \texttt{require ``yaml''}
    \item parsování řetězce \texttt{YAML.load(File.read(``config.yml''))}
    \item uložení dat \texttt{File.write(``config.yml'', data.to\_yaml)}
    \item s hvězdičkou: RRGGBB, \#RRGGBB i R,G,B
  \end{itemize}
\begin{verbatim}
colors:
  F4: ff0000
  R1: 00ffff
  R2: 00eeee
\end{verbatim}
\end{frame}

\section{Persistence dat a formát JSON}

\begin{frame}{Ukládání strukturovaných dat}
  \begin{itemize}
    \item často mám data ve formě struktury (kombinace hash+pole, různý stupeň vnoření)
    \item z různých důvodů můžu chtít data uložit na disk a pak je znovu načítat
    \item (zejména efektivita zpracování, případně data z externích/webových zdrojů)
    \item bylo by dobré mít možnost uložit a načíst rovnou celý hash
    \item odpověď jsou strukturované metaformáty - YAML, XML, JSON
  \end{itemize}
\end{frame}

\begin{frame}{Práce s JSON}
  \begin{itemize}
    \item v Ruby je k mání knihovna -- \texttt{require ``json''}
    \item generování JSON: \texttt{hash.to\_json}
    \item čtení JSON: \texttt{JSON[data]}
  \end{itemize}
\end{frame}

\begin{frame}{Příklad -- výsledky běhu kolem rybníka}
  \begin{itemize}
    \item dva soubory -- \texttt{ages.csv}, \texttt{times.csv}
    \item chci v jednom skriptu (tasku) načíst, spárovat a uložit
    \item a v jiném už rovnou načíst zpracovaná data
    \item a vypsat tabulku výsledků včetně ročníků narození
    \item pozor! JSON nezná symboly, uloží se jako řetězce
  \end{itemize}
\end{frame}

\section{Použití cizích API}

\begin{frame}{K čemu to?}
  \begin{itemize}
    \item spousta informací na webu je poskytována ve strojově čitelné formě
    \item API -- rozhraní mezi aplikacemi
    \item s využitím webových služeb naše možnosti exponenciálně rostou (počasí, doprava, mapy, atd atd.)
    \item spousta věcí se dá udělat jako \emph{mashup} -- sice nic neumím, ale umím to dát dohromady
  \end{itemize}
\end{frame}


\begin{frame}{Typy / formáty}
  \begin{itemize}
    \item URL -- rovnou dostanu např. obrázek po zadání správného URL
    \item XML -- velmi obecný, ale komplikovaný formát (``vypadá jako HTML'')
    \item JSON -- velmi jednoduchý a kompaktní formát, vyvinutý pro JS (v podstatě jen číslo, řetězec, pole, hash)
  \end{itemize}
\end{frame}


\begin{frame}{URL API -- google maps}
  \begin{itemize}
    \item stačí správně vymyslet
    \item pozor na usage limits (v produkci je nutné lokální cache...)
    \item QR platba: \\
    {\tiny \url{http://qr-platba.cz/pro-vyvojare/restful-api/\#generator-czech-image}}
    \item Google Maps static API: \\
    {\tiny \url{https://developers.google.com/maps/documentation/staticmaps/}}
  \end{itemize}
\end{frame}


\begin{frame}{Jednoduchý mashup: mapa o-závodů}
  \begin{itemize}
    \item ORIS API -- http://oris.orientacnisporty.cz/API
    \item úkol: vypišme kalendář MTBO závodů v roce 2017
    \item {\tiny \url{http://oris.orientacnisporty.cz/API/?format=json\&method=getEventList\&sport=3\&datefrom=2017-01-01\&dateto=2017-12-31}}
    \item {\tiny \url{https://developers.google.com/maps/documentation/static-maps/intro}}
    \item API klíč -- v praxi si musíte pořídit vlastní (ale nic to nestojí)
    \item AIzaSyC6\_NpMTN-8olOHWzaiwjeZ8eu\_J3XI1lM
  \end{itemize}
\end{frame}


\section{Interaktivní mapa}

\begin{frame}{Jednoduchý mashup: mapa o-závodů}
  \begin{itemize}
    \item seznam závodů máme z předchozího kroku, teď bychom jenom chtěli, aby byla mapa prozměnu od seznamu a klikací
    \item {\tiny \url{http://oris.orientacnisporty.cz/API/?format=json\&method=getEventList\&sport=3\&datefrom=2017-01-01\&dateto=2017-12-31}}
    \item {\tiny \url{http://api.mapy.cz}}
  \end{itemize}
\end{frame}

\begin{frame}{Aspoň něco bychom si k tomu říct měli}
  \begin{itemize}
    \item HTML: jazyk (založený na XML, takže nám trochu připomene SVG)
    \item JS: jazyk povětšinou běžící v prohlížeči, umožňující client-side interaktivitu (vypadá jako něco mezi C++ a Ruby)
    \item největší radost udělají spolu
  \end{itemize}
\end{frame}

\begin{frame}{Klasický postup: upravovat vzor}
  Většinou máme k dispozici nějaký příklad, ze kterého se dá vyjít – a to často i v situaci, kdy o tématu nevíme vůbec nic
  \begin{itemize}
    \item {\tiny \url{http://api.mapy.cz/view?page=instruction}}
    \item {\tiny \url{http://api.mapy.cz/view?page=markerlayer}}
  \end{itemize}
\end{frame}

\begin{frame}{A to je vše, přátelé!}
  \begin{center}
    \includegraphics[width=0.8\textwidth]{looney_tunes}
  \end{center}
\end{frame}

\end{document}
