\documentclass{beamer}

\def\Tiny{\fontsize{6pt}{6pt}\selectfont}
\def\supertiny{\fontsize{4pt}{4pt}\selectfont}

\mode<presentation>
{
  \usetheme{Warsaw}
  % \setbeamercovered{transparent}
  \usecolortheme{crane}
}


\usepackage{graphicx, ifthen, listings}

\usepackage[czech]{babel}
% \usefonttheme{professionalfonts}
\usepackage{times}
\usepackage{amsmath}
\usepackage[utf8]{inputenc}
\usepackage{wrapfig}

\usepackage[T1]{fontenc}

\lstset{ basicstyle=\tiny, stringstyle=\ttfamily, showstringspaces=false }

\everymath{\displaystyle}

\setbeamerfont{frametitle}{size=\large}
\setbeamerfont{subsection in toc}{size=\scriptsize}

\makeatletter\newenvironment{blackbox}{%
   \begin{lrbox}{\@tempboxa}\begin{minipage}{0.95\columnwidth}}{\end{minipage}\end{lrbox}%
   \colorbox{black}{\usebox{\@tempboxa}}
}\makeatother

\title[IMF (4)]{Informatika pro moderní fyziky (4)\\vstupní a výstupní soubory pro výpočetní programy}

\author[Franti\v{s}ek HAVL\r{U}J, ORF ÚJV Řež]{Franti\v{s}ek HAVL\r{U}J\\{\scriptsize \emph{e-mail: haf@ujv.cz}}}

\date{akademický rok 2019/2020\\16. října 2019}

\institute[ORF ÚJV Řež]
{ÚJV Řež\\oddělení Reaktorové fyziky a podpory palivového cyklu}

\AtBeginSection[]
{
\begin{frame}<beamer>
\frametitle{Obsah}
\tableofcontents[currentsection,hideothersubsections]
\end{frame}
}

\begin{document}

\begin{frame}
  \titlepage
\end{frame}

\begin{frame}
  \tableofcontents
\end{frame}

\section{Kde je chyba?}

\begin{frame}{Ladění programů}
  \begin{itemize}
    \item v každém programu je aspoň jedna chyba
    \item není důležité nedělat chyby, ale je nutné je umět najít
    \item když si program/Ruby na něco stěžuje, tak si to přečtěte, jinak se nic nedozvíte
    \item pokud nepoznám, v čem je chyba, jsem bezezbytku ztracen
    \item následují tři úlohy, kde je úkolem najít všechny chyby
    \item z didaktických důvodů postupujte metodou tupého spouštění a postupného opravování
  \end{itemize}
\end{frame}

\section{Hrabání listí dělá pořádek (Rake)}

\begin{frame}{Spousta skriptů, spousta zmatku}
  \begin{itemize}
    \item mám jeden projekt/práci a potřebuju udělat víc věcí
    \item zatím jsme měli jeden skript na jednu věc
    \item což skončí hromadou .rb souborů, kde nebudu vědět co dělá který a budu v tom mít trochu zmatek
    \item nehledě na to, že bych mohl chtít sdílet nějakou konfiguraci (jména souborů atd.)
  \end{itemize}
\end{frame}

\begin{frame}{Nástroj Rake}
  \begin{itemize}
    \item alternativa k unixovému MAKE, ale v Ruby (Ruby MAKE = Rake)
    \item nejjednodušší -- nastrkám si do jednoho Rakefile víc úloh (\texttt{task}) a ty pak snadno spustím
    \item složitější -- můžu specifikovat závislosti
  \end{itemize}
\end{frame}

\begin{frame}[fragile]{Rakefile - příklad}
  \begin{block}{obsah Rakefile}
        \scriptsize
    \begin{verbatim}
    desc "rearrange keff into a nice table"
    task :rearrange do
    ...
    end

    desc "find something somewhere"
    task :find do
    ...
    end
    \end{verbatim}
  \end{block}
  \begin{block}{spuštění}
        \scriptsize
    \begin{verbatim}
rake find
rake -T
    \end{verbatim}
  \end{block}
\end{frame}

\begin{frame}[fragile]{Spouštění programů z Ruby}
  Je otrava psát pořád cestu ke gnuplotu a vůbec, takže lze samozřejmě vyrobit rake task:
    \scriptsize
\begin{verbatim}
task :plot do
  system("\"C:/Program Files/gnuplot/bin/gnuplot.exe\" plot1.gp")
end
\end{verbatim}
\end{frame}

\section{Zpracování textu -- dokončení a rozšíření}

\subsection{Zápis všech výsledků do tabulky}

\begin{frame}{Jak na to}
  Máme všechno, co potřebujeme:
  \begin{itemize}
    \item načtení keff z~jednoho výstupního souboru (\texttt{File.foreach}, \texttt{include} a \texttt{split})
    \item procházení adresáře (\texttt{Dir.each})
    \item zápis do souboru (\texttt{File.open} s~parametrem \texttt{w})
  \end{itemize}
  Takže už to stačí jen vhodným způsobem spojit dohromady!
\end{frame}

\begin{frame}[fragile]{Realizace}
  \scriptsize
  \begin{verbatim}
    Dir["*o"].each do |filename|
      keff = nil

      File.foreach(filename) do |line|
        if line.include?("final estimated combined")
          a = line.split("=")
          b = a[1].split
          keff = b[0]
        end
      end

      puts "#{filename} #{keff}"
    end
  \end{verbatim}
\end{frame}

\begin{frame}[fragile]{Výstup}
  Výsledkem je perfektní tabulka:
  {\scriptsize
  \begin{verbatim}
    outputs/c_0_0o 0.94800
    outputs/c_0_10o 0.99800
    outputs/c_0_1o 0.94850
    outputs/c_0_2o 0.95000
    outputs/c_0_3o 0.95250
    outputs/c_0_4o 0.95600
    ...
  \end{verbatim}}
  Hloupé je, že nikde nemáme tu polohu tyčí.
\end{frame}

\begin{frame}[fragile]{Víc by se nám hodilo}
  něco jak toto:
  {\scriptsize
  \begin{verbatim}
    0   0  0.94800
    0 640  0.99800
    0  64  0.94850
    0 128  0.95000
    0 192  0.95250
    0 256  0.95600
    ...
  \end{verbatim}
  }
  (rozsah je 0 -- 640, my máme kroky 0 -- 10)
\end{frame}


\begin{frame}[fragile]{A protože přehlednost je nade vše}
  něco jak toto:
  {\scriptsize
  \begin{verbatim}
    keff           0           64          128    ...
      0      0.94800      0.94900      0.95000    ...
     64      0.94850      0.94950      0.95050    ...
    128      0.95000      0.95100      0.95200    ...
    192      0.95250      0.95350      0.95450    ...
    256      0.95600      0.95700      0.95800    ...
    320      0.96050      0.96150      0.96250    ...
    384      0.96600      0.96700      0.96800    ...
    448      0.97250      0.97350      0.97450    ...
    512      0.98000      0.98100      0.98200    ...
    576      0.98850      0.98950      0.99050    ...
    640      0.99800      0.99900      1.00000    ...
    ...
  \end{verbatim}
  }
  -- alternativně můžete vyrobit tabulku v excelu!
\end{frame}

\begin{frame}{Navážeme na úspěchy z minulých týdnů}
  \begin{itemize}
    \item vykreslit graf! pro každou z 11 poloh R1 jedna čára (závislost keff na R2)
    \item (= csv soubor, gnuplot, znáte to)
    \item najít automaticky kritickou polohu R2 pro každou z 11 poloh R1
    \item a zase graf... (kritická poloha R2 v závislosti na R1)
  \end{itemize}
\end{frame}

\section{Načítání složitějšího výstupu}

\begin{frame}[fragile]{HELIOS}
  Tabulka výstupů:
  \scriptsize
  \begin{verbatim}
List name       : list
List Title(s)  1) This is a table
               2) of some data
               3) in many columns
               4) and has a long title!

               bup        kinf          ab          ab        u235        u238       pu239
 0001     0.00E+00     1.16949  9.7053E-03  7.6469E-02  1.8806E-04  7.5509E-03  1.0000E-20
 0002     0.00E+00     1.13213  9.7478E-03  7.9058E-02  1.8806E-04  7.5509E-03  1.0000E-20
 0003     1.00E+01     1.13149  9.7488E-03  7.9070E-02  1.8797E-04  7.5509E-03  2.3647E-09
 0004     5.00E+01     1.13004  9.7521E-03  7.9093E-02  1.8760E-04  7.5506E-03  5.3144E-08
 0005     1.00E+02     1.12826  9.7559E-03  7.9218E-02  1.8714E-04  7.5503E-03  1.8746E-07
 0006     1.50E+02     1.12664  9.7594E-03  7.9407E-02  1.8668E-04  7.5500E-03  3.7495E-07
 0007     2.50E+02     1.12399  9.7657E-03  7.9869E-02  1.8577E-04  7.5493E-03  8.4139E-07
 0008     5.00E+02     1.12007  9.7812E-03  8.1065E-02  1.8351E-04  7.5476E-03  2.1882E-06
 0009     1.00E+03     1.11561  9.8203E-03  8.3169E-02  1.7914E-04  7.5443E-03  4.8723E-06
 0010     2.00E+03     1.10542  9.9329E-03  8.6731E-02  1.7088E-04  7.5376E-03  9.6873E-06
 0011     3.00E+03     1.09354  1.0067E-02  8.9717E-02  1.6316E-04  7.5309E-03  1.3879E-05
 0012     4.00E+03     1.08126  1.0207E-02  9.2299E-02  1.5591E-04  7.5242E-03  1.7571E-05
 0013     6.00E+03     1.05755  1.0474E-02  9.6562E-02  1.4258E-04  7.5105E-03  2.3760E-05
  \end{verbatim}
\end{frame}

\begin{frame}{Co bychom chtěli}
  \begin{itemize}
    \item mít načtené jednotlivé tabulky (zatím jen jednu, ale bude jich víc)
    \item asi po jednotlivých sloupcích, sloupec = pole (hodnot po řádcích)
    \item sloupce se jmenují, tedy použijeme \texttt{Hash}
    \item \texttt{table['kinf']}
    \item pozor na \texttt{ab}, asi budeme muset vyrobit něco jako \texttt{ab1}, \texttt{ab2} (ale to až za chvíli)
  \end{itemize}
\end{frame}

\begin{frame}{Nástrahy, chytáky a podobně}
  \begin{itemize}
    \item tabulka skládající se z více bloků
    \item více tabulek
    \item tabulky mají jméno \texttt{list name} a popisek \texttt{list title(s)}
  \end{itemize}
\end{frame}

\begin{frame}{Jak uspořádat data?}
  \begin{itemize}
    \item pole s tabulkami + pole s názvy + pole s titulky?
    \pause
    \item co hashe tabulky[název] a titulky[název]?
    \pause
    \item nejchytřeji: \texttt{ \{\"a\" => \{:title => \"Table title\", data => \{\"kinf\" => ...\}\}\} }
    \pause
    \item \"nová\" syntaxe: \texttt{ \{\"a\" => \{title: \"Table title\", data: \{\"kinf\" => ...\}\}\} }
  \end{itemize}
\end{frame}

\begin{frame}{Z příkazové řádky}
  \begin{itemize}
    \item a co takhle z toho udělat skript, který lze pustit s argumentem = univerzální
    \item \texttt{ruby read\_helios.rb helios1.out}
    \item vypíše seznam všech tabulek, seznam jejich sloupců, počet řádků
    \item pole \texttt{ARGV}
    \item vylepšení – provede pro všechny zadané soubory: \texttt{ruby read\_helios.rb helios1.out helios2.out}
  \end{itemize}
\end{frame}

\section{Automatizace tvorby vstupů -- zobecnění}

\begin{frame}[fragile]{Určení poloh tyčí}
  Ve vstupním souboru si najdeme relevantní část:
  \scriptsize
  \begin{verbatim}
    c ---------------------------------
    c polohy tyci (z-plochy)
    c ---------------------------------
    c
    67 pz 47.6000    $ dolni hranice absoberu r1
    68 pz 40.4980    $ dolni hranice hlavice r1
    69 pz 44.8000    $ dolni hranice absoberu r2
    70 pz 37.6980    $ dolni hranice hlavice r2
  \end{verbatim}
\end{frame}

\begin{frame}[fragile]{Určení poloh tyčí}
  příklad \texttt{c1\_10\_20}:
  \scriptsize
  \begin{verbatim}
    c ---------------------------------
    c polohy tyci (z-plochy)
    c ---------------------------------
    c
    67 pz 47.6000    $ dolni hranice absoberu r1
    68 pz 40.4980    $ dolni hranice hlavice r1
    69 pz 44.8000    $ dolni hranice absoberu r2
    70 pz 37.6980    $ dolni hranice hlavice r2
  \end{verbatim}
\end{frame}

\begin{frame}[fragile]{Výroba šablon}
  Jak dostat polohy tyčí do vstupního souboru? Vyrobíme šablonu, tzn nahradíme
  \begin{verbatim}
    67 pz 47.6000    $ dolni hranice absoberu r1
  \end{verbatim}
  \pause
  nějakou značkou (\emph{placeholder}):
  \begin{verbatim}
    67 pz %r1%    $ dolni hranice absoberu r1
  \end{verbatim}
\end{frame}

\begin{frame}{A co takhle trocha zobecnění?}
  \begin{itemize}
    \item když budu chtít přidat další tyče nebo jiné parametry, bude to děsně bobtnat
    \item funkce \texttt{process("template", "inputs/c\_\#\{i1\}\_\#\{i2\}", \{'r1' => r1, 'r2' => r2, .....\})}
    \item všechno víme, známe, umíme
  \end{itemize}
\end{frame}

\begin{frame}{A to je vše, přátelé!}
  \begin{center}
    \includegraphics[width=0.8\textwidth]{looney_tunes}
  \end{center}
\end{frame}

\end{document}
