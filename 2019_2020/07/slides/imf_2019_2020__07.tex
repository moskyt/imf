\documentclass{beamer}

\def\Tiny{\fontsize{6pt}{6pt}\selectfont}
\def\supertiny{\fontsize{4pt}{4pt}\selectfont}

\mode<presentation>
{
  \usetheme{Warsaw}
  % \setbeamercovered{transparent}
  \usecolortheme{crane}
}


\usepackage{graphicx, ifthen, listings}

\usepackage[czech]{babel}
% \usefonttheme{professionalfonts}
\usepackage{times}
\usepackage{amsmath}
\usepackage[utf8]{inputenc}
\usepackage{wrapfig}

\usepackage[T1]{fontenc}

\lstset{ basicstyle=\tiny, stringstyle=\ttfamily, showstringspaces=false }

\everymath{\displaystyle}

\setbeamerfont{frametitle}{size=\large}
\setbeamerfont{subsection in toc}{size=\scriptsize}

\makeatletter\newenvironment{blackbox}{%
   \begin{lrbox}{\@tempboxa}\begin{minipage}{0.95\columnwidth}}{\end{minipage}\end{lrbox}%
   \colorbox{black}{\usebox{\@tempboxa}}
}\makeatother

\title[IMF (7)]{Informatika pro moderní fyziky (7)\\tvorba textových dokumentů}

\author[Franti\v{s}ek HAVL\r{U}J, ORF ÚJV Řež]{Franti\v{s}ek HAVL\r{U}J\\{\scriptsize \emph{e-mail: haf@ujv.cz}}}

\date{akademický rok 2019/2020\\6. listopadu 2019}

\institute[ORF ÚJV Řež]
{ÚJV Řež\\oddělení Reaktorové fyziky a podpory palivového cyklu}

\AtBeginSection[]
{
\begin{frame}<beamer>
\frametitle{Obsah}
\tableofcontents[currentsection,hideothersubsections]
\end{frame}
}

\begin{document}

\begin{frame}
  \titlepage
\end{frame}

\begin{frame}
  \tableofcontents
\end{frame}

\section{Výroba dokumentu v praxi}

\begin{frame}{Úkol na dnešek}
  \begin{itemize}
    \item vyrobit hezčí tabulky (rámeček, formátování data)
    \item přepracovat tabulky do tří sloupců --  chytré je vyrobit tabulku třeba o šesti sloupcích (jakože tři dvousloupce), pak už se to na stránku v klidu vejde

    \item vymyslet něco lepšího, než postupné zapisování do souboru
  \end{itemize}
\end{frame}

\begin{frame}[fragile]{Jak na tabulky}
  \begin{itemize}
    \item tabulky budou dost rozsáhlé a montovat je přímo nějak do latexových vstupů je asi spíš nepraktické, naštěstí to jde i jinak
    \item naštěstí má LaTex příkaz \texttt{\textbackslash{}input}, kterým můžeme prostě vložit do dokumentu nějaký externí soubor
    \item takže si nejdřív přichystáme soubory s tabulkami a pak se na ně budeme už jenom odkazovat
  \end{itemize}
\end{frame}


\begin{frame}[fragile]{Vícesloupcová tabulka - postup}
  \begin{itemize}
    \item načtu data do paměti -- 2D pole \texttt{[ ["1.2.2019", 4.3, -4.8], ["3.2.2019", 4.19, -4.5], ...]}
    \item spočítám počet řádků \texttt{k} (pozor, 22/3 = 7)
    \item vygeneruju tabulku pro BC -- na prvním řádku budu chtít prvky \texttt{0}, \texttt{0+k}, \texttt{0+2*k} a tedy například boritou v prostředním sloupci dostanu jako \texttt{data[0+k][1]} a datum v posledním sloupci jako \texttt{data[0+2*k][0]}
    \item dám pozor v posledním řádku, aby poslední prvky existovaly
    \item vygeneruju tabulku pro AO a ohlídám si, aby to nebylo copy-paste toho samého kódu
    \item pamatuju si, že generuju tabulky do samostatného souboru a do hlavního latexového dokumentu je vkládám pomocí příkazu input
  \end{itemize}
\end{frame}

\section{Na šablony chytře}

\begin{frame}{Co se nám nelíbí na generování vstupu}
  \begin{itemize}
    \item je to hrozně roztahané uvnitř zdrojáku
    \item je to dost nepřehledné
    \item nevidíme strukturu texového dokumentu
    \item představte si složitější dokument...
  \end{itemize}
\end{frame}

\begin{frame}{ERb (Embedded Ruby)}
  \begin{itemize}
    \item lepší šablona - ``aktivní text''
    \item používá se například ve webových aplikacích
    \item hodí se ale i na generování latexových dokumentů, resp. všude, kde nám nesejde na whitespace
    \item poměrně jednoduchá syntaxe, zvládne skoro všechno
  \end{itemize}
\end{frame}

\begin{frame}[fragile]{Základní syntaxe ERb (1)}
  \begin{block}{ }
    Jakýkoli Ruby příkaz, přiřazení, výpočet ...
    \scriptsize
    \begin{verbatim}
      <% a = b + 5 %>
      <% list = ary * ", " %>
    \end{verbatim}
  \end{block}
\end{frame}

\begin{frame}[fragile]{Základní syntaxe ERb (2)}
  \begin{block}{ }
    Pokud chci něco vložit, stačí přidat rovnítko
    \scriptsize
    \begin{verbatim}
      <%= a %>
      <%= ary[1] %>
      <%= b + 5 %>
    \end{verbatim}
  \end{block}
\end{frame}

\begin{frame}[fragile]{Základní syntaxe ERb (3)}
  \begin{block}{ }
    Radost je možnost použít bloky a tedy i iterátory apod. v propojení s vkládaným textem:
    \scriptsize
    \begin{verbatim}
      <% (1..5).each do |i| %>
      Number <%= i %>
      <% end %>
      <% ary.each do |x| %>
      Array contains <%= x %>
      <% end %>
    \end{verbatim}
  \end{block}
\end{frame}

\begin{frame}{ERb -- shrnutí}
  \begin{itemize}
    \item dobrý sluha, ale špatný pán
    \item můžu s tím vyrobit hromadu užitečných věcí na malém prostoru
    \item daň je velké riziko zamotaného kódu a nízké přehlednosti (struktura naprosto není patrná na první pohled, proto je namístě ji držet maximálně jednoduchou)
  \end{itemize}
\end{frame}

\begin{frame}{Důležité upozornění}
  \begin{itemize}
    \item oddělení modelu a view
    \item přestože lze provádět zpracování dat a výpočty přímo v ERb, je to nejvíc nejhorší nápad
    \item je chytré si všechno připravit v modelu (tj. v Ruby skriptu, kterým data chystáme)
    \item a kód ve view (tj. v ERb šabloně) omezit na naprosté minimum
  \end{itemize}
\end{frame}

\begin{frame}[fragile]{Jak ze šablony udělat výsledek}
  \scriptsize
  \begin{block}{Příklad překladu ERb}
    \scriptsize
    \begin{verbatim}
require "erb_compiler"

erb(template, filename, {:x => 1, :y => 2})
    \end{verbatim}
  \end{block}
  třetí parametr je hash, který nám vlastně definuje proměnné dostupné uvnitř šablony při překladu
\end{frame}

\begin{frame}[fragile]{Příklad -- kreslení grafů z minula}
  \begin{block}{template.gp}
    \scriptsize
    \begin{verbatim}
set terminal png
set output "plot_<%=n%>.png"
plot "data_<%=n%>.csv"
    \end{verbatim}
  \end{block}
  \begin{block}{}
    \scriptsize
    \begin{verbatim}
(1..10).each do |i|
  erb("template.gp", "plot_#{i}.gp", {:n => i})
end
    \end{verbatim}
  \end{block}
\end{frame}

\begin{frame}[fragile]{Takže v latexu třeba}
\scriptsize
\begin{verbatim}
  \subsection{Koncentrace kyseliny borité}

  <% cycles.each do |i| %>

  \subsubsection{Kampaň <%= i %>}

  \begin{center}
    \includegraphics[width=0.8\textwidth]{bc_<%= "%02d" % i %>_bc.eps}
  \end{center}

  <% end %>
\end{verbatim}
\end{frame}

\begin{frame}{Dva úkoly na začátek}
  \begin{enumerate}
    \item pomocí ERb vyrobte v LaTeXu malou násobilku do tabulky
    \item s jedinou šablonou vyrobte 3 dokumenty s násobilkou 7x7, 9x9, 11x11 (musíte parametrizovat)
  \end{enumerate}
\end{frame}

\begin{frame}{A teď už to jenom dejte dohromady...}
  \begin{enumerate}
    \item připravit si základní kostru dokumentu v latexu
    \item převést na šablonu: mít seznam souborů, správně generovat kapitoly
    \item vyrobit grafy
    \item vložit grafy do šablony
    \item vyrobit tabulky
    \item vložit tabulky do šablony
    \item A JE TO!
  \end{enumerate}
\end{frame}


\begin{frame}{A to je vše, přátelé!}
  \begin{center}
    \includegraphics[width=0.8\textwidth]{looney_tunes}
  \end{center}
\end{frame}

\end{document}
