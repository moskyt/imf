\documentclass{beamer}

\def\Tiny{\fontsize{6pt}{6pt}\selectfont}
\def\supertiny{\fontsize{4pt}{4pt}\selectfont}

\mode<presentation>
{
  \usetheme{Warsaw}
  % \setbeamercovered{transparent}
  \usecolortheme{crane}
}

\usepackage{graphicx, ifthen, listings, fancyvrb}

\usepackage[czech]{babel}
% \usefonttheme{professionalfonts}
\usepackage{times}
\usepackage{amsmath}
\usepackage[utf8]{inputenc}
\usepackage{wrapfig}

\usepackage[T1]{fontenc}

\lstset{ basicstyle=\tiny, stringstyle=\ttfamily, showstringspaces=false }

\everymath{\displaystyle}

\setbeamerfont{frametitle}{size=\large}
\setbeamerfont{subsection in toc}{size=\scriptsize}

\makeatletter\newenvironment{blackbox}{%
   \begin{lrbox}{\@tempboxa}\begin{minipage}{0.95\columnwidth}}{\end{minipage}\end{lrbox}%
   \colorbox{black}{\usebox{\@tempboxa}}
}\makeatother

\title[IMF (12)]{Informatika pro moderní fyziky (12)\\ procvičení práce s datovými strukturami a opakování}

\author[Franti\v{s}ek HAVL\r{U}J, ORF ÚJV Řež]{Franti\v{s}ek HAVL\r{U}J\\{\scriptsize \emph{e-mail: haf@ujv.cz}}}

\date{akademický rok 2019/2020\\11. prosince 2019}

\institute[ORF ÚJV Řež]
{ÚJV Řež\\oddělení Reaktorové fyziky a podpory palivového cyklu}

\AtBeginSection[]
{
\begin{frame}<beamer>
\frametitle{Obsah}
\tableofcontents[currentsection,hideothersubsections]
\end{frame}
}

\begin{document}

\begin{frame}
  \titlepage
\end{frame}

\begin{frame}
  \tableofcontents
\end{frame}

\section{Pole}

\begin{frame}{Co je pole}
  \begin{itemize}
    \item pole je seznam
    \item prvky jsou očíslované od 0 do N-1
    \item prvky nemusí být stejného typu
    \item prvek může být cokoliv (klidně pole nebo hash)
  \end{itemize}
\end{frame}

\begin{frame}[fragile]{Vytvoření nového pole}
  \begin{block}{}
    \begin{verbatim}
      a = []
      b = [1, 2, "xyz"]
      c = [[1,2], [3,4]]
    \end{verbatim}
  \end{block}
\end{frame}

\begin{frame}[fragile]{Čtení z pole}
  \begin{block}{Jeden prvek}
    \begin{verbatim}
      a[1]
      a[-2]
    \end{verbatim}
  \end{block}
  \begin{block}{Výřez}
    \begin{verbatim}
      a[0...3]
      a[5..-1]
    \end{verbatim}
  \end{block}
\end{frame}
%
\begin{frame}[fragile]{Zápis do pole}
  \begin{block}{Nastavení konkrétního prvku}
    \begin{verbatim}
      a[1] = "XYZ"
    \end{verbatim}
  \end{block}
  \begin{block}{Přidání nakonec}
    \begin{verbatim}
      a << 5
    \end{verbatim}
  \end{block}
  \begin{block}{Sčítání funguje i u polí}
    \begin{verbatim}
      a += [1, 2, 3]
    \end{verbatim}
  \end{block}
\end{frame}


\begin{frame}[fragile]{Iterace přes pole}
  \begin{block}{Obyčejná}
    \begin{verbatim}
      a.each do |x|
        ...
      end
    \end{verbatim}
  \end{block}
  \begin{block}{Pole dvojic (trojic, ...)}
    \begin{verbatim}
      a = [ [1,2], [3,4] ]
      a.each do |x, y|
        ...
      end
    \end{verbatim}
  \end{block}
\end{frame}

\section{Hash}

\begin{frame}{Co je hash}
  \begin{itemize}
    \item neboli asociativní pole
    \item neboli slovník (dictionary)
    \item je to také seznam, ale není číslovaný – mám klíč a k němu hodnotu
    \item klíč může být naprosto cokoliv
  \end{itemize}
\end{frame}

\begin{frame}[fragile]{Vytvoření nového hashe}
  \begin{block}{}
    \begin{verbatim}
      a = {}
      b = {"a" => 5, "b" => 111}
      c = {color: "blue", line_width: 3}
    \end{verbatim}
  \end{block}
\end{frame}

\begin{frame}[fragile]{Čtení z hashe}
  \begin{block}{Pomocí klíče}
    \begin{verbatim}
      a[1]
      a["xyz"]
    \end{verbatim}
  \end{block}
  \begin{block}{Všechny klíče/hodnoty}
    \begin{verbatim}
      a.keys
      a.values
    \end{verbatim}
  \end{block}
\end{frame}
%
\begin{frame}[fragile]{Zápis do hashe}
  \begin{block}{Obdobně jako u pole}
    \begin{verbatim}
      a[1] = "XYZ"
      b["a"] = 33
      c[:color] = "red"
    \end{verbatim}
  \end{block}
\end{frame}


\begin{frame}[fragile]{Iterace přes hash}
  \begin{block}{Podle očekávání -- vlastně je to pole dvojic!}
    \begin{verbatim}
      hash.each do |key, value|
        ...
      end
    \end{verbatim}
  \end{block}
\end{frame}

\section{Úkol: docházkový systém}

\begin{frame}{Data}
  \begin{itemize}
    \item seznam zaměstnanců, pracovišť, ID karet
    \item záznamy o průchodech přes vrátnici
    \item jména jsou unikátní
    \item ID karet jsou unikátní v rámci pracoviště (ne globálně)
  \end{itemize}
\end{frame}

\begin{frame}{Zjistěte}
  \begin{itemize}
    \item kde pracuje nejvíce lidí?
    \item kdo odpracoval nejvíc hodin?
    \item kdo pracoval nejvíc dní?
    \item která čísla ID karet jsou používána více lidmi?
    \item na kterém pracovišti je nejvíce absencí?
  \end{itemize}
\end{frame}

\begin{frame}{Do grafu vykreslete}
  \begin{itemize}
    \item počty absencí v každém dni pro každé pracoviště (osa x: dny, řady: pracoviště)
    \item odpracované hodiny v každém dni pro každé pracoviště (osa x: dny, jeden graf na pracoviště)
  \end{itemize}
\end{frame}

\begin{frame}{A to je vše, přátelé!}
  \begin{center}
    \includegraphics[width=0.8\textwidth]{looney_tunes}
  \end{center}
\end{frame}

\end{document}
