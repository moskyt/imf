\documentclass{beamer}

\def\Tiny{\fontsize{6pt}{6pt}\selectfont}
\def\supertiny{\fontsize{4pt}{4pt}\selectfont}

\mode<presentation>
{
  \usetheme{Warsaw}
  % \setbeamercovered{transparent}
  \usecolortheme{crane}
}


\usepackage{graphicx, ifthen, listings}

\usepackage[czech]{babel}
% \usefonttheme{professionalfonts}
\usepackage{times}
\usepackage{amsmath}
\usepackage[utf8]{inputenc}
\usepackage{wrapfig}

\usepackage[T1]{fontenc}

\lstset{ basicstyle=\tiny, stringstyle=\ttfamily, showstringspaces=false }

\everymath{\displaystyle}

\setbeamerfont{frametitle}{size=\large}
\setbeamerfont{subsection in toc}{size=\scriptsize}

\makeatletter\newenvironment{blackbox}{%
   \begin{lrbox}{\@tempboxa}\begin{minipage}{0.95\columnwidth}}{\end{minipage}\end{lrbox}%
   \colorbox{black}{\usebox{\@tempboxa}}
}\makeatother

\title[IMF (2)]{Informatika pro moderní fyziky (2)\\základy Ruby, zpracování textu}

\author[Franti\v{s}ek HAVL\r{U}J, ORF ÚJV Řež]{Franti\v{s}ek HAVL\r{U}J\\{\scriptsize \emph{e-mail: haf@ujv.cz}}}

\date{akademický rok 2012/2013\\11. prosince 2012}

\institute[ORF ÚJV Řež]
{ÚJV Řež\\oddělení Reaktorové fyziky a podpory palivového cyklu}

\AtBeginSection[]
{
\begin{frame}<beamer>
\frametitle{Obsah}
\tableofcontents[currentsection,hideothersubsections]
\end{frame}
}

\begin{document}

\begin{frame}
  \titlepage
\end{frame}

\begin{frame}
  \tableofcontents
\end{frame}

\section{Co jsme se naučili minule}

\begin{frame}{}
  \begin{itemize}
    \item základní principy automatizace
    \item CSV soubory a Gnuplot
    \item příkazový řádek / terminál
    \item dávkové (BAT) soubory
    \item představení skriptovacích jazyků
    \item interpret Ruby a IRb
  \end{itemize}
\end{frame}


\section{Úvod do jazyka Ruby}

\subsection{Ještě chvilku v IRb}

\begin{frame}[fragile]{OOP - volání metod}
  Klasickým příkladem je například počet znaků v řetězci.
  \begin{block}{procedurální jazyky}
    \begin{verbatim}
      strlen("retezec")
    \end{verbatim}
  \end{block}
  \pause
  Můžeme místo toho nahlížet na řetězec jako na objekt:
  \pause
  \begin{block}{objektově orientované jazyky}
    \begin{verbatim}
      "retezec".length
    \end{verbatim}
  \end{block}
\end{frame}

\begin{frame}[fragile]{Hrátky s řetězci}
  \begin{block}{Délka řetězce}
    \begin{verbatim}
      "krabice".length
      "kocour".size
    \end{verbatim}
  \end{block}
  \pause
  \begin{block}{Ořez mezer}
    \begin{verbatim}
      "   hromada ".strip
      " koleso   ".lstrip
    \end{verbatim}
  \end{block}
\end{frame}

\begin{frame}[fragile]{Hrátky s řetězci}
  \begin{block}{Hledání}
    \begin{verbatim}
      "koleno na kole".include?("kole")
      "koleno na kole".count("kole")
    \end{verbatim}
  \end{block}
  \pause
  \begin{block}{Nahrazení}
    \begin{verbatim}
      "volej kolej".sub("olej", "yber")
      "baba a deda".gsub("ba", "ta")
    \end{verbatim}
  \end{block}
\end{frame}

\begin{frame}{Dokumentace}
  \begin{block}{Google is your friend}
    \texttt{ruby api string}
  \end{block}
  \begin{block}{API dokumentace}
    \texttt{http://www.ruby-doc.org/core-1.9.3/String.html}
  \end{block}
\end{frame}

\subsection{Pole}

\begin{frame}[fragile]{Ztracen v poli}
  \begin{block}{Literál, přiřazení}
    \begin{verbatim}
      a = []
      a << 1
      a << "string"
      b = []
    \end{verbatim}
  \end{block}
  \pause
  \begin{block}{Délka, řazení, vypletí, převracení}
    \begin{verbatim}
      [4, 2, 6].sort
      [2, 5, 3, 3, 4, 1, 2, 1].uniq.sort
      [4, 2, 6].reverse
    \end{verbatim}
  \end{block}
\end{frame}

\begin{frame}[fragile]{Ztracen v poli}
  \begin{block}{Indexace}
    \begin{verbatim}
      a = [1,2,3]
      a[1]
      a[3]
    \end{verbatim}
  \end{block}
  \pause
  \begin{block}{Do mínusu, odkud kam}
    \begin{verbatim}
      a = [1,2,3,4,5,6]
      a[-1]
      a[-2]
      a[0..3]
    \end{verbatim}
  \end{block}
\end{frame}

\begin{frame}[fragile]{Pole z řetězů}
  \begin{block}{Řetězec, pole znaků}
    \begin{verbatim}
      "kopr"[2]
      "mikroskop"[0..4]
    \end{verbatim}
  \end{block}
  \pause
  \begin{block}{Leccos funguje!}
    \begin{verbatim}
      "abcd".reverse
      [1,2,3].size
    \end{verbatim}
  \end{block}
  \begin{block}{Sekáček na maso}
    \begin{verbatim}
      "a b  c d".split
      "a b,c d".split(",")
    \end{verbatim}
  \end{block}
\end{frame}

\begin{frame}[fragile]{Operátor a operatér}
  \begin{block}{Malé bezvýznamné plus}
    \begin{verbatim}
      "alfa" + "beta"
      [1, 2] + [3, 4]
    \end{verbatim}
  \end{block}
  \pause
  \begin{block}{Násobilka}
    \begin{verbatim}
      "kolo" * 5
      [1, 2, 3] * 3
      ["a", "b", "c"] * ","
    \end{verbatim}
  \end{block}
\end{frame}

\begin{frame}[fragile]{}
  \begin{block}{Převádět přes ulici}
    \begin{verbatim}
      "123".to_i
      1250.to_s
      "0.6".to_f
    \end{verbatim}
  \end{block}
  \pause
  \begin{block}{Hash / slovník}
    \begin{verbatim}
      a = {}
      a["xyz"] = 555
      b = {'a' => 4, 'c' => 6}
    \end{verbatim}
  \end{block}
\end{frame}

\begin{frame}[fragile]{}
  \begin{block}{Vocaď pocaď}
    \begin{verbatim}
      (1..4)
      (0...10)
      (1..5).to_a
    \end{verbatim}
  \end{block}
  \pause
  \begin{block}{Symbolika}
    \begin{verbatim}
      "letadlo"
      :letadlo
    \end{verbatim}
  \end{block}
\end{frame}

\begin{frame}[fragile]{Úlohy}
  \begin{block}{Konverze II}
    \begin{itemize}
      \item vyzkoumejte, jak se chová \texttt{to\_f} a \texttt{to\_i} pro řetězce, které nejsou tak úplně číslo
    \end{itemize}
  \end{block}
  \pause
  \begin{block}{Palindrom}
    \begin{itemize}
      \item z libovolného řetězce vyrobte palindrom (osel $\to$ oselleso)
      \item z libovolného řetězce vyrobte palindrom s lichým počtem znaků (osel $\to$ oseleso)
    \end{itemize}
  \end{block}
\end{frame}

\subsection{Vstup a výstup}

\begin{frame}[fragile]{Výpis z účtu}
  \begin{block}{Tiskem}
    \begin{verbatim}
      print "jedna"
      puts "dve"
    \end{verbatim}
  \end{block}
  \pause
  \begin{block}{Inspektor Clouseau}
    \begin{verbatim}
      puts "2 + 2 = #{2+2}"
      puts [1,2,3].inspect
    \end{verbatim}
  \end{block}
\end{frame}

\begin{frame}[fragile]{Cyklistika}
  \begin{block}{Jednoduchý rozsah}
    \begin{verbatim}
      (1..5).each do
        puts "Cislo"
      end
    \end{verbatim}
  \end{block}
  \pause
  \begin{block}{S polem a proměnnou}
    \begin{verbatim}
      [1, 2, 3].each do |i|
        puts "Cislo #{i}"
      end
    \end{verbatim}
  \end{block}
\end{frame}

\begin{frame}{Úlohy}
  \begin{itemize}
    \item vypište prvních deset druhých mocnin (1 * 1 = 1, 2 * 2 = 4 atd.)
    \item vypište malou násobilku
    \item vypište prvních N členů Fibonacciho posloupnosti (1, 1, 2, 3, 5, 8 ...)
    \item metodou Erathostenova síta nalezněte prvočísla menší než N
  \end{itemize}
\end{frame}

\begin{frame}[fragile]{Česko čte dětem}
  \begin{block}{Šikovný iterátor}
    \begin{verbatim}
      IO.foreach("data.txt") do |line|
        ...
      end
    \end{verbatim}
  \end{block}
  \pause
  \begin{block}{V kuse}
    \begin{verbatim}
      string = IO.read("data.txt")
    \end{verbatim}
  \end{block}
\end{frame}

\begin{frame}{Úlohy}
  V souboru \texttt{data/text\_1.txt}:
  \begin{itemize}
    \item spočítejte všechny řádky
    \item spočítejte všechny řádky s výskytem slova kapr
    \item spočítejte počet výskytů slova kapr (po řádcích i v kuse)
  \end{itemize}
\end{frame}

\begin{frame}[fragile]{Zápis do katastru}
  \begin{block}{Soubor se otevře a pak už to známe}
    \begin{verbatim}
      f = File.open("text.txt", 'w')
      f.puts "Nazdar!"
      f.close
    \end{verbatim}
  \end{block}
  \pause
  \begin{block}{The Ruby way}
    \begin{verbatim}
      File.open("text.txt", 'w') do |f|
        f.puts "Nazdar!"
      end
    \end{verbatim}
  \end{block}
\end{frame}

\begin{frame}{Úlohy}
  Z dat v souboru \texttt{data/data\_two\_1.csv}:
  \begin{itemize}
    \item vyberte pouze druhý sloupec
    \item sečtěte oba sloupce do jednoho
    \item vypočtěte součet obou sloupců
    \item vypočtěte průměr a RMS druhého sloupce
  \end{itemize}
  S hvězdičkou: 
  \begin{itemize}
    \item použijte soubory \texttt{*multi*}
    \item proveďte pro všechny čtyři CSV soubory 
  \end{itemize}
\end{frame}

\section{Zpracování textu}

\subsection{Obecný rozbor}

\begin{frame}{Problém č. 2: mnoho výpočtů, inženýrova smrt}
  \begin{block}{Zadání}
    Při přípravě základního kritického experimentu je pomocí MCNP potřeba najít kritickou polohu regulační tyče R2.

    Jak se tato poloha změní při změně polohy tyče R1?
  \end{block}
\end{frame}

\begin{frame}{Co máme k~dispozici?}
  \begin{block}{MCNP}
    Pokud připravíme vstupní soubor (v~netriviální formě obsahující polohy regulačních tyčí R1 a R2), spočítá nám keff.
  \end{block}
  Potřebovali bychom ale něco na:
  \begin{enumerate}
    \item vytvoření velkého množství vstupních souborů
    \item extrakci keff z~výstupních souborů
    \item popřípadě na vyhodnocení získaných poloh tyčí a keff
  \end{enumerate}
\end{frame}

\begin{frame}{Pracovní postup}
  \begin{enumerate}
    \item načíst keff z~výstupního souboru MCNP
    \pause
    \item vygenerovat potřebné vstupní soubory
    \pause
    \item vyrobit BAT soubor na spuštění výpočtů
    \pause
    \item načíst výsledky ze všech výstupních souborů do jedné tabulky
    \pause
    \item buď zpracovat ručně (Excel), nebo být Myšpulín a vyrobit skript (úkol s~hvězdičkou)
  \end{enumerate}
\end{frame}

\subsection{Načítání výstupního souboru}

\begin{frame}[fragile]{}
  Nejprve najdeme, kde je ve výstupu z~MCNP žádané keff:
  {\tiny
  \begin{verbatim}
.....
         the k(trk length) cycle values appear normally distributed at the 95 percent confidence level

-----------------------------------------------------------------------------------------------------------------------------------
|                                                                                                                                 |
| the final estimated combined collision/absorption/track-length keff = 1.00353 with an estimated standard deviation of 0.00024   |
|                                                                                                                                 |
| the estimated 68, 95, & 99 percent keff confidence intervals are 1.00329 to 1.00377, 1.00305 to 1.00400, and 1.00289 to 1.00416 |
|                                                                                                                                 |
| the final combined (col/abs/tl) prompt removal lifetime = 1.0017E-04 seconds with an estimated standard deviation of 2.7364E-08 |
.....
  \end{verbatim}
  }
\end{frame}

\begin{frame}{Algoritmus}
  \begin{enumerate}
    \item najít řádek s~keff
    \pause
    \item vytáhnout z~něj keff, takže například:
    \pause
    \item rozdělit podle rovnítka
    \pause
    \item druhou část rozdělit podle mezer
    \pause
    \item vzít první prvek
  \end{enumerate}
\end{frame}


\begin{frame}[fragile]{Realizace (1/5)}
  \scriptsize
  \begin{verbatim}
    keff = nil

    IO.foreach("c1_1o") do |line|

    end

    puts keff
  \end{verbatim}
\end{frame}

\begin{frame}[fragile]{Realizace (2/5)}
  \scriptsize
  \begin{verbatim}
    keff = nil

    IO.foreach("c1_1o") do |line|
      if line.include?("final estimated combined")

      end
    end

    puts keff
  \end{verbatim}
\end{frame}

\begin{frame}[fragile]{Realizace (3/5)}
  \scriptsize
  \begin{verbatim}
    keff = nil

    IO.foreach("c1_1o") do |line|
      if line.include?("final estimated combined")
        a = line.split("=")

      end
    end

    puts keff
  \end{verbatim}
\end{frame}

\begin{frame}[fragile]{Realizace (4/5)}
  \scriptsize
  \begin{verbatim}
    keff = nil

    IO.foreach("c1_1o") do |line|
      if line.include?("final estimated combined")
        a = line.split("=")
        b = a[1].split

      end
    end

    puts keff
  \end{verbatim}
\end{frame}

\begin{frame}[fragile]{Realizace (5/5)}
  \scriptsize
  \begin{verbatim}
    keff = nil

    IO.foreach("c1_1o") do |line|
      if line.include?("final estimated combined")
        a = line.split("=")
        b = a[1].split
        keff = b[0]
      end
    end

    puts keff
  \end{verbatim}
\end{frame}


\subsection{Sestavení vstupního souboru}

\begin{frame}[fragile]{Určení poloh tyčí}
  Ve vstupním souboru si najdeme relevantní část:
  \scriptsize
  \begin{verbatim}
    c ---------------------------------
    c polohy tyci (z-plochy)
    c ---------------------------------
    c
    67 pz 47.6000    $ dolni hranice absoberu r1
    68 pz 40.4980    $ dolni hranice hlavice r1
    69 pz 44.8000    $ dolni hranice absoberu r2
    70 pz 37.6980    $ dolni hranice hlavice r2
  \end{verbatim}
\end{frame}

\begin{frame}[fragile]{Výroba šablon}
  Jak dostat polohy tyčí do vstupního souboru? Vyrobíme šablonu, tzn nahradíme
  \begin{verbatim}
    67 pz 47.6000    $ dolni hranice absoberu r1
  \end{verbatim}
  \pause
  nějakou značkou (\emph{placeholder}):
  \begin{verbatim}
    67 pz %r1%    $ dolni hranice absoberu r1
  \end{verbatim}
\end{frame}

\begin{frame}{Chytáky a zádrhele}
  \begin{itemize}
    \item kromě samotné plochy konce absorbéru je nutno správně umístit i z-plochu konce hlavice o 7,102 cm níže
    \item obecně je na místě ohlídat si, že placeholder nebude kolidovat s ničím jiným
  \end{itemize}
  Doporučené nástroje jsou:
  \begin{itemize}
    \item již známá funkce \texttt{sub} pro nahrazení jednoho řetězce jiným
    \item pro pragmatické lenochy funkce \texttt{IO.read} načítající celý soubor do řetězce (na což nelze v Pascalu ani pomyslet)
    \item možno ovšem použít i \texttt{IO.readlines} (v čem je to lepší?)
  \end{itemize}
\end{frame}

\begin{frame}[fragile]{Realizace}
  \scriptsize
  \begin{verbatim}
    DELTA = 44.8000 - 37.6980

    template = IO.read('template')
    (0..10).each do |i1|
      (0..10).each do |i2|
        r1 = i1 * 50
        r2 = i2 * 50
        File.open("inputs/c_#{i1}_#{i2}", "w") do |f|
          s = template.sub("%r1%", r1.to_s)
          s = s.sub("%r1_%", (r1 - DELTA).to_s)
          s = s.sub("%r2%", r2.to_s)
          s = s.sub("%r2_%", (r2 - DELTA).to_s)
          f.puts template
        end
      end
    end
  \end{verbatim}
\end{frame}

\subsection{Zápis všech výsledků do tabulky}

\begin{frame}{Jak na to}
  Máme všechno, co potřebujeme:
  \begin{itemize}
    \item načtení keff z~jednoho výstupního souboru (\texttt{IO.foreach}, \texttt{include} a \texttt{split})
    \item procházení adresáře (\texttt{Dir.each})
    \item zápis do souboru (\texttt{File.open} s~parametrem \texttt{w})
  \end{itemize}
  Takže už to stačí jen vhodným způsobem spojit dohromady!
\end{frame}

\begin{frame}[fragile]{Realizace}
  \scriptsize
  \begin{verbatim}
    Dir["*o"].each do |filename|
      keff = nil

      IO.foreach(filename) do |line|
        if line.include?("final estimated combined")
          a = line.split("=")
          b = a[1].split
          keff = b[0]
        end
      end

      puts "#{filename} #{keff}"
    end
  \end{verbatim}
\end{frame}

\begin{frame}[fragile]{Výstup}
  Výsledkem je perfektní tabulka:
  {\scriptsize
  \begin{verbatim}
    outputs/c_0_0o 0.94800
    outputs/c_0_10o 0.99800
    outputs/c_0_1o 0.94850
    outputs/c_0_2o 0.95000
    outputs/c_0_3o 0.95250
    outputs/c_0_4o 0.95600
    ...
  \end{verbatim}}
  Hloupé je, že nikde nemáme tu polohu tyčí.
\end{frame}


\begin{frame}[fragile]{Chytrá horákyně}
  ... by jistě vyrobila toto:
  {\scriptsize
  \begin{verbatim}
    0 0 0.94800
    0 10 0.99800
    0 1 0.94850
    0 2 0.95000
    0 3 0.95250
    0 4 0.95600
    ...
  \end{verbatim}
  }
  Nápovědou je funkce \texttt{split} (podle podtržítka) a funkce \texttt{to\_i} (co asi dělá?)
\end{frame}

\begin{frame}[fragile]{Realizace chytré horákyně}
  \scriptsize
  \begin{verbatim}
    Dir["outputs/*o"].each do |filename|
      keff = nil

      IO.foreach(filename) do |line|
        if line.include?("final estimated combined")
          a = line.split("=")
          b = a[1].split
          keff = b[0]
        end
      end

      s = filename.split("_")
      puts "#{s[1].to_i} #{s[2].to_i} #{keff}"
    end
  \end{verbatim}
\end{frame}

\begin{frame}{A to je vše, přátelé!}
  \begin{center}
    \includegraphics[width=0.8\textwidth]{looney_tunes}
  \end{center}
\end{frame}

\end{document}
