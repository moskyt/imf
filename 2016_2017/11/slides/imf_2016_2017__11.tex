\documentclass{beamer}

\def\Tiny{\fontsize{6pt}{6pt}\selectfont}
\def\supertiny{\fontsize{4pt}{4pt}\selectfont}

\mode<presentation>
{
  \usetheme{Warsaw}
  % \setbeamercovered{transparent}
  \usecolortheme{crane}
}

\usepackage{graphicx, ifthen, listings, fancyvrb}

\usepackage[czech]{babel}
% \usefonttheme{professionalfonts}
\usepackage{times}
\usepackage{amsmath}
\usepackage[utf8]{inputenc}
\usepackage{wrapfig}

\usepackage[T1]{fontenc}

\lstset{ basicstyle=\tiny, stringstyle=\ttfamily, showstringspaces=false }

\everymath{\displaystyle}

\setbeamerfont{frametitle}{size=\large}
\setbeamerfont{subsection in toc}{size=\scriptsize}

\makeatletter\newenvironment{blackbox}{%
   \begin{lrbox}{\@tempboxa}\begin{minipage}{0.95\columnwidth}}{\end{minipage}\end{lrbox}%
   \colorbox{black}{\usebox{\@tempboxa}}
}\makeatother

\title[IMF (8)]{Informatika pro moderní fyziky (8)\\ HTML a JS - interaktivní dokumenty}

\author[Franti\v{s}ek HAVL\r{U}J, ORF ÚJV Řež]{Franti\v{s}ek HAVL\r{U}J\\{\scriptsize \emph{e-mail: haf@ujv.cz}}}

\date{akademický rok 2015/2016\\1. prosince 2015}

\institute[ORF ÚJV Řež]
{ÚJV Řež\\oddělení Reaktorové fyziky a podpory palivového cyklu}

\AtBeginSection[]
{
\begin{frame}<beamer>
\frametitle{Obsah}
\tableofcontents[currentsection,hideothersubsections]
\end{frame}
}

\begin{document}

\begin{frame}
  \titlepage
\end{frame}

\begin{frame}
  \tableofcontents
\end{frame}

\section{Co jsme se naučili minule}

\begin{frame}{}
  \begin{itemize}
    \item použití cizích API – data v JSON, mashup s GoogleMaps Static API
    \item základy HTML + ERb
  \end{itemize}
\end{frame}

\section{Něco o kódování}

\begin{frame}{Kódování}
  \begin{itemize}
    \item znaky v počítači: 1 znak = 1 byte = 256 možností
    \item cca polovina je ``normální text'', zbytek jsou tak trochu speciální znaky
    \item jsou tam ale jen `západní' znaky, na středoevropské se nedostalo
    \item co teprv azbuky, japonština, čínština, ...
  \end{itemize}
\end{frame}

\begin{frame}{Kódování}
  \begin{itemize}
    \item varianta 1: nahrazovat druhou polovinu znaků tím, co zrovna potřebuju (ěščř...)
    \item výhoda: nezabírá místo, pořád platí znak=byte, jednoduché řešení
    \item nevýhoda: pro různé jazyky různá kódování
    \item další nevýhoda: na leckterý jazyk (neevropský) to naprosto nestačí
  \end{itemize}
\end{frame}

\begin{frame}{Kódování}
  \begin{itemize}
    \item varianta 2: přejít na reprezentaci jednoho znaku více byty
    \item výhoda: podstatně se rozšíří počet znaků, takže není nutné `přepínat'
    \item nevýhoda: zvětšuje velikost textu
    \item další nevýhoda: míra rozsypanosti čaje při špatné interpretaci se zvyšuje (je možné způsobit i nečitelnost `obyčejných' znaků)
  \end{itemize}
\end{frame}

\begin{frame}{Kódování - obecné problémy}
  \begin{itemize}
    \item soubory neobsahují informaci o tom, v jakém kódování jsou, takže je nutné dodávat metadata
    \item manipulace s kódováním je otravná, obtížná a snadno způsobí problémy
    \item je nutné synchronizovat/řešit kódování na všech vrstvách aplikace (soubory / databáze / databázový klient / aplikace / server / klient)
    \item ... pokud existuje 100 různých variant, řeší se to obvykle tím, že se vymyslí nějaká 101. ...
  \end{itemize}
\end{frame}

\begin{frame}{Kódování}
  \begin{itemize}
    \item naštěstí to (minimálně v euroamerickém prostředí) už dneska nemusíme řešit, existuje jednotný standard -- Unicode (UTF-8)
    \item starší systémy občas poskytují data v jiných kódováních, je dobré vědět, že pro češtinu se můžete setkat s \texttt{Windows/CP1250} a \texttt{ISO-9981-2}
    \item pokud budete chtít pracovat s exotickými jazyky (Afrika, Oceánie), budete asi muset využít UTF-16/32 (naneštěstí opět více variant)
    \item z historických důvodů je nejednotnost také pro japonská kódování
    \item to hlavní: používejte UTF-8. Tečka. Nezapomeňte: v editoru, v hlavičce HTML, v hlavičce LaTeXu, všude...
  \end{itemize}
\end{frame}

\section{JavaScript}

\begin{frame}{JavaScript}
  \begin{itemize}
    \item trochu obskurní jazyk bez ladu a skladu
    \item dá se vkládat do HTML a používá se pro client-side aplikace
    \item dříve to bylo trochu zlo, dneska se z toho stala
    \item rozšířenost dalece převyšuje kvalitu
  \end{itemize}
\end{frame}

\begin{frame}{JS frameworky}
  \begin{itemize}
    \item vrstva ``překrývající'' JS a usnadňující základní věci, často zejména manipulaci se stránkou a
    \item bez nich je JS stěží využitelný
    \item aktuální standard je jQuery
    \item množství dalších knihoven a rozšíření
  \end{itemize}
\end{frame}

\begin{frame}{JavaScript v HTML}
  \begin{itemize}
    \item event-driven jazyk
    \item kód se nespouští ``jen tak'', ale pouze na základě události
    \item kliknutí, načtení stránky, uplynutí času, klávesnice ...
    \item pro nás nejužitečnější \texttt{onclick}; např. <a href=''\#'' onclick=''...'' >...</a>
  \end{itemize}
\end{frame}

\begin{frame}{Schovávání a zobrazování}
  \begin{itemize}
    \item každému tagu můžu přiřadit id <img src=''..'' id=''image1'' >
    \item v jQuery se potom na příslušný tag odkážu takto: \texttt{\$('\#image1')}
    \item \texttt{\$('\#image1').hide();}
    \item \texttt{<a href=''\#'' onclick=''\$('\#image1').hide();''>}
    \item navíc: je potřeba přidat \texttt{return false} na konec handleru
    \item obdobně \texttt{show} a \texttt{toggle}
    \item kousek CSS: \\ \texttt{<img src=''...'' style=''display:none'' >}
  \end{itemize}
\end{frame}

\begin{frame}{JavaScript - jak jQuery nahrát}
  \begin{itemize}
    \item je možné se v dokumentu odkázat na jiné JS soubory
    \item obvykle je to lepší než tam text dokumentu kopírovat... (přehlednost, caching)
    \item odkaz se vkládá do hlavičky dokumentu (\texttt{<head>})
    \item \texttt{<script src=''jquery-1.4.4.min.js'' type=''text/javascript'' ></script>}
  \end{itemize}
\end{frame}

\begin{frame}{Úkoly}
  \begin{itemize}
    \item doplnit přepínání mezi grafem a tabulkou
    \item vylepšit z programátorského hlediska (DRY princip a AO/boritá)
  \end{itemize}
\end{frame}

\section{Stylování dokumentů}

\begin{frame}{CSS}
  \begin{itemize}
    \item jazyk pro popis vzhledu HTML dokumentu
    \item v nejjednodušším přístupu definuje styly pro jednotlivé typy elementů
    \item dále umožňuje definovat tzv. třídy (skupiny elementů, pomocí atributu \texttt{class} v HTML) a také styly pro konkrétní elementy (id)
    \item složitější selekory -- vnoření, souslednost atd.
  \end{itemize}
\end{frame}

\begin{frame}[fragile]{CSS - jednoduchý příklad}
  \begin{itemize}
    \item základní selektory: \emph{element}, \emph{.třída} a \emph{\#id}
    \item základní syntaxe \emph{vlastnost: hodnota;}
  \end{itemize}
  \scriptsize
  \begin{verbatim}
    a {
      color: blue;
      text-decoration: none;
      font-weight: bold;
    }
    #core_map { float:left; }
  \end{verbatim}
\end{frame}

\begin{frame}{CSS - kam s ním}
  \begin{itemize}
    \item přímo k tagu (\texttt{<div style=``display:none''>}) -- možná dobré na rychlé ladění/patlání, ale vesměs vždy špatně; jedinou výjimkou je \texttt{display:none} pro elementy, které mají být vidět až později, tam to jinak nejde
    \item v hlavičce (\texttt{head}) HTML dokumentu: \texttt{<style type=``text/css''>...</style>}
    \item v externím souboru (obvykle jediné správné řešení) -- pomocí tagu \texttt{link} v hlavičce
    \item my vystačíme se \texttt{style} tagem v hlavičcce
  \end{itemize}
\end{frame}

\begin{frame}{Rychlé procvičení}
  \begin{itemize}
    \item vezměte si svoje krásné HTML a doplňte do něj trochu toho stylování
    \item minimum: změnit font (\texttt{font-family}), nastavit rozumné velikosti písma a barvy
    \item zrušit podtrhávání odkazů (\texttt{text-decoration})
    \item bonus: jiná barva odkazů na přepínání tabulka/graf
  \end{itemize}
\end{frame}


\section{Všechno dohromady}

\begin{frame}{Zadání dnešní úlohy}
  \begin{itemize}
    \item každý den data z 1-9 detektorů (\texttt{data/*.csv})
    \item detektor má svoji polohu v AZ (VR-1 Vrabec, 8x8 čtvercových pozic) -- včetně data je uvedena na prvním řádku CSV souboru
    \item je potřeba hezky zobrazit na každý den mapu AZ a grafy signálů z detektorů
    \item viz \texttt{html/document.html}
  \end{itemize}
\end{frame}

\begin{frame}{Jak vložit do HTML}
  \begin{itemize}
    \item jsou různé metody, jak vložit ze souboru
    \item my se bez toho v pohodě obejdeme - vložíme přímo (\texttt{File.read})
    \item v tu chvíli je totiž mj. možné naplácat na SVG objekty javascriptové handlery
    \item ... tedy na čtverečku můžu mít onclick
  \end{itemize}
\end{frame}

\begin{frame}{Zpracovat data}
  \begin{itemize}
    \item pro každý CSV soubor chceme mít graf (gnuplot/png)
    \item pro každý den mapu (nejdřív obyčejnou, potom klikací, pak třeba s textem)
  \end{itemize}
\end{frame}

\begin{frame}{Další JS chytrosti}
  \begin{itemize}
    \item v jQuery už známe \texttt{\$(`\#id')}
    \item ale ve skutečnosti jde použít jakýkoli CSS selektor, takže třeba \texttt{\$(`p')}
    \item pokročilý CSS selektor -- vnoření: \texttt{\#my\_list img} vybere všechny obrázky (\texttt{img}) které jsou uvnitř elementu s id \texttt{my\_list}
    \item ... použiju v situacích, kdy chci schovat nějakou množinu elementů a pak jeden z nich zobrazit (tj. když mám hromadu obrázků a chci, aby byl vidět jen jeden)
  \end{itemize}
\end{frame}

\begin{frame}{Další CSS chytrosti}
  \begin{itemize}
    \item normálně se jednotlivé elementy řadí pod sebe
    \item můžu místo toho použít tzv. floating, kdy se začnou elementy řadit nalevo nebo napravo
    \item efekt znáte např. z webových galerií, kde mi fotky vyplní celou šířku okna a jdou po řádcích
    \item \texttt{float:left}
    \item barvu pozadí nastavím např. \texttt{background-color:red} nebo \texttt{background-color:\#ddddff}
  \end{itemize}
\end{frame}

\begin{frame}{Další SVG chytrosti}
  \begin{itemize}
    \item kromě \texttt{rect} se bude hodit také \texttt{text}
    \item jako text se zobrazí obsah příslušného elementu
    \item opět použiju atributy \texttt{x}, \texttt{y} (levý dolní roh) a můžu přihodit \texttt{text-anchor=``middle''}, aby to byl dolní prostředek
    \item pozor, text mi překryje čtvereček, takže budu muset zopakovat onclick!
  \end{itemize}
\end{frame}


\begin{frame}{A to je vše, přátelé!}
  \begin{center}
    \includegraphics[width=0.8\textwidth]{looney_tunes}
  \end{center}
\end{frame}

\end{document}
