\documentclass{beamer}

\def\Tiny{\fontsize{6pt}{6pt}\selectfont}
\def\supertiny{\fontsize{4pt}{4pt}\selectfont}

\mode<presentation>
{
  \usetheme{Warsaw}
  % \setbeamercovered{transparent}
  \usecolortheme{crane}
}


\usepackage{graphicx, ifthen, listings}

\usepackage[czech]{babel}
% \usefonttheme{professionalfonts}
\usepackage{times}
\usepackage{amsmath}
\usepackage[utf8]{inputenc}
\usepackage{wrapfig}

\usepackage[T1]{fontenc}

\lstset{ basicstyle=\tiny, stringstyle=\ttfamily, showstringspaces=false }

\everymath{\displaystyle}

\setbeamerfont{frametitle}{size=\large}
\setbeamerfont{subsection in toc}{size=\scriptsize}

\makeatletter\newenvironment{blackbox}{%
   \begin{lrbox}{\@tempboxa}\begin{minipage}{0.95\columnwidth}}{\end{minipage}\end{lrbox}%
   \colorbox{black}{\usebox{\@tempboxa}}
}\makeatother

\title[IMF (3)]{Informatika pro moderní fyziky (3)\\rozšíření RubyGems, ladění programů a zpracování textu}

\author[Franti\v{s}ek HAVL\r{U}J, ORF ÚJV Řež]{Franti\v{s}ek HAVL\r{U}J\\{\scriptsize \emph{e-mail: haf@ujv.cz}}}

\date{akademický rok 2016/2017, 19. října 2016}

\institute[ORF ÚJV Řež]
{ÚJV Řež\\oddělení Reaktorové fyziky a podpory palivového cyklu}

\AtBeginSection[]
{
\begin{frame}<beamer>
\frametitle{Obsah}
\tableofcontents[currentsection,hideothersubsections]
\end{frame}
}

\begin{document}

\begin{frame}
  \titlepage
\end{frame}

\begin{frame}
  \tableofcontents
\end{frame}

\section{Co jsme se naučili minule}

\begin{frame}{}
  \begin{itemize}
    \item základy jazyka Ruby na všechny způsoby
    \item vstup a výstup na terminál i do souboru
    \item první kroky ve zpracování dat (čtení ze souboru, zpracování CSV tabulky - ještě musíme dodělat)
  \end{itemize}
\end{frame}
\section{Rozšíření Ruby: RubyGems a Bundler}

\begin{frame}{Knihovny (gemy) jsou základ}
  \begin{itemize}
    \item existují mnohá rozšíření, tzv. knihovny -- v ruby se jim říká rubygems
    \item aktuálně nás zajímá něco na práci s excelovskými soubory
    \item gemy jdou sice instalovat na systémové úrovni, ale z toho je pak zase jenom neštěstí
    \item použijeme radši \textbf{bundler}, správce gemů pro každého: vyřeší za nás závislosti a postará se o snadnou instalaci
  \end{itemize}
\end{frame}

\begin{frame}{Máme bundler?}
  \begin{itemize}
    \item otestujeme rubygems: \texttt{gem -v}
    \item pokud není, zapláčeme, protože jsme asi špatně nainstalovali Ruby
    \item otestume bundler: \texttt{bundle -v}
    \item pokud bundler není, doinstalujeme \texttt{gem install bundler}
  \end{itemize}
\end{frame}

\begin{frame}{Jak na to}
  \begin{itemize}
    \item najdu si, která knihovna mě zajímá (třeba na rubygems.org nebo kdekoli jinde): my bychom rádi rubyXL \texttt{https://github.com/weshatheleopard/rubyXL}
    \item vytvořím si prázdný \texttt{Gemfile} -- tam se specifikuje, které gemy chci používat: \texttt{bundle init}
    \item do gemfilu -- je to normální Ruby skript! -- dopíšu \texttt{gem "rubyXL"}
    \item nainstaluju: \texttt{bundle install --path vendor/bundle} (ten parametr stačí poprvé, bundler si to pak pamatuje v konfiguráku \texttt{bundle/.config})
  \end{itemize}
\end{frame}

\begin{frame}{Jak použít?}
  \begin{itemize}
    \item na začátku svého skriptu pak musím nahrát bundler:
    \item \texttt{require "bundler/setup"}
    \item a teď už můžu nahrát jakýkoli gem:
    \item \texttt{require "rubyXL"}
  \end{itemize}
\end{frame}

\subsection{Vytvoříme excelovskou tabulku}

\begin{frame}[fragile]{RTFM, RTFM, RTFM}
  \begin{itemize}
    \item na stránkách \texttt{rubyXL} se nachází spousta příkladů a návodů -- https://github.com/weshatheleopard/rubyXL
    \item kromě toho má i slušnou dokumentaci (GIYF / "rubyxl docs") -- http://www.rubydoc.info/gems/rubyXL/3.3.15
    \item napoprvé navedu do začátku:
  \end{itemize}
  {\tiny
  \begin{verbatim}
    workbook = RubyXL::Workbook.new
    worksheet = workbook[0]
    worksheet.add_cell(0, 0, 'A1')
    workbook.write("data.xlsx")
  \end{verbatim}
  }
\end{frame}

\begin{frame}{Jednoduché cvičení}
  \begin{itemize}
    \item použijte soubor \texttt{data\_two\_1.csv}
    \item vytvořte excelovský soubor se dvěma listy, na obou bude sloupec 1, sloupec 2 a součet
    \item na jednom součet bude jako číslo (sečte to váš skript)
    \item na druhém bude součet jako excelovský vzorec
  \end{itemize}
\end{frame}


\section{Kde je chyba?}

\begin{frame}{Ladění programů}
  \begin{itemize}
    \item v každém programu je aspoň jedna chyba
    \item není důležité nedělat chyby, ale je nutné je umět najít
    \item když si program/Ruby na něco stěžuje, tak si to přečtěte, jinak se nic nedozvíte
    \item pokud nepoznám, v čem je chyba, jsem bezezbytku ztracen
    \item následují tři úlohy, kde je úkolem najít všechny chyby
    \item z didaktických důvodů postupujte metodou tupého spouštění a postupného opravování
  \end{itemize}
\end{frame}

\begin{frame}[fragile]{Úloha 1}
    \scriptsize
\begin{verbatim}
  # vypsat nasobilku (na kazdem radku deset soucinu)

  (1..10),each do |i|
    (1..10).each do |i|
      puts "#{i} * #{j} = #{i*j} "
    end
    puts
  end
\end{verbatim}
\end{frame}

\begin{frame}[fragile]{Úloha 2}
    \scriptsize
\begin{verbatim}
  # vypsat soucet vsech cisel z prvniho sloupce souboru "data.csv"

  File.foreach("data.csv") do |row|
    ary = line.strip.split
    y = ary[1]
    x += y
  end
  puts x
\end{verbatim}
\end{frame}

\begin{frame}[fragile]{Úloha 3}
    \scriptsize
\begin{verbatim}
  # v souboru "data.txt" najde nejmensi a nejvetsi cislo

  File.read_lines("data.txt").each do |line|
    ary = line.strip.split
    ary.each do |x|
      if x > max
        max = x
      end
      if x < min
        min = x
      end
    end
  end
  puts "Minimum = #{min}"
  puts "Maximum = #{max}"
\end{verbatim}
\end{frame}

\section{Dodělávka z minula: jehla v kupce sena}

\begin{frame}{Zadání}
  \begin{block}{\# 1}
    Adresář plný CSV souborů (stovky souborů) obsahuje data, která jsou záznamy signálů s lineární závislostí.
    
    V pěti z nich jsou ale poruchy - data ležící zcela mimo přímku.
    
    Kde?
  \end{block}
\end{frame}

\begin{frame}{Příklad - dobrý signál}
  \begin{center}
      \includegraphics[width=0.6\columnwidth]{search_good}
      \end{center}
\end{frame}
\begin{frame}{Příklad - špatný signál}
  \begin{center}
      \includegraphics[width=0.6\columnwidth]{search_bad}
      \end{center}
\end{frame}

\begin{frame}{Řešení}
  \begin{itemize}
    \item stačí vykreslit grafy pro všechny
    \item \texttt{Dir} pro najití souborů
    \item připravit a spustit \texttt{gnuplot}  
    \item kouknu a vidím 
  \end{itemize}
\end{frame}

\begin{frame}{Řešení 2}
  \begin{itemize}
    \item trocha matematiky po inženýrsku
    \item odečtu vhodnou lineární funkci
    \item podívám se na rozdíl mezi minimem a maximem
  \end{itemize}
  nebo
  \begin{itemize}
    \item sleduju rozdíl dvou po sobě jdoucích hodnot 
    \item pokud se mi sgn(dx) změní, tak je jasno
  \end{itemize}
\end{frame}

%
% \begin{frame}[fragile]{}
%   {\tiny
%   \begin{verbatim}
%     Dir["../package/search/*.csv"].each do |fn|
%       xx, yy = [], []
%       File.foreach(fn) do |line|
%         ary = line.strip.split.map(&:to_f)
%         xx << ary[0]
%         yy << ary[1]
%       end
%       ...
%   \end{verbatim}
%   }
% \end{frame}
%
% \begin{frame}[fragile]{}
%   {\tiny
%   \begin{verbatim}
%     n = xx.size / 2
%     xnum1, ynum1, den1 = 0, 0, 0
%     (0...n).each do |i|
%       xnum1 += xx[i]
%       ynum1 += yy[i]
%       den1 += 1
%     end
%     xnum2, ynum2, den2 = 0, 0, 0
%     (n...xx.size).each do |i|
%       xnum2 += xx[i]
%       ynum2 += yy[i]
%       den2 += 1
%     end
%
%   \end{verbatim}
%   }
% \end{frame}
%
% \begin{frame}[fragile]{}
%   {\tiny
%   \begin{verbatim}
%   xavg1 = xnum1/den1
%   yavg1 = ynum1/den1
%   xavg2 = xnum2/den2
%   yavg2 = ynum2/den2
%   yavg = (ynum1 + ynum2) / (den1 + den2)
%
%   a = (yavg2 - yavg1) / (xavg2 - xavg1)
%
%   y2 = []
%   (0...xx.size).each do |i|
%     y2 << yy[i] - a*xx[i]
%   end
%
%   puts File::basename(fn) if y2.max - y2.min > 1.0
%
%   \end{verbatim}
%   }
% \end{frame}

\section{Zpracování textu}

\subsection{Obecný rozbor}

\begin{frame}{Problém č. 2: mnoho výpočtů, inženýrova smrt}
  \begin{block}{Zadání}
    Při přípravě základního kritického experimentu je pomocí MCNP potřeba najít kritickou polohu regulační tyče R2.

    Jak se tato poloha změní při změně polohy tyče R1?
  \end{block}
\end{frame}

\begin{frame}{Co máme k~dispozici?}
  \begin{block}{MCNP}
    Pokud připravíme vstupní soubor (v~netriviální formě obsahující polohy regulačních tyčí R1 a R2), spočítá nám keff.
  \end{block}
  Potřebovali bychom ale něco na:
  \begin{enumerate}
    \item vytvoření velkého množství vstupních souborů
    \item extrakci keff z~výstupních souborů
    \item popřípadě na vyhodnocení získaných poloh tyčí a keff
  \end{enumerate}
\end{frame}

\begin{frame}{Pracovní postup}
  \begin{enumerate}
    \item načíst keff z~výstupního souboru MCNP
    \pause
    \item vygenerovat potřebné vstupní soubory
    \pause
    \item vyrobit BAT soubor na spuštění výpočtů
    \pause
    \item načíst výsledky ze všech výstupních souborů do jedné tabulky
    \pause
    \item buď zpracovat ručně (Excel), nebo být Myšpulín a vyrobit skript (úkol s~hvězdičkou)
  \end{enumerate}
\end{frame}

\subsection{Načítání výstupního souboru}

\begin{frame}[fragile]{}
  Nejprve najdeme, kde je ve výstupu z~MCNP žádané keff:
  {\tiny
  \begin{verbatim}
.....
         the k(trk length) cycle values appear normally distributed at the 95 percent confidence level

-----------------------------------------------------------------------------------------------------------------------------------
|                                                                                                                                 |
| the final estimated combined collision/absorption/track-length keff = 1.00353 with an estimated standard deviation of 0.00024   |
|                                                                                                                                 |
| the estimated 68, 95, & 99 percent keff confidence intervals are 1.00329 to 1.00377, 1.00305 to 1.00400, and 1.00289 to 1.00416 |
|                                                                                                                                 |
| the final combined (col/abs/tl) prompt removal lifetime = 1.0017E-04 seconds with an estimated standard deviation of 2.7364E-08 |
.....
  \end{verbatim}
  }
\end{frame}

\begin{frame}{Algoritmus}
  \begin{enumerate}
    \item najít řádek s~keff
    \pause
    \item vytáhnout z~něj keff, takže například:
    \pause
    \item rozdělit podle rovnítka
    \pause
    \item druhou část rozdělit podle mezer
    \pause
    \item vzít první prvek
  \end{enumerate}
\end{frame}


\begin{frame}[fragile]{Realizace (1/5)}
  \scriptsize
  \begin{verbatim}
    keff = nil

    File.foreach("c1_1o") do |line|

    end

    puts keff
  \end{verbatim}
\end{frame}

\begin{frame}[fragile]{Realizace (2/5)}
  \scriptsize
  \begin{verbatim}
    keff = nil

    File.foreach("c1_1o") do |line|
      if line.include?("final estimated combined")

      end
    end

    puts keff
  \end{verbatim}
\end{frame}

\begin{frame}[fragile]{Realizace (3/5)}
  \scriptsize
  \begin{verbatim}
    keff = nil

    File.foreach("c1_1o") do |line|
      if line.include?("final estimated combined")
        a = line.split("=")

      end
    end

    puts keff
  \end{verbatim}
\end{frame}

\begin{frame}[fragile]{Realizace (4/5)}
  \scriptsize
  \begin{verbatim}
    keff = nil

    File.foreach("c1_1o") do |line|
      if line.include?("final estimated combined")
        a = line.split("=")
        b = a[1].split

      end
    end

    puts keff
  \end{verbatim}
\end{frame}

\begin{frame}[fragile]{Realizace (5/5)}
  \scriptsize
  \begin{verbatim}
    keff = nil

    File.foreach("c1_1o") do |line|
      if line.include?("final estimated combined")
        a = line.split("=")
        b = a[1].split
        keff = b[0]
      end
    end

    puts keff
  \end{verbatim}
\end{frame}


\section{Automatizace tvorby vstupů}

\begin{frame}[fragile]{Určení poloh tyčí}
  Ve vstupním souboru si najdeme relevantní část:
  \scriptsize
  \begin{verbatim}
    c ---------------------------------
    c polohy tyci (z-plochy)
    c ---------------------------------
    c
    67 pz 47.6000    $ dolni hranice absoberu r1
    68 pz 40.4980    $ dolni hranice hlavice r1
    69 pz 44.8000    $ dolni hranice absoberu r2
    70 pz 37.6980    $ dolni hranice hlavice r2
  \end{verbatim}
\end{frame}

\begin{frame}[fragile]{Výroba šablon}
  Jak dostat polohy tyčí do vstupního souboru? Vyrobíme šablonu, tzn nahradíme
  \begin{verbatim}
    67 pz 47.6000    $ dolni hranice absoberu r1
  \end{verbatim}
  \pause
  nějakou značkou (\emph{placeholder}):
  \begin{verbatim}
    67 pz %r1%    $ dolni hranice absoberu r1
  \end{verbatim}
\end{frame}

\begin{frame}{Chytáky a zádrhele}
  \begin{itemize}
    \item kromě samotné plochy konce absorbéru je nutno správně umístit i z-plochu konce hlavice o 7,102 cm níže
    \item obecně je na místě ohlídat si, že placeholder nebude kolidovat s ničím jiným
  \end{itemize}
  Doporučené nástroje jsou:
  \begin{itemize}
    \item již známá funkce \texttt{sub} pro nahrazení jednoho řetězce jiným
    \item pro pragmatické lenochy funkce \texttt{File.read} načítající celý soubor do řetězce (na což nelze v Pascalu ani pomyslet)
    \item možno ovšem použít i \texttt{File.readlines} (v čem je to lepší?)
  \end{itemize}
\end{frame}

\begin{frame}[fragile]{Realizace}
  \scriptsize
  \begin{verbatim}
    DELTA = 44.8000 - 37.6980

    template = File.read("template")
    (0..10).each do |i1|
      (0..10).each do |i2|
        r1 = i1 * 50
        r2 = i2 * 50
        s = template.sub("%r1%", r1.to_s)
        s = s.sub("%r1_%", (r1 - DELTA).to_s)
        s = s.sub("%r2%", r2.to_s)
        s = s.sub("%r2_%", (r2 - DELTA).to_s)
        File.write("inputs/c_#{i1}_#{i2}", s)
      end
    end
  \end{verbatim}
\end{frame}

\subsection{Zápis všech výsledků do tabulky}

\begin{frame}{Jak na to}
  Máme všechno, co potřebujeme:
  \begin{itemize}
    \item načtení keff z~jednoho výstupního souboru (\texttt{File.foreach}, \texttt{include} a \texttt{split})
    \item procházení adresáře (\texttt{Dir.each})
    \item zápis do souboru (\texttt{File.open} s~parametrem \texttt{w} anebo \texttt{File.write})
  \end{itemize}
  Takže už to stačí jen vhodným způsobem spojit dohromady!
\end{frame}

\begin{frame}[fragile]{Realizace}
  \scriptsize
  \begin{verbatim}
    Dir["*o"].each do |filename|
      keff = nil

      File.foreach(filename) do |line|
        if line.include?("final estimated combined")
          a = line.split("=")
          b = a[1].split
          keff = b[0]
        end
      end

      puts "#{filename} #{keff}"
    end
  \end{verbatim}
\end{frame}

\begin{frame}[fragile]{Výstup}
  Výsledkem je perfektní tabulka:
  {\scriptsize
  \begin{verbatim}
    outputs/c_0_0o 0.94800
    outputs/c_0_10o 0.99800
    outputs/c_0_1o 0.94850
    outputs/c_0_2o 0.95000
    outputs/c_0_3o 0.95250
    outputs/c_0_4o 0.95600
    ...
  \end{verbatim}}
  Hloupé je, že nikde nemáme tu polohu tyčí.
\end{frame}


\begin{frame}[fragile]{Chytrá horákyně}
  ... by jistě vyrobila toto:
  {\scriptsize
  \begin{verbatim}
    0 0 0.94800
    0 10 0.99800
    0 1 0.94850
    0 2 0.95000
    0 3 0.95250
    0 4 0.95600
    ...
  \end{verbatim}
  }
  Nápovědou je funkce \texttt{split} (podle podtržítka) a funkce \texttt{to\_i} (co asi dělá?)
\end{frame}

\begin{frame}[fragile]{Realizace chytré horákyně}
  \scriptsize
  \begin{verbatim}
    Dir["outputs/*o"].each do |filename|
      keff = nil

      File.foreach(filename) do |line|
        if line.include?("final estimated combined")
          a = line.split("=")
          b = a[1].split
          keff = b[0]
        end
      end

      s = filename.split("_")
      puts "#{s[1].to_i} #{s[2].to_i} #{keff}"
    end
  \end{verbatim}
\end{frame}

\subsection{Co dál?}

\begin{frame}{Navážeme na úspěchy z minulých týdnů}
  \begin{itemize}
    \item vykreslit graf! pro každou z 11 poloh R1 jedna čára (závislost keff na R2)
    \item (= csv soubor, gnuplot, znáte to)
    \item najít automaticky kritickou polohu R2 pro každou z 11 poloh R1
    \item a zase graf... (kritická poloha R2 v závislosti na R1)
  \end{itemize}
\end{frame}

\begin{frame}{A to je vše, přátelé!}
  \begin{center}
    \includegraphics[width=0.8\textwidth]{looney_tunes}
  \end{center}
\end{frame}

\end{document}
