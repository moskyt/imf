\documentclass{beamer}

\def\Tiny{\fontsize{6pt}{6pt}\selectfont}
\def\supertiny{\fontsize{4pt}{4pt}\selectfont}

\mode<presentation>
{
  \usetheme{Warsaw}
  % \setbeamercovered{transparent}
  \usecolortheme{crane}
}

\usepackage{graphicx, ifthen, listings, fancyvrb}

\usepackage[czech]{babel}
% \usefonttheme{professionalfonts}
\usepackage{times}
\usepackage{amsmath}
\usepackage[utf8]{inputenc}
\usepackage{wrapfig}

\usepackage[T1]{fontenc}

\lstset{ basicstyle=\tiny, stringstyle=\ttfamily, showstringspaces=false }

\everymath{\displaystyle}

\setbeamerfont{frametitle}{size=\large}
\setbeamerfont{subsection in toc}{size=\scriptsize}

\makeatletter\newenvironment{blackbox}{%
   \begin{lrbox}{\@tempboxa}\begin{minipage}{0.95\columnwidth}}{\end{minipage}\end{lrbox}%
   \colorbox{black}{\usebox{\@tempboxa}}
}\makeatother

\title[IMF (11)]{Informatika pro moderní fyziky (12)\\ dokončení klikací mapy AZ; HTML + JS mapa; zadání zápočtových úloh}

\author[Franti\v{s}ek HAVL\r{U}J, ORF ÚJV Řež]{Franti\v{s}ek HAVL\r{U}J\\{\scriptsize \emph{e-mail: haf@ujv.cz}}}

\date{akademický rok 2016/2017\\21. prosince 2016}

\institute[ORF ÚJV Řež]
{ÚJV Řež\\oddělení Reaktorové fyziky a podpory palivového cyklu}

\AtBeginSection[]
{
\begin{frame}<beamer>
\frametitle{Obsah}
\tableofcontents[currentsection,hideothersubsections]
\end{frame}
}

\begin{document}

\begin{frame}
  \titlepage
\end{frame}

\begin{frame}
  \tableofcontents
\end{frame}

\section{K zápočtovým úlohám}

\begin{frame}{Obecně:}
\begin{itemize}
  \item ke každému zadání jsou k dispozici vzorová data
  \item já to budu testovat i na datech jiných
  \item očekávám, že všechno proběhne na jedno spuštění skriptu / rake tasku
  \item každý má k dispozici jeden pokus řádný a jeden opravný
\end{itemize}
\end{frame}

\begin{frame}{Klasifikace}
  \begin{itemize}
    \item F - nejde to spustit, ani pro zadaná data to v podstatných bodech nesplňuje zadání
    \item E - pro zadaná data to funguje, ale pro jiná čísla to nechodí
    \item D - obecně to funguje, ale stejně chybí drobnosti ze zadání
    \item C - všechno funguje jak má
    \item B - funguje a navíc jsou výstupy hezké a přehledné, soubory nejsou generovány “na velkou hromadu”, ale roztříděné do složek apod.
    \item A - kromě výše uvedeného jsou splněny i požadavky formy (správné odsazování, rozumná jména funkcí a proměnných) a efektivity (je to rozumně naprogramované - vhodné použití funkcí, datových struktur atd.)
  \end{itemize}
\end{frame}

\begin{frame}{Známka se snižuje o stupeň, pokud:}
  \begin{itemize}
    \item jsou někde ve skriptech použity absolutní cesty, takže je budu muset upravovat (výjimkou jsou cesty k programům jako např. gnuplot, které ovšem musí být umístěny v proměnné někde na začátku skriptu (abych to nemusel lovit)
    \item bude v kódu něco, co limituje použití na OS Windows (backslash v cestě, kódování win1250 atd.)
  \end{itemize}
  (a podobně)
\end{frame}

\section{Dokončení z minula}

\begin{frame}{Zadání dnešní úlohy}
  \begin{itemize}
    \item každý den data z 1-9 detektorů (\texttt{data/*.csv})
    \item detektor má svoji polohu v AZ (VR-1 Vrabec, 8x8 čtvercových pozic) -- včetně data je uvedena na prvním řádku CSV souboru
    \item je potřeba hezky zobrazit na každý den mapu AZ a grafy signálů z detektorů
    \item viz \texttt{html/document.html}
  \end{itemize}
\end{frame}

\section{Interaktivní mapa}

\begin{frame}{Jednoduchý mashup: mapa o-závodů}
  \begin{itemize}
    \item ORIS API -- http://oris.orientacnisporty.cz/API
    \item úkol: vypišme kalendář MTBO závodů v roce 2016
    \item {\tiny \url{http://oris.orientacnisporty.cz/API/?format=json\&method=getEventList\&sport=3\&datefrom=2016-01-01\&dateto=2016-12-31}}
    \item {\tiny \url{http://api.mapy.cz}}
  \end{itemize}
\end{frame}

\begin{frame}{Klasický postup: upravovat vzor}
  \begin{itemize}
    \item {\tiny \url{http://api.mapy.cz/view?page=instruction}}
    \item {\tiny \url{http://api.mapy.cz/view?page=markerlayer}}
  \end{itemize}
\end{frame}

\begin{frame}{A to je vše, přátelé!}
  \begin{center}
    \includegraphics[width=0.8\textwidth]{looney_tunes}
  \end{center}
\end{frame}

\end{document}
